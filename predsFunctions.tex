\section{Functions}
\label{sec:preds-and-fns}

The semantics of predicates (i.e., functions returning booleans) and
(non-aggregate) functions in PartiQL are identical to those of SQL when their
inputs are those that are allowed by SQL. PartiQL makes the following extensions
for the cases where the inputs are beyond those allowed by SQL.
 
\subsection{Inputs with wrong types:}
\label{sec:fns-with-wrong-inputs}

\almann{TODO confirm that this is implemented correctly in the implementation}

Unlike SQL where typing issues can be detected during query compilation, the
permissive option of PartiQL has to define semantics for the case where the
inputs of a function are not compatible with the function/predicate arguments.
Furthermore, PartiQL facilitates propagating missing input attributes to
respective missing output attributes.

Alike SQL, all functions have input argument types that they conform to. For example, the
function \gl{log} expects numbers. All functions return \MISSING when they
input data whose types do not conform to the input argument types. Since no
function (other than \gl{IS MISSING}) has \MISSING as an input argument
type, it follows that all functions return \MISSING when one of their inputs
is \MISSING.

\begin{example} 
\label{xmpl:missing-on-wrong-types}

The query

\begin{tabbing}
\gl{SELECT VALUE \{'a':3*v.a, 'b':3*(CAST v.b AS INTEGER)\}}\\
\gl{FROM [\{'a':1, 'b':'1'\}, \{'a':2\}] v}
\end{tabbing}

\noindent results into \gl{\ob \{'a':3, 'b':3\}, \{'a':6\} \cb}. Notice how the
missing \gt{b} attribute in the input leads to a respective missing attribute in
the output.
\end{example}

\begin{example}
Each one of these expressions returns \MISSING: \gt{5 + missing}, \gt{5 >
'a'}, \gt{NOT \{a:1\}}.
\end{example}

\subsubsection{Equality}
\label{sec:equality}
Equality never fails in the type-checking mode and never returns \MISSING in
the permissive mode. Instead, it can compare values of any two types, according
to the rules of the PartiQL type system. For example, \gt{5 = 'a'} is
\gl{false}.

Since PartiQL variables may bind to composite values (collections, tuples),
PartiQL extends the semantics of equality for these cases. In particular,
equality in PartiQL is {\em deep equality}, defined as follows: 

\begin{enumerate}
\item Given two arrays $x$ and $y$ that have the same length $l$, the result of
$x=y$ is the result of 
\[\gt{eqg}(x[0], y[0])\ \gt{AND}\ \ldots\ \gt{AND}\ \gt{eqg}(x[l], y[l])\]
The \gt{eqg}, unlike the \gt{=}, returns true when a \NULL is compared to a \NULL 
or a \MISSING to a \MISSING. When the arrays $x$ and
$y$ do not have the same length, the $x=y$ is \gt{false}.

\almann{Yannis, it is a bit of a tragedy, that we cannot use equality for
nested nulls/missing--the implementation does this for arrays/tuples/bags.\\
I see no reason to make \NULL and \MISSING short-circuit in recursive
equality--yes it is less orthogonal, but really not useful.\\
Yannis: Fixed. I changed to using the \gl{eqg}. Fixed the following examples also.}

\item A similar straightforward equality applies to tuples: 
They have to have the same attributes. Then equality $t_1 =
t_2$ is true if 
\[\gt{eqg}(t_1.a_, t_2.a_1)\ \gl{AND}\ \ldots\ \gl{AND}\ \gt{eqg}(t_1.a_n, t_2.a_n)\]
where $a_1, \ldots, a_n$ are the attributes that appear in $t_1$ and
$t_2$.%

\almann{TODO define this in terms of duplicate attribute names/values.
Yannis: Are we going too far with duplicate attribute names?
I thought the idea was to let them happen, let them propagate
but do not redefine the language primitives to accomodate this possibility.
I can see hard time supporting it beyond pass-thru.
}
 
\item Equality for bags is similarly straightforward: Two bags $x$ and $y$ are
equal if and only if every element $e$ of $x$ that appears $n$ times in $x$ also
appears $n$ times in $y$.
\end{enumerate}

\begin{example}

The following are true:

\begin{tabbing}
\ \ \ \=\gt{\ob 3, 2, 4, 2 \cb = \ob 2, 2, 3, 4 \cb}\\
\>\gt{\{'a':1, 'b':2\} = \{'b':2, 'a':1\}}\\
\>\gt{\{'a':[0,1], 'b':2\} = \{'b':2, 'a':[0,1]\}}\\
\end{tabbing}

The following are false:

\begin{tabbing}
\ \ \ \=\gt{\ob 3, 4, 2 \cb = \ob 2, 2, 3, 4 \cb}\\
\>\gt{\{'a':1, 'b':2\} = \{'a':1\}}\\
\>\gt{\{'a':[0,1], 'b':2\} = \{'b':2, 'a':[0,1,2]\}}\\
\end{tabbing}

The following are also false. 

\begin{tabbing}
\ \ \ \=\gt{\{'a':1, 'b':2\} = \{'a':1\}}\\
\>\gt{\{'a':1, 'b':2\} = \{'a':1, 'b':null\}}\\
\>\gt{\{'a':[0,1], 'b':2\} = \{'b':2, 'a':[null,1]\}}\\
\end{tabbing}
\end{example}

