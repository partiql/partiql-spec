\section{\gl{ORDER BY} clause}
\label{section:order-by}
SQL's \gl{ORDER BY} orders the output data. Similarly, the PartiQL \gl{ORDER BY}
is responsible for turning its input bag into an array. In the following
aspects, PartiQL extends the SQL semantics to resolve issues that are not
relevant in SQL but emerge when working on Ion data.
%
\begin{enumerate}
%
\item SQL's \gl{ORDER BY} clause orders its input using an expanded version of
the less-than function, which we call the {\em order-by less-than} and denote by
$\order$. The PartiQL $\order$ semantics (Section~\ref{sec:order-by-less-than})
also specify an order among values of heterogeneous types, including complex
values.
%
\item The interaction of \gl{ORDER BY} with a \gl{UNION} (or any other set
operator) of SFW queries requires attention since, unlike SQL, in PartiQL there
are no binding tuples (or any tuples at all for that matter) after a \gl{SELECT
VALUE} clause. Section~\ref{sec:order-by-and-setops} elaborates on this aspect
of PartiQL.
%
\item Unlike SQL, the input of an PartiQL query may also have order, because it
is an array. The user may want to preserve the order of the input into the
output. In this case, the \gl{AT} structure in the \gl{FROM} clause (recall,
Section~\ref{sec:single-item-from}) can capture the input order and the
\gl{ORDER BY} can recreate it. However, this order preservation mechanism is
tedious for the user. Thus, \gl{ORDER BY} also offers an order preservation
directive (Section~\ref{sec:order-preservation}).
%
\end{enumerate}

Sections~\ref{sec:orderby-sql-compatibility}
and~\ref{sec:select-variables-in-order} discuss SQL compatibility issues.


\subsection{PartiQL \gl{ORDER BY} Syntax}
\label{sec:orderby-syntax}
Similar to SQL, the PartiQL \gl{ORDER BY} clause syntax is: 
\[
\begin{array}{ll}
\gl{ORDER BY }( & e_1\ [\gl{ASC} | \gl{DESC}]?\ [\gl{NULLS FIRST}|\gl{NULLS LAST}]? \\
 & \vdots \\
 & e_m\ [\gl{ASC} | \gl{DESC}]?\ [\gl{NULLS FIRST}|\gl{NULLS LAST}]? \\
 &) \\
 &| \gl{PRESERVE}
\end{array}
\] 
\noindent (Figure~\ref{figure:query:bnf}), where $e_1 \ldots e_m$ is a list of
\textit{ordering expressions}.  In PartiQL a SFW query with \gl{ORDER BY}
outputs an array, whereas a SFW query without \gl{ORDER BY} outputs a bag. 

Alike SQL's \gl{ORDER BY} clause, the \gl{NULLS FIRST} and \gl{NULLS LAST}
keywords indicate whether \NULL and \MISSING values are ordered before
or after all other values. Notice that in PartiQL, the \gl{NULLS FIRST} and
\gl{NULLS LAST} refer to both \NULL and \MISSING.


\subsection{The PartiQL order-by less-than function}
\label{sec:order-by-less-than}
The \gl{ORDER BY} clause sorts its input using the \textit{order-by less-than
function} $\order$, which is able to compare values of different types (unlike
SQL). In particular:

\almann{Yannis, what about SEXP type here, it exists in Ion and in our PartiQL
implementation?  Also, I checked, this is exactly aligned with the definition I
implemented in the codebase so that is good.}

\begin{enumerate}
\item \gl{NULL} and \gl{MISSING} are always first or last and compare equally
according to $\order$. In other words, $\order$ cannot distinguish between
\gl{NULL} and \gl{MISSING}.
\item The boolean values are coming first among the non-absent values (i.e., $b
\order x$ is always true if $b$ is boolean and $x$ is not a \gl{NULL} or a
\gl{MISSING} or a boolean).  \gl{false} comes before \gl{true}.
\item The numbers come next. The comparisons between number values do not depend
on precision or specific type. Given two numbers $x$ and $y$, the PartiQL $x
\order y$ behaves identical to the SQL order-by less-than function. Namely, if
$x$ and $y$ are not the special values \gl{`-inf'}, \gl{`inf'} or \gl{`nan'},
then $x \order y$ is the same with $x \gl{<} y$. The special value \gl{`nan'}
comes before \gl{`-inf'}, which comes before all normal numeric values, which
are followed by \gl{`+inf'}.
\item Timestamp values follow and are compared by the absolute point of time
irrespective of precision or local UTC offset.
\item The text types come next ordered by their lexicographical ordering by
Unicode scalar irrespective of their specific type.
\item The LOB types follow and are ordered by their lexicographical ordering by
octet.
\item Arrays come next, and their values compare lexicographically based on the
comparison of their elements, recursively. Notice that given an array $[e_1,
\ldots, e_m]$ and a longer array $[e_1, \ldots, e_m, e_{m+1}, \ldots, e_n]$ that
has the same first $m$ values, the former array comes first.
\item Tuple values follow and compare lexicographically based on the sorted
attributes (as defined recursively), first by the attribute name, and secondly
by the attribute values themselves.
\item Bag values come last (except, of course, when \gl{NULLS LAST} is
specified) and their values compare by first reducing them to arrays by sorting
their elements and then comparing the resulting arrays.
\end{enumerate}

\subsection{Dependency of \gl{ORDER BY} semantics on the Presence of Set Operators}
\label{sec:order-by-and-setops}
Coming up...
\eat{
Similar to SQL, the presence of bag/set operators (such as \gl{UNION} and
\gl{UNION ALL}) in a SFW query results in different semantics for \gl{ORDER BY}
and, hence, PartiQL itemizes the semantics depending on the presence or absence
of set operators.

\subsubsection{SFW without bag/set operators} 
\label{sec:orderby-without-setops}
The \gl{ORDER BY} clause inputs a bag of binding tuples $B_{\gl{ORDER}}^{in}$,
thus each ordering expression $e_i$ can utilize variables from the preceding
\gl{FROM} or \gl{GROUP BY} clauses. The \gl{ORDER BY} clause outputs an array of
sorted binding tuples $B_{\gl{ORDER}}^{out}$.  The semantics are similar to
SQL's. Formally: 

Let $b_1, \ldots, b_n$ be the binding tuples in $B_{\gl{ORDER}}^{in}$, and for
each $b_j \in B_{\gl{ORDER}}^{in}$, let $v_{j,1} \ldots v_{j,m}$ be the results
of evaluating of the ordering expressions $e_1 \ldots e_m$. The order of the
tuples in the output depends on the comparison according to $\order$. Formally,
given two input binding tuples $b_j$ and $b_k$, $b_j$ appears before $b_k$ in
$B_{\gl{ORDER}}^{out}$ if $[v_{j,1} \ldots v_{j,m}] \order [v_{k,1} \ldots
v_{k,m}]$. If neither $[v_{j,1} \ldots v_{j,m} \order [v_{k,1} \ldots v_{k,m}]$
nor $[v_{k,1} \ldots v_{k,m} \order [v_{j,1} \ldots v_{j,m}]$ PartiQL
nondeterministically decides the order between $b_j$ and $b_k$. Sorting in
descending order is defined analogously.


\subsubsection{\gl{ORDER BY} with bag/set operators} 
\label{sec:order-with-set}
Consider SFW queries of the form
\[ q_1 \gl{ UNION } q_2 \gl{ ORDER BY } e_1 \ldots e_m \] \noindent where $q_1$
and $q_2$ are also SFW queries. (The specification generalizes in the obvious
way to include \gl{NULLS LAST} and/or \gl{DESC}.) As expected from SQL (and will
be further detailed for PartiQL in Section~\ref{section:setop}), the \gl{UNION}
clause is evaluated after the individual SFW queries $q_1$ and $q_2$. It outputs
a bag of values, which is in turn input by \gl{ORDER BY}. Thus, in the presence
of bag/set operators, instead of binding tuples, the \gl{ORDER BY} clause inputs
a bag of values $C = \gl{<<} u_1 \gl{, } \ldots \gl{, } u_n \gl{>>}$ and
outputs an array of sorted values. 

The ordering expressions $e_1 \ldots e_m$ use the special \gl{CURRENT} variable
to range over the elements (values) of their input bag $C$. Formally, suppose
the query is evaluated within environment $\env$. For each $u \in C$, let $v_1
\ldots v_m$ be the evaluation of the ordering expressions $e_1 \ldots e_m$
within environment $\env' = \langle \gl{CURRENT} : u \rangle || \env$. Given two
input element values $u_j$ and $u_k$, the \gl{ORDER BY} clause orders them by
comparing the results of the ordering expressions when evaluated in the
environment $\langle \gl{CURRENT} : u_j \rangle || \env$ with the results of the
ordering expressions in $\langle \gl{CURRENT} : u_k \rangle || \env$.

\begin{example}
The following query unions the results of two queries and then orders them
according to modulo 10.

\begin{tabbing}
\ \ \=\gl{(}\=\gl{SELECT VALUE }\texttt{x}\\
\>\>\gl{FROM }\texttt{[25, 11, 9] x }\gl{)} \\
\>\gl{UNION}\\
\>\>\gl{(}\=\gl{SELECT VALUE }\texttt{y}\\
\>\>\gl{FROM }\texttt{[3, 10] y }\gl{)} \\
\>\gl{ORDER BY mod(CURRENT, }\texttt{10}\gl{)}\\ 
\end{tabbing}

\noindent The result is the array \texttt{[10, 11, 3, 25, 9]}.
\end{example}


\subsection{Automatic order preservation}
\label{sec:order-preservation}
PartiQL also provides the \gl{ORDER BY PRESERVE} clause in lieu of the usual
\gl{ORDER BY} that uses expressions to order. The \gl{ORDER BY PRESERVE} is
applicable only when there is no \gl{GROUP BY} clause.

It orders the output elements according to the order of the respective input
elements. Technically it is the syntactic sugar \textbf{(S)} that is reduced to
core PartiQL \textbf{(C)} as shown below. Only the \gl{FROM} and \gl{ORDER}
clauses are shown, as the \gl{SELECT} and \gl{WHERE} are identical and \gl{GROUP
BY} is inapplicable. ~\\
\begin{tabular}{@{}l@{~}l@{~}l@{~}l@{~}l@{~}l@{}}
\textbf{(C)}    & \gl{FROM}     & $e_1$ \gl{AS} $v_1$ \gl{AT} $p_1,$&
\textbf{(S)}   & \gl{FROM}     & $e_1$ \gl{AS} $v_1,$  \\
    &                   & $\ldots$                                  &       &
    & $\ldots$                  \\
    &                   & $e_n$ \gl{AS} $v_n$ \gl{AT} $p_n$ &       & \gl{FROM}
    & $e_n$ \gl{AS} $v_n$   \\
    & \gl{ORDER BY} & $p_1, \ldots, p_n$                        &       &
    \gl{ORDER BY}    & \gl{PRESERVE}         \\
\end{tabular} \\

\highlight{Mistyping cases} It is possible that an \gl{AT} is introduced while
the \gl{FROM} clause iterates over one or more bags, i.e., over collections
whose elements have no order. In this case, the \gl{ORDER BY} will produce a
non-deterministic order. 

\begin{example}
Consider the following \gl{FROM} clause, where \gl{x} iterates over an array but
\gl{y} iterates over a bag.
\begin{tabbing}
\ \ \ \=\gl{SELECT x.a AS a, y.b AS b}\\
\>\gl{FROM [3, 2] AS x, \ob 4, 5 \cb\ AS y}\\
\>\gl{ORDER BY PRESERVE}
\end{tabbing}
It is equivalent to 
\begin{tabbing}
\ \ \ \=\gl{SELECT VALUE [x, y]}\\
\>\gl{FROM [3, 2] AS x AT xp, \ob 4, 5 \cb AS y AT yp}\\
\>\gl{ORDER BY xp, yp}
\end{tabbing}
The result is an array. One possible result is
\begin{tabbing}
\ \ \ \=\gl{[ [3, 4], [3, 5], [2, 5], [2, 4] ]}
\end{tabbing}
Another possible result is 
\begin{tabbing}
\ \ \ \=\gl{[ [3, 4], [3, 5], [2, 4], [2, 5] ]}
\end{tabbing}
Indeed, there are four possible results: The 3's are always ahead of the 2's;
all possible permutations of 4's and 5's are possible.
\end{example}
}

\subsection{SQL Compatibility \gl{ORDER BY} clauses}
\label{sec:orderby-sql-compatibility}

For SQL-compatibility, PartiQL allows the \gl{CURRENT} variable to be omitted
from ordering expressions. Then when the \gl{CURRENT} variable binds tuples, the
ordering expressions can refer directly to the attributes of those tuples.

\yannis{We have generally triggered rewritings like this by the presence of
schema. However, the ORDER BY is guaranteed to have single input (the bag $C$)
and thus it won't be plagued by the semantics, efficiency and stability problems
that led us to restrict the expansion of the environment in the case of FROM
clause items to schemaful variables. Thus I allowed it to not be
schema-specific. Plus, ORDER BY without this syntactic sugar would require us to
teach ``CURRENT" to even first time users.}

\almann{This is true except in the case of nested scopes, should we restrict
this similarly to the single FROM clause rewrites?\\
Put another way, I think this has to be schema-ful, but fortunately this is not
an issue with typical use of \lstinline|(*q1*) UNION (*q2*) ORDER BY (*e*)|
since $q1$ and $q2$ if they are SFW will have implicit schema and thus can be
easily explained.}

The complete scoping rules are as follows. When all of the following conditions
are satisfied:
\begin{enumerate}
\item an PartiQL path expression ordering expression $as$ appears in the
\gl{ORDER BY} of a \gl{UNION ... ORDER BY} query, where $a$ is an identifier and
$s$ is the potentially empty suffix of the path.
\item the expression $as$ is evaluated in database environment $\db$ and
variables' environment $\env$, which defines variables $v_1,\ldots, v_n$ and
none of them is named $a$. 
\item none of the variables $v_1,\ldots, v_n$ may bind to a tuple that has an
attribute $a$.
\end{enumerate}
\noindent then the path expression $as$ resolves to $\gl{CURRENT}.as$. 

The most common and useful way to have the 3rd condition be satisfied is when
the \gl{UNION ... ORDER BY} is a top-level query and, thus, the variables
environment $\env$ is empty.

\subsection{Use of \gl{SELECT} variables in \gl{ORDER BY} for SQL compatibility}
\label{sec:select-variables-in-order}
Recall from Section~\ref{section:environment-and-sfw} that \gl{ORDER BY} is
evaluated before \gl{SELECT}. For SQL-compatibility, given $\gl{SELECT } e_i
\texttt{ AS } a_i$, PartiQL also supports the syntactic sugar of using $a_i$ in
lieu of $e_i$ in the \gl{ORDER BY} clause. Therefore, both SFW queries below are
equivalent: ~\\
\begin{tabular}{@{}l@{~}l@{~}l@{~}l@{~}l@{~}l@{}}
(1) & \texttt{SELECT}   & $e_i$ \texttt{ AS } $a_i$   & (2)   & \texttt{SELECT}
& $e_i$ \texttt{ AS } $a_i$   \\
    & \texttt{FROM}     & \ldots                      &       & \texttt{FROM}
    & \ldots                      \\
    & \texttt{ORDER BY} & $a_i$                       &       & \texttt{ORDER
    BY} & $e_i$                       \\
\end{tabular} \\

\subsection{Coercion of literals for SQL compatibility}
\label{sec:literal-conversion}
\yannis{TO DO: move to comparisons. It's not an ORDER BY issue.}

Notice that definition of \gl{<} dismissed the SQL coercions. In SQL, given
explicit literals in a query, coercions may happen.

\begin{example}
The query
\begin{verbatim} 
SELECT * FROM foo WHERE 9 < '10'
\end{verbatim}
is equivalent to
\begin{verbatim} 
SELECT * FROM foo WHERE 9 < 10
\end{verbatim}
because an automatic coercion of string to number will be introduced. 
\end{example}

This aspect of SQL compatibility is introduced by rewriting. Namely, given a
query with incompatible types 
