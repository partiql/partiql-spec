\section{\select\ clauses}
\label{sec:select-values}

Core PartiQL SFW queries have a \gl{SELECT VALUE} clause (in lieu of SQL's
\gl{SELECT} clause) that can create outputs that are collections of anything
(e.g., collections of tuples, collections of scalars, collections of arrays,
collections of mixed type elements, etc.) Section~\ref{sec:select-values-core}
describes the \gl{SELECT VALUE} clause.

SQL's well-known \gl{SELECT} clause can be used as a mere syntactic sugar over
\gl{SELECT VALUE}, when we consider the top-level query. In particular,
Section~\ref{sec:sql-select} shows that SQL's \gl{SELECT} is the special case
where the \gl{SELECT VALUE} produces collections of tuples. Furthermore, when
\select is used as a subquery it is coerced into a scalar or a tuple, in the
ways that SQL coerces the results of subqueries.

Section~\ref{sec:pivot} describes \gl{PIVOT}, which can be used
instead of \gl{SELECT VALUE}. \gl{PIVOT} creates a tuple, with a data
dependent number of attribute/value pairs, where not only the values but the
attributes as well could be originating from the data found in the binding
tuples.

\subsection{\select \values core clause}
\label{sec:select-values-core}
The \select \values clause inputs a bag of binding tuples or an array of
binding tuples (from the other clauses of the SQL query) and outputs a bag or an
array. For example, if the query only has \select \values, \from, and
\gl{WHERE} clauses, then the bindings that are output by the \gl{WHERE} clause
are input by the \select \values clause. Unlike SQL, the output of a \select
\values clause need not be a bag or array of tuples. It is a bag or array of
any kind of PartiQL values. For example, it may be a bag of integers, or a
bag of arrays, etc. Indeed, the values may be heterogeneous. For example, the
output may even be a bag that has both integers and arrays. 

The core PartiQL clause:

\begin{lstlisting}
SELECT VALUE (*$e$*)
\end{lstlisting}

\noindent inputs a bag or an array (depending on the presence or non-presence of
\gl{ORDER BY}) of binding tuples and outputs respectively a bag or an array of
values. Let $\db$ and $\env$ be the environments of the SFW query. For each
input binding tuple $b \in B^{in}_{\gl{SELECT}}$, \gl{SELECT VALUE} outputs a
value $v$, where $\db, (\env \| b) \vdash e \rightarrow v$. Notice that PartiQL
expressions $e$ (Figure~\ref{figure:query:bnf} \linequery{expression query}) will typically be
tuple or array or bag constructors \linequery{constructors}, 
which enable the construction of respective results.
In general $e$ can be any expression.

\begin{example}
This example illustrates a \gl{SELECT VALUE} that creates a collection of
numbers.

\begin{lstlisting}
SELECT VALUE 2*x.a
FROM [{'a':1}, {'a':2}, {'a':3}] as x
\end{lstlisting}

The result is \lstinline|<<2, 4, 6>>|.
\end{example}

\subsubsection{Tuple constructors} 
\label{sec:tuple-constructor}

A \textit{tuple constructor} is of the form

\begin{lstlisting}
{(*$a_1$*):(*$e_1$*), (*$\ldots$*), (*$a_n$*):(*$e_n$*)}
\end{lstlisting}

\noindent whereas $a_1 \ldots a_n, e_1 \ldots e_n$ are expressions, potentially
being themselves constructors.

\begin{example}
The query:

\begin{lstlisting}
SELECT VALUE {'a':v.a, 'b':v.b}
FROM [{'a':1, 'b':1}, {'a':2, 'b':2}] AS v
\end{lstlisting}

\noindent results into \lstinline|<<{'a':1, 'b':1}, {'a':2, 'b':2}>>|.
\end{example}

\paragraph{Treatment of mistyped attribute names}

It is possible that an expression $a_i$ that computes an attribute name results
into a non-string, i.e., a value that is not a legitimate attribute name. In
such cases, under the permissive mode the attribute-value pair will be
dismissed. Under the type checking mode the query will fail.

\begin{example}
In the permissive mode, the query:

\begin{lstlisting}
SELECT VALUE {v.a: v.b}
FROM [{'a':'legit', 'b':1}, {'a':400, 'b':2}] AS v
\end{lstlisting}

\noindent results into \lstinline|<<{'legit':1}, {}>>|. Notice that the attempt
to create an attribute named \gt{400} failed, thus leading to a tuple with no
attributes.
\end{example}

\paragraph{Treatment of duplicate attribute names}

It is possible that the constructed tuples contain twice or more the same
attribute name. 

\begin{example}
The query:

\begin{lstlisting}
SELECT VALUE {v.a: v.b,  v.c: v.d}
FROM [{'a':'same', 'b':1, 'c':'same', 'd':2}] AS v
\end{lstlisting}

\noindent results into \lstinline|<<{'same':1, 'same':2}>>|. 
Recall, a \texttt{same} path will only pick one of the two values. 
\end{example}

\subsubsection{Array Constructors} 
\label{sec:array-constructor}

An array constructor has the form:

\begin{lstlisting}
[(*$e_0$*), (*$\ldots$*), (*$e_{n-1}$*)]
\end{lstlisting}

\noindent where $e_1 \ldots e_{n-1}$ are expressions. Notice that the arrays
produced by such constructor will always have size $n$.

\begin{example}
The query:

\begin{lstlisting}
SELECT VALUE [v.a, v.b]
FROM [{'a':1, 'b':1}, {'a':2, 'b':2}] AS V
\end{lstlisting}

\noindent results into \lstinline|<<[1, 1], [2, 2]>>| 
\end{example}

In the interest of compatibility to SQL, PartiQL also allows array constructors
to be denoted with parentheses instead of brackets, when there are at least two
elements in the array, i.e., $n \geq 2$:

\begin{lstlisting}
((*$e_0$*), (*$\ldots$*), (*$e_{n-1}$*))
\end{lstlisting}

See Section~\ref{sec:select-coercion-array} for uses of this feature in SQL
compatibility.

\subsubsection{Bag Constructors}
A bag constructor has the form:

\begin{lstlisting}
<<(*$e_0$*), (*$\ldots$*), (*$e_{n-1}$*)>>
\end{lstlisting}

\noindent where $e_1 \ldots e_n$ are expressions.

\begin{example}
The query:

\begin{lstlisting}
SELECT VALUE <<v.a, v.b>>
FROM [{'a':1, 'b':1}, {'a':2, 'b':2}] AS v
\end{lstlisting}

\noindent results into \lstinline|<< <<1, 1>>, <<2, 2>> >>|. 
\end{example}

\subsubsection{Treatment of \MISSING in \gl{SELECT VALUE}}
\label{sec:treatment-missing-select-value}

\MISSING may behave differently from \NULL and differently from scalars.
The following itemizes the behavior of \MISSING in a number of cases:

\begin{itemize}
\item \textbf{when constructing tuples} Whenever during tuple construction an
attribute value evaluates to \MISSING, then the particular attribute/value is
omitted from the constructed tuple.

\begin{example} The query

\begin{lstlisting}
SELECT VALUE {'a':v.a, 'b':v.b}
FROM [{'a':1, 'b':1}, {'a':2}]
\end{lstlisting}

\noindent results into \lstinline|<<{'a':1, 'b':1}, {'a':2}>>|.
\end{example}

\item \textbf{when constructing arrays} Whenever an array element evaluates to
\MISSING, the resulting array will contain a \MISSING.

\begin{example}
The query

\begin{lstlisting}
SELECT VALUE [v.a, v.b]
FROM [{'a':1, 'b':1}, {'a':2}]
\end{lstlisting}

\noindent results into \lstinline|<<[1, 1], [2, MISSING]>>|.
\end{example}

Upon output serialization the \MISSING will convert to the symbol that the
serialization has chosen for serializing \MISSING.

\item \textbf{when constructing bags} Whenever an element of a bag evaluates to
\MISSING, the resulting bag will contain a corresponding \MISSING. 

\begin{example}
The query

\begin{lstlisting}
SELECT VALUE v.b
FROM [{'a':1, 'b':1}, {'a':2}]
\end{lstlisting}

\noindent results into \lstinline|<<1, MISSING>>| because \lstinline|{'a':2}.b|
evaluated to \MISSING.
\end{example}

\begin{example}
The query

\begin{lstlisting}
SELECT VALUE <<v.a, v.b>>
FROM [{'a':1, 'b':1}, {'a':2}]
\end{lstlisting}

\noindent results into \lstinline|<< <<1, 1>>, <<2, MISSING>> >>|.
\end{example}
\end{itemize}

\subsection{Pivoting a Collection into a Variable-Width Tuple}
\label{sec:pivot}
The \gl{PIVOT} clause may appear in lieu of \gl{SELECT VALUE}. The
\gl{PIVOT} clause outputs a tuple; in contrast, a \gl{SELECT VALUE}
outputs a collection (bag or array). The syntax is

\begin{lstlisting}
PIVOT (*$e_v$*) AT (*$e_a$*) (*$c$*)
\end{lstlisting}

\noindent where the other clauses, $c$, are the usual \gl{FROM}, \gl{WHERE},
etc. The semantics are similar to \gl{SELECT VALUE}. Let $\db$ and $\env$ be the
environments of the SFW query. For each input binding tuple $b \in
B^{in}_{\gl{PIVOT}}$, \gl{PIVOT} outputs an attribute name/value pair $a, v$,
where the name $a$ is the result of $e_a$ and the value $v$ is the result of
$e_v$. (Technically, $\db, (\env \| b) \vdash e_a \mapsto a$ and $\db, (\env \|
b) \vdash e_v \mapsto v$.) Regardless of whether $B^{in}_{\gl{PIVOT}}$ is a bag
(i.e., the SFW query did not have an \gl{ORDER BY}) or an array (i.e., the SFW
query had an \gl{ORDER BY}), the output tuple is unordered. Schema may be
applied extantly to obtain an ordered tuple.

\begin{example}
The query:

\begin{lstlisting}
PIVOT t.price AT t.symbol
FROM [{'symbol':'tdc', 'price': 31.52}, {'symbol': 'amzn', 'price': 840.05}] AS t
\end{lstlisting}

\noindent results into the tuple \lstinline|{'tdc':31.52, 'amzn':840.05}|.
\end{example}

The treatment of \MISSING is same to the treatment of \MISSING by \select
\values (Section~ref{{sec:tuple-constructor}}). Namely, whenever an attribute
name or attribute value evaluates to \MISSING, the corresponding attribute
name/value pair will not appear in the tuple.

\begin{example}
The query

\begin{lstlisting}
PIVOT t.price AT t.symbol
FROM [{'symbol':25, 'price':31.52}, {'symbol':'amzn', 'price':840.05}] AS t
\end{lstlisting}

\noindent results into the tuple \lstinline|{'amzn': 840.05}| since
\lstinline|25| is not a legitimate attribute name.
\end{example}

\subsection{SQL \select list as Syntactic Sugar of \select \values}
\label{sec:sql-select}

\subsubsection{\select Without \gl{*}}
\label{sec:select-without-star}

The SQL syntax:

\begin{lstlisting}
SELECT (*$e_1$*) AS (*$a_1$*), (*$\ldots$*), (*$e_n$*) AS (*$a_n$*)
\end{lstlisting}

\noindent is syntactic sugar for:

\begin{lstlisting}
SELECT VALUE {(*$a'_1$*):(*$e_1$*), (*$\ldots$*), (*$a'_n$*):(*$e_n$*)}
\end{lstlisting}    

\noindent whereas if the attribute name $a_i$ is written as an identifier (e.g.,
\lstinline|a| or \lstinline|"a"|) it is replaced by a single-quoted form $a'_i$
(e.g., \lstinline|'a'|).

When the expression $e_i$ is of the form $e_i'\gl{.}n$ (i.e. a path that
navigates into tuple attribute $n$), PartiQL follows SQL in allowing the
attribute name to be optional. In this case, 

\begin{lstlisting}
SELECT (*$\ldots e_i$*).(*$n \ldots$*)
\end{lstlisting}

\noindent is equivalent to 

\begin{lstlisting}
SELECT (*$\ldots e_i$*).(*$n \ldots$*) AS (*$n$*)
\end{lstlisting}

In the case that the expression $e_i$ is not of the form $e_i'\gl{.}n$ the
clause:

\begin{lstlisting}
SELECT (*$\ldots e_i$ \ldots*)
\end{lstlisting}

\noindent is equivalent to 

\begin{lstlisting}
SELECT (*$\ldots e_i$*) AS (*$a_i \ldots$*)
\end{lstlisting}

\noindent where $a_i$ is a system-generated name. SQL and PartiQL do not
provide a standard convention. 

\subsubsection{SQL's \gl{*}} 
\label{sec:sql-star}

Consider a query whose \from\ defines a variable $x$ that has no schema and the
\select\ clause includes at least one $x.\gl{*}$. Let us first consider the
simpler case where the \gl{SELECT} clause is a single item $x$\gl{.*}. Then the
clause

\begin{lstlisting}
SELECT (*$x$*).*
\end{lstlisting}

\noindent reduces to

\begin{lstlisting}
SELECT VALUE CASE WHEN NOT (*$x$*) IS TUPLE THEN {'_1': (*$x$*)} ELSE (*$x$*) END 
\end{lstlisting}

\noindent Notice that PartiQL extends the \gl{.*} to also operate on $x$
bindings that are not tuples. These are converted to singleton tuples with a
synthetic name.

\begin{example}
The query

\begin{lstlisting}
SELECT x.*
FROM [{'a':1, 'b':1}, {'a':2}, 'foo'] AS x
\end{lstlisting}

\noindent results into \lstinline|<< {'a':1, 'b':1}, {'a':2}, {'_1':'foo'} >>|.
Notice that the input has a non-tuple that was converted to a tuple with a
synthetic attribute name \lstinline|_1|, this is because the result of a
traditional \gl{SELECT} is always a container of tuples.
\end{example}

We generalize the semantics of a \gl{SELECT} list, where at least one of the
items is a \gl{.*} item, we use the function \tupleunion. When all of $t_1, t_2,
\ldots, t_n$ are tuples $\tupleunion(t_1, t_2,\ldots,t_n)$ outputs a tuple $t$
such that for each attribute name/value pair $n:v$ of any $t_i$, the tuple $t$
has a respective $n:v$. Notice the possibility that the output $t$ has duplicate
attribute names because either (i) two different inputs $t_i$ and $t_j$ had the
same attribute name, or (ii) because an input $t_i$ already had a duplicate
attribute name. 

Using \tupleunion, we rewrite the \select clause as illustrated by the following
example, which has two \gl{.*} items and one conventional item. The
generalization to more items, of either kind should be obvious. Notice that if
$v_1$ (resp. $v_3$) is bound to a non-tuple value $v$, then it is treated as if
it were the tuple \gl{\{'\_1':$v_1$\}} (resp. \gl{\{'\_2':$v_3$\}}. 

\begin{lstlisting}
SELECT (*$v_1$*).*, (*$e_2$*) AS (*$a$*), (*$v_3$*).*
\end{lstlisting}

\noindent is equivalent to

\begin{lstlisting}
SELECT VALUE TUPLEUNION(
    CASE WHEN (*$v_1$*) IS TUPLE THEN (*$v_1$*) ELSE {'_1': (*$v_1$*)} END,
    {'(*$a$*)':(*$e_2$*)},
    CASE WHEN (*$v_3$*) IS TUPLE THEN (*$v_3$*) ELSE {'_2': (*$v_3$*)} END
)
\end{lstlisting}

Notice that the attribute names \gl{\_1}, \gl{\_2} have been invented.


\subsection{Examples with combinations of multiple features}

\begin{example} 
\label{xmpl:nesting-readings}
A SFW subquery may appear in the \gl{SELECT VALUE} clause of a query, enabling
the creation of nested results.  

Consider the database

\begin{tabbing}
\ \ \ \=\gt{sensors : [}\=\gt{\{'sensor':1\},}\\
\>\>\gt{\{'sensor':2\}}\\
\>\>\gt{]}\\
\ \ \ \=\gt{logs: [}\=\gt{\{'sensor':1, 'co':0.4\},}\\
\>\>\gt{\{'sensor':1, 'co':0.2\},}\\
\>\>\gt{\{'sensor':2, 'co':0.3\}}\\
\>\>\gt{]}
\end{tabbing}

The query

\begin{tabbing}
\ \ \ \=\gt{SELECT VALUE \{}\=\gt{'sensor': s.sensor,}\\
\>\>\gt{'readings': (}\=\gt{SELECT VALUE l.co}\\
\>\>\>\gt{FROM logs AS l}\\         
\>\>\>\gt{WHERE l.sensor = s.sensor}\\ 
\>\>\>\gt{)}\\
\>\>\!\!\gt{\}}\\
\>\gt{FROM sensors AS s}
\end{tabbing}

\noindent results into

\begin{tabbing}
\ \ \ \=\gt{\ob }\=\gt{\{'sensor':1, 'readings':\ob 0.4, 0.2 \cb\},}\\
\>\>\gt{\{'sensor':2, 'readings':\ob 0.3 \cb\}}\\
\>\gt{\cb}
\end{tabbing}

\noindent Notice that each tuple of the result has a nested array, which has
been created by the inner \gl{SELECT VALUE}. 

The query could also have been written using \gl{SELECT} (instead of \gl{SELECT
VALUE}) for the outer query, as follows:

\begin{tabbing}
\ \ \ \=\gt{SELECT }\=\gt{s.sensor AS sensor,}\\
\>\>\gt{(}\=\gt{SELECT VALUE l.co}\\
\>\>\>\gt{FROM logs AS l}\\         
\>\>\>\gt{WHERE l.sensor = s.sensor}\\ 
\>\>\>\gt{) AS readings}\\
\>\gt{FROM sensors AS s}
\end{tabbing}

Furthermore, the \gt{AS sensor} could be ommitted (as in SQL).
\end{example}

\begin{example}
This example shows how the combined action of \gl{UNPIVOT} and \gl{PIVOT}
enables to analyze the attribute names. Consider the following database that has
a sequence of measurements of various gases.

\begin{tabbing}
\ \ \ \gt{sensors : [}\=\gt{\{'no2':0.6, 'co':0.7, 'co2':0.5\},}\\
\>\gt{\{'no2':0.5, 'co':0.4, 'co2':1.3\}}\\
\>\gt{]}
\end{tabbing}

The following query keeps only the carbon oxides.
\footnote{The query author is pretty weak in chemistry and cannot enumerate the
carbon oxides explicitly in her query.}

\begin{tabbing}
\ \ \ \=\gt{SELECT VALUE (}\=\gt{PIVOT v AT g}\\
\>\>\gt{FROM UNPIVOT r AS v AT g}\\
\>\>\gt{WHERE g LIKE 'co\%')}\\
\>\gt{FROM sensors AS r}
\end{tabbing}

The result is 

\begin{tabbing}
\ \ \ \gt{[}\=\gt{\{'co':0.7, 'co2':0.5\},}\\
\>\gt{\{'co':0.4, 'co2':1.3\}}\\
\>\gt{]}
\end{tabbing}

Intuitively, the \gl{UNPIVOT} turns every instance of the tuple \gt{t} into a
collection. The \gl{WHERE} filters the collections. The \gl{PIVOT} pivots the
filtered collections back into tuples.
\end{example}
