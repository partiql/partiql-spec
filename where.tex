\section{\gt{WHERE} clause}
\label{sec:where}
 
The \gl{WHERE} clause inputs the bindings that have been produced from the
\gl{FROM} clause and outputs the ones that satisfy its condition.

The boolean predicates follow SQL's 3-valued logic. Recall, PartiQL has two
kinds of absent values: \gl{NULL} and \gl{MISSING}. As far as the boolean
connectives and \gl{IS NULL} are concerned a \gl{NULL} input and a \gl{MISSING}
input behave identically. For example, \gl{MISSING AND TRUE} is equivalent to
\gl{NULL AND TRUE}: they both result into \gl{NULL}.

For the semantics of equality and of other functions, see
Section~\ref{sec:preds-and-fns}.

Alike SQL, when the expression of the \gt{WHERE} clause expression evaluates to
an absent value or a value that is not a Boolean, PartiQL eliminates the
corresponding binding. 

\begin{example} The result of
\begin{tabbing}
\gt{SELECT VALUES v.a}\\
\gt{FROM [\{'a':1, 'b':true\}, \{'a':2, 'b':null\}, \{'a':3\}] v}\\
\gt{WHERE v.b}
\end{tabbing}
\noindent is \texttt{\ob 1 \cb}.
\end{example}

The predicate \gl{IS MISSING} allows distinguishing between \gt{NULL} and
\gt{MISSING}: \gt{NULL IS MISSING} results to false; \gt{MISSING IS MISSING}
results to true.

\eat{
\subsection{Semantics of \gl{IN}}
\label{sec:in}
\yannis{TO DO.}
}