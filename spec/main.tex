\documentclass{article}


%------------------------------------------------------------------------------- 
% Math packages
%------------------------------------------------------------------------------- 

% The "amsmath" package provides advanced math extensions.
\usepackage{amsmath}

% The "amssymb" package adds new symbols to be used in math mode.
\usepackage{amssymb}

% The "amsthm" package adds the "proof" environment and "theoremstyle" command.
\usepackage{amsthm}
\theoremstyle{definition}

% The "faktor" package adds the "faktor" macro for variable substitution (i.e. vulgar fractions). This depends on the "amssymb" package.
\usepackage{faktor}

% The "semantic" package adds new macros for (PL-style) inference rules.
\usepackage[inference]{semantic}

%------------------------------------------------------------------------------- 
% Figure packages
%------------------------------------------------------------------------------- 

% The "fancyvrb" package provides advanced customization of verbatim environments, such as font families, numbering lines, box borders etc.
\usepackage{fancyvrb}

% The "graphicx" package allows including external graphic files.
%\usepackage{graphicx}

% The "subfig" package allows multiple sub-figures within a single figure, where sub-figures can be separately captioned and labeled, e.g. Figure % 1.2(a). This is a replacement for the older "subfigure" package.
\usepackage{subfig}

% HACK: The caption package (included by the subfig package) requires a counter for ACM's copyright box.
\newcounter{copyrightbox}

% The "float" package allows the "H" option for figures, which places a float % at a precise location.
\usepackage{float}

% The "caption" package allows captions for figures that are not actually in a floating environment (e.g. framed environment).
%\usepackage{caption}

% The "mdframed" package creates framed regions that can break across pages.
\usepackage{mdframed}

% The "algorithm2e" package provides keywords for typesetting algorithms. The "noend" option disables the printing of the "end" keywords. Use "algomargin" to decrease the margins for all algorithms.


% The "multirow" package allows table cells to span more than one row.
\usepackage{multirow}

% The "balance" package allows columns of the last page to be of equal height.
\usepackage{balance}

% The "fixltx2e" package prevents two-column figures from being placed out-of-order wrt regular (one-column) figures.
%\usepackage{fixltx2e}

% The "beramono" package provides Bitstream Vera Mono, which has a bold typewritter fontface.
\usepackage[scaled]{beramono}
\usepackage[T1]{fontenc}

% The "courier" package provides Courier, which has a bold typewritter fontface.
%\usepackage{courier}


%------------------------------------------------------------------------------- 
% Misc packages
%------------------------------------------------------------------------------- 

% The "optional" package allows multiple versions of the document via optional text.
%\usepackage{optional}

% The "lmodern" package allows for better font size support https://ctan.org/pkg/lm
\usepackage{lmodern}

% The "upquote" package formats quotes appropriately in verbatim
\usepackage{upquote}

% The "xstring" package allows switch/case conditionals.
\usepackage{xstring}

% The "xcolor" package allows colored text and backgrounds.
\usepackage[table]{xcolor}

% The "xspace" package allows us to write unit macros without protected spacing
\usepackage{xspace}

% The "soul" package allows highlighting.
\usepackage{soul}

% The "ulem" package allows highlighting.
\usepackage[normalem]{ulem}

% The "tocloft" packages allows generating custom lists that are similar to table of contents, list of figures etc.
% \usepackage[subfigure]{tocloft}

% The "hyperref" package allows creating hyperlinks. Note that it must be the last package loaded, and will automatically includes the "url" package.
\usepackage[colorlinks]{hyperref}
\hypersetup{ 
  linkcolor=blue
} 

% The "hypcap" package fixes "hyperref" so that hyperlinks go to the top of a float (as opposed to its caption).
\usepackage[all]{hypcap}

% Adjust whitespace before/after floats
\setlength{\textfloatsep}{6pt plus 1.0pt minus 2.0pt}
\setlength{\floatsep}{6pt plus 1.0pt minus 1.0pt}

%%%%
%% Use better margins
%%%%
\usepackage[margin=0.7in]{geometry}

%%%%
%% Control header and footer 
%%%%
\usepackage{lastpage} % used in customized footer
\usepackage{fancyhdr}
\setlength{\headheight}{16pt} 
\pagestyle{fancy}
\fancyhf{}
\lhead{--- DRAFT ---}
\rhead{\rightmark}
\lfoot{\today}
\rfoot{Page \thepage\ of \pageref{LastPage}}

%%%%
%% Watermark  
%%%%
\usepackage[firstpage]{draftwatermark} % to remove replace `firstpage` with `nostamp`
\SetWatermarkScale{5.0}


%%%%
%% Formatting source code 
%% 
%% Use Cases: 
%% 1. Include a file with source code 
%%    \lstinputlisting[language=Java]{HelloWorld.java}
%% 
%% 2. Environment 
%%    \begin{lstlisting}[language=Scheme] 
%%    (foldl + 0 '(1 2 3))
%%    \end{lstlisting}
%%
%% 3. Inline with English text
%%      \lstinline{int i;}
%%%%
\usepackage{listings}

\lstset{
language=SQL,                           % Code langugage
basicstyle=\ttfamily,                   % Code font, Examples: \footnotesize, \ttfamily
keywordstyle=\textbf,                   % Keywords font ('*' = uppercase)
escapeinside={(*}{*)},                  % Escape syntax e.g. (*$\Longrightarrow$*)
% commentstyle=\color{gray},              % Comments font
% numbers=left,                           % Line nums position
% numberstyle=\tiny,                      % Line-numbers fonts
% stepnumber=1,                           % Step between two line-numbers
% numbersep=5pt,                          % How far are line-numbers from code
% tabsize=2,                              % Default tab size
% captionpos=b,                           % Caption-position = bottom
% breaklines=true,                        % Automatic line breaking?
% breakatwhitespace=false,                % Automatic breaks only at whitespace?
% showspaces=false,                       % Dont make spaces visible
% showtabs=false,                         % Dont make tabs visible
frame=none,                             % A frame around the code
aboveskip=0.25in,                       % Top margin
belowskip=0.25in,                       % Bottom margin
xleftmargin=0.25in,                     % Left margin
xrightmargin=0.25in,                    % Right margin
columns=flexible,                       % Column format
morekeywords={                          % PartiQL additional keywords
  MISSING,
  VALUE, PIVOT, UNPIVOT,
  STRING, STRUCT, TUPLE, DECIMAL, INT, BOOL,
},
}

%------------------------------------------------------------------------------- 
% Macros
%------------------------------------------------------------------------------- 

% Define our own compact enumerate
\newenvironment{compact_enum}
{\setlength{\leftmargini}{1em}
\begin{enumerate}
  \setlength{\labelsep}{.3em} 
  \setlength{\itemsep}{.4em}
  \setlength{\parskip}{0pt}
  \setlength{\parsep}{0pt}}
{\end{enumerate}}

% Define our own compact itemize
\newenvironment{compact_item}
{\setlength{\leftmargini}{1em}
\begin{itemize}
  \setlength{\labelsep}{.3em} 
  \setlength{\itemsep}{.4em}
  \setlength{\parskip}{0pt}
  \setlength{\parsep}{0pt}}
{\end{itemize}}


%------------------------------------------------------------------------------- 
% Database symbols
%------------------------------------------------------------------------------- 

\def\join{$\bowtie$}
\def\ojoin{\setbox0=\hbox{$\bowtie$}%
  \rule[-.02ex]{.25em}{.4pt}\llap{\rule[\ht0]{.25em}{.4pt}}}
\def\leftouterjoin{\mathbin{\ojoin\mkern-5.8mu\bowtie}}
\def\rightouterjoin{\mathbin{\bowtie\mkern-5.8mu\ojoin}}
\def\fullouterjoin{\mathbin{\ojoin\mkern-5.8mu\bowtie\mkern-5.8mu\ojoin}}
\def\semijoin{\mbox{$\mathrel{\raise1pt\hbox{\vrule height5pt depth0pt\hskip-1.5pt$>$\hskip -2.5pt$<$}}$}}
\def\antisemijoin{\overline{\semijoin}}

%------------------------------------------------------------------------------- 
% Grammar symbols (BNFs and Tree Grammars) 
%------------------------------------------------------------------------------- 

% Formatting commands for tree grammars
\newcommand{\gn}[1]  {\textit{#1}}           % (N)on-terminal
\newcommand{\gt}[1]  {\texttt{\textbf{#1}}}  % (T)erminal
\newcommand{\gl}[1]  {\texttt{\textbf{#1}}}  % (L)iteral
\newcommand{\gs}[1]  {\textit{\textbf{#1}}}  % (S)pecial construct
\newcommand{\gob}[0] {[}                     % (O)ptional begin
\newcommand{\goe}[0] {]}                     % (O)ptional end
\newcommand{\gr}[0]  {...}                   % (R)epeat
\newcommand{\gp}[0]  {$\rightarrow$}         % (P)roduction rule
\newcommand{\gd}[0]  {$|$}                   % (D)isjunction

%------------------------------------------------------------------------------- 
% Misc symbols
%-------------------------------------------------------------------------------

% Undirected single quote mark
\chardef\singlequote=13




% Let's make sure each section starts on its own page
\let\oldsection\section
\renewcommand\section{\clearpage\oldsection}

\newcounter{issue}

\newcommand{\reminder}[1]{\textbf{OPEN~\refstepcounter{issue}: \emph{#1}}}
\renewcommand{\reminder}[1]{} % comment this line to enable \reminder
%\newcommand{\reminder}[1]{{\refstepcounter{issue}}}
\newcommand{\reminderclosed}[1]{\textbf{RESOLVED~\refstepcounter{issue}: \emph{#1}}}
\renewcommand{\reminderclosed}[1]{} % comment this line to enable \reminderclosed
%\newcommand{\reminderclosed}[1]{{\refstepcounter{issue}}}
\newcommand{\reminderfuture}[1]{\textbf{FUTURE~\refstepcounter{issue}: \emph{#1}}}
\renewcommand{\reminderfuture}[1]{} % comment this line to enable \reminderfuture
%\newcommand{\reminderfuture}[1]{{\refstepcounter{issue}}}


\newcommand{\almann}[1]{\reminder{Almann: #1}}
\newcommand{\yannis}[1]{\reminder{Yannis: #1}}
\newcommand{\yannisclosed}[1]{\reminderclosed{Yannis: #1}}
\newcommand{\almannclosed}[1]{\reminderclosed{Almann: #1}}	
\newcommand{\yannisfuture}[1]{\reminderfuture{Yannis: #1}}
\newcommand{\almannfuture}[1]{\reminderfuture{Almann: #1}}

\newcommand{\eat}[1]{}

\newcommand{\TRUE}{\gl{TRUE}\xspace}
\newcommand{\FALSE}{\gl{FALSE}\xspace}
\newcommand{\NULL}{\gl{NULL}\xspace}
\newcommand{\MISSING}{\gl{MISSING}\xspace}

\def\ground{\bot}
\def\collscan{S^C}
\def\outercollscan{O^C}
\def\tuplescan{S^T}
\def\outertuplescan{O^T}
\def\tuplenav{N^T}
\def\arraynav{N^A}
\def\constructcoll{C^C}
\def\corr{R}
\def\flat{F}

\def\evalto{\leadsto}
\def\group{f_\gl{GROUP}}
\def\order{<^o}
%\def\order{f_\gl{ORDER}}x

\def\arr{\operatorname{array}}
\def\bag{\operatorname{bag}}
\def\tuple{\operatorname{tuple}}
\def\map{\operatorname{map}}
\def\fst{\operatorname{fst}}
\def\sub{\operatorname{sub}}
\def\sort{\operatorname{sort}}
\def\setop{\operatorname{set\_op}}
\def\unionall{\operatorname{union\_all}}
\def\intersectall{\operatorname{intersect\_all}}
\def\exceptall{\operatorname{except\_all}}
\def\setopeq{\overset{@}{=}}


% Operators
\newcommand{\g}{\{\!\{\langle \rangle\}\!\}}
\newcommand{\term}[1]{\ensuremath{\ddot{#1}}}
\newcommand{\op}[1]{\ensuremath{\operatorname{#1}}}
%\newcommand{\scancoll}{\ensuremath{\ggg^C}}
\newcommand{\scanbag}{\ensuremath{\ggg^{\{\!\{\}\!\}}}}
\newcommand{\leftscanbag}{\ensuremath{{\stackrel{\circ}{\ggg}}^{\{\!\{\}\!\}}}}
\newcommand{\scanarray}{\ensuremath{\ggg^{[]}}}
\newcommand{\scantuple}{\ensuremath{\ggg^{\{\}}}}
%\newcommand{\leafscancoll}{\ensuremath{\bar{\ggg}^C}}
\newcommand{\leafscanbag}{\ensuremath{\bar{\ggg}^{\{\!\{\}\!\}}}}
\newcommand{\leafscanarray}{\ensuremath{\bar{\ggg}^{[]}}}
\newcommand{\leafscantuple}{\ensuremath{\bar{\ggg}^{\{\}}}}
\newcommand{\navarray}{\ensuremath{\gl{[]}}}
\newcommand{\navtuple}{\ensuremath{\bullet}}
\newcommand{\functioncall}{\ensuremath{\lambda}}
\newcommand{\constructarray}{\ensuremath{\square^{[]}}}
\newcommand{\constructbag}{\ensuremath{\square^{\{\{\}\}}}}
\newcommand{\constructtuple}{\ensuremath{\square^{\{\}}}}
\newcommand{\returnarray}{\ensuremath{r^{[]}}}
\newcommand{\returnbag}{\ensuremath{r^{\{\!\{\}\!\}}}}
\newcommand{\returntuple}{\ensuremath{r^{\{\}}}}
\newcommand{\applyplan}{\ensuremath{\alpha}}
\newcommand{\applyfunction}{\ensuremath{\lambda}}
\newcommand{\unnest}{\ensuremath{unnest}}
\newcommand{\unnestout}{\ensuremath{unnest^o}}
\newcommand{\nav}{\ensuremath{nav}}
%\newcommand{\select}{\ensuremath{\sigma}}
\newcommand{\project}{\ensuremath{\pi}}
\newcommand{\innerjoin}{\ensuremath{\Join}}
\newcommand{\psemijoin}{\ensuremath{\hat\ltimes}}
\newcommand{\pjoin}{\ensuremath{\hat\innerjoin}}
\newcommand{\kjoin}{\ensuremath{\bar\ltimes}}
\newcommand{\outerjoinline}{\setbox0=\hbox{$\Join$}%
    \rule[0.1ex]{.27em}{.4pt}\llap{\rule[1.3ex]{.27em}{.4pt}}}
\newcommand{\leftjoin}{\ensuremath{\mathbin{\outerjoinline\mkern-5.8mu\Join}}}
\newcommand{\rightjoin}{\ensuremath{\mathbin{\Join\mkern-5.8mu\outerjoinline}}}
\newcommand{\fulljoin}{\ensuremath{\mathbin{\outerjoinline\mkern-5.8mu\Join\mkern-6.4mu\outerjoinline}}}
\newcommand{\inner}{\ensuremath{\stackrel{\leftarrow}{\Join}}}
\newcommand{\outercorr}{\ensuremath{\stackrel{\leftarrow}{\leftjoin}}}
\newcommand{\groupby}{\ensuremath{\gamma}}
\newcommand{\union}{\ensuremath{\Cup}}
\newcommand{\intersect}{\ensuremath{\Cap}}
\newcommand{\except}{\ensuremath{\setminus}}
\newcommand{\distinct}{\ensuremath{\delta}}
%\newcommand{\semijoin}{\ensuremath{\ltimes}}
\newcommand{\antijoin}{\ensuremath{\vartriangleright}}
\newcommand{\val}[1]{\ensuremath{\textbf{x}}}

% GENERAL definitions
\newcommand{\sql}{SQL Compatibility}
\newcommand{\pl}{Functional Programming}

% MODEL definitions
\newcommand{\bt}{\texttt{\`}}
\newcommand{\ob}{<\!\!<}
\newcommand{\cb}{>\!\!>}
\newcommand{\out}{\{-}
\newcommand{\cut}{-\}}

\newcommand{\ionquote}[1]{\gl{\bt} #1 \gl{\bt}}
 
% ENVIRONMENT definitions
\newcommand{\env}{\rho}
\newcommand{\typeenv}{\Gamma}

\newcommand{\db}{\env_{0}}
\newcommand{\typedb}{\typeenv_{0}}

\newcommand{\benv}{\ensuremath{(\db, \env)}}

\newcommand{\type}{\tau}

\newcommand{\eval}{\rightarrow}
\newcommand{\eqv}{\Leftrightarrow}
\newcommand{\neqv}{\nLeftrightarrow}

% FROM clause definitions
\newcommand{\from}{\gl{FROM}\xspace}
\newcommand{\JOIN}{\gl{JOIN}\xspace}
\newcommand{\CROSSJOIN}{\gl{CROSS JOIN}\xspace}
\newcommand{\LEFTCJOIN}{\gl{LEFT CROSS JOIN}\xspace}
\newcommand{\FULLCJOIN}{\gl{FULL CROSS JOIN}\xspace}
\newcommand{\LEFTJOIN}{\gl{LEFT JOIN}\xspace}
\newcommand{\FULLJOIN}{\gl{FULL JOIN}\xspace}
\newcommand{\isCollection}{\gl{isCollection}\xspace}
\newcommand{\at}{\gl{AT}\xspace}
\newcommand{\as}{\gl{AS}\xspace}
\newcommand{\on}{\gl{ON}\xspace}
\newcommand{\unpivot}{\gl{UNPIVOT}\xspace}

% SELECT clause definitions
\newcommand{\select}{\gl{SELECT}\xspace}
\newcommand{\values}{\gl{VALUE}\xspace}
\newcommand{\pivot}{\gl{PIVOT}\xspace}
\newcommand{\pivotPairs}{\gl{PIVOTPAIRS}\xspace}
\newcommand{\tupleunion}{\gl{TUPLEUNION}\xspace}
\newcommand{\anv}{\gl{ANV}\xspace}
\newcommand{\case}{\gl{CASE}\xspace}
\newcommand{\when}{\gl{WHEN}\xspace}
\newcommand{\isTuple}{\gl{isTuple}\xspace}
\newcommand{\inventName}{\gl{inventName}\xspace}

\newtheorem{example}{Example}

\newcommand{\inrecord}[1]   { {#1}_{\operatorname{in}} } % {\dot{#1}}
\newcommand{\outrecord}[1]  { {#1}_{\operatorname{out}} } % {\ddot{#1}}
\newcommand{\inbinding}[1]  { {#1}_{\operatorname{in}} }
\newcommand{\outbinding}[1] { {#1}_{\operatorname{out}} }

\newcommand{\id}[1]{{#1}^{id}}
\newcommand{\evalf}{\gl{eval}}

\newcommand{\update}[2]{\gl{update}(#1,#2)}
\newcommand{\insertbag}[2]{\gl{insertinbag}(#1,#2)}
\newcommand{\inserttuple}[3]{\gl{inserinttuple}(#1,#2,#3)}
\newcommand{\append}[2]{\gl{append}(#1,#2)}
\newcommand{\delete}[1]{\gl{delete}(#1)}
\newcommand{\insertorder}[2]{\gl{insertorder}(#1,#2)}

\newcommand{\highlight}[1]{\noindent\textbf{#1:}}

\newcounter{query}

\begin{document}
\title{PartiQL Specification}
\author{The PartiQL Specification Committee}
\maketitle
\newpage

\tableofcontents
\newpage
\input{license}
\section{Introduction}
\label{sec:introduction}

\paragraph{Draft Status} This document is currently a working draft and subject
to change.  Certain sections are marked as ``work in progress'' (WIP) and will
be expanded soon.

\paragraph{Audience} This document presents the formal syntax and semantics of
PartiQL. It is oriented to PartiQL query processor builders who need the full
and formal detail on PartiQL.

SQL users who are not interested in the full detail and the complete formalism
but are interested in learning how PartiQL extends SQL may also read the
tutorial. Unlike this formal specification, the tutorial has a ``how to''
orientation and is primarily driven by examples.

\paragraph{PartiQL core and PartiQL syntactic sugar}
In the interest of precision and succinctness, we tier the PartiQL specification
in two layers: The PartiQL core is a functional programming language with
composable aspects. Three aspects of the PartiQL core syntax and semantics are
characteristic of its functional orientation: Every (sub)query and every (sub)
expression input and output PartiQL data. Second, each clause of a SELECT query
is itself a function. Third, every (sub)query evaluates within the environment
created by the database names and the variables of the enclosing queries.

Then we layer ``syntactic sugar'' features over the core. Commonly, syntactic
sugar achieves well-known SQL syntax and semantics. Formally, every syntactic
sugar feature is explained by reduction to the core.

\subsection{Notation}
\label{sec:notation}

We use extended Backus-Naur form (EBNF) to describe
grammars. Non-terminal terms appear in italics, e.g., \nt{value}, and
terminal appear in typewriter font, e.g., \terminal{NULL}. A
production rules uses \prod to define the expansion of a non-terminal
and \alt for alternatives, e.g.,

\begin{displaymath}
  \begin{grammar}
    \nt{value} & \prod & \nt{string} & \mbox{\nt{value} expands into either \nt{string}}\\
               & \alt  & \terminal{null} & \mbox{or the literal \terminal{null}}\\
  \end{grammar}
\end{displaymath}

We also use $\seplof{\x{}}{,}$ to denote a possibly empty list of
comma separated elements $\x{}$, i.e.,

\begin{displaymath}
  \begin{array}{lll}
    \seplof{\x{}}{,} & = & \x{1}, \ldots, \x{n} \quad n \geq 0 \\
  \end{array}
\end{displaymath}

\section{Data Model}
\label{sec:model}

\begin{figure}[ht!]
\centering
\begin{tabular}{|r|lrl|}
\hline
 1  & \gn{value}                    & \gp   & \gn{absent\_value} \\
 2  &                               & \gd   & \gn{scalar\_value} \\
 3  &                               & \gd   & \gn{tuple\_value} \\
 4  &                               & \gd   & \gn{collection\_value} \\
 5  & \gn{absent\_value}            & \gp   & \NULL \\
 6  &                               & \gd   & \MISSING \\
 7  & \gn{scalar\_value}            & \gp   & \ionquote{\gs{ion\_literal}} \\
 8  &                               & \gd   & \gs{sql\_literal} \\
 9  & \gn{tuple\_value}             & \gp   & \gl{\{\ \}} \\
10  &                               & \gd   & \gl{\{} \gs{string\_value} \gl{:} \gn{value} (\gl{,} \gs{string\_value} \gl{:} \gn{value})* \gl{\}} \\ 
11  &                               &       & \ \ \ \ \ \ \ \gn{value} cannot be \MISSING \\
12  & \gn{collection\_value}        & \gp   & \gn{array\_value} \\
13  &                               & \gd   & \gn{bag\_value} \\
14  & \gn{array\_value}             & \gp   & \gl{[} (\gn{value} (\gl{,} \gn{value})*)? \gl{]} \\
15  & \gn{bag\_value}               & \gp   & \gl{\ob} (\gn{value} (\gl{,} \gn{value})*)? \gl{\cb} \\
\hline
\end{tabular}
\caption{BNF Grammar for PartiQL Values}
\label{figure:values:bnf}
\end{figure}

\newcommand{\linevalues}[1]{%
    \IfEqCase*{#1}{%
    {value}{BNF lines~1--4}%
    {missing}{(BNF line~6)}%
    {scalar}{(BNF lines~7--8)}%
    {tuple}{(BNF lines~9--11)}%
    {tuplemissing}{(BNF line~11)}
    {collection}{(BNF lines~12--13)}%
    }[\errmessage{Unable to ref #1 for value BNF}]%
}

Figure~\ref{figure:values:bnf} shows the BNF grammar for PartiQL values. A
PartiQL database generally contains one or more PartiQL \textit{named values}. A
\textit{name}, is an identifier, such as a table name, that is associated with a
PartiQL value.  Section~\ref{section:environment-and-sfw} defines what these
names are, and how SQL qualified names work, in detail.

The type of a value is \gn{absent}, \gn{scalar}, \gn{tuple} or \gn{collection}.
Further subtyping applies to scalars, tuples, and collections. We will often use
the name \emph{complex value} to refer to any non-scalar and non-absent value.
That is, complex values include \textit{tuples} and \textit{collections}. A
tuple is a set of attribute name/value pairs, where each name is a string (as in
SQL). A tuple in the data model is unordered. A conventional SQL tuple is an
ordered tuple since the schema dictates the order of the attributes and certain
SQL operations may use the order---support for this is described in detail in
Section~\ref{sec:schema}.

\yannis{OS: Next paragraph is odd. We give up on data model and talk about a known format (JSON)
and Ion (as a format). I recommend that we, instead, cast the Ion type system as PartiQL's 
recommended extension over SQL. I'm not sure why JSON deserves a special mention.
}

\almann{How does the following look?}

PartiQL's data model extends SQL to Ion's type system to cover schema-less and 
nested data. Such values can be directly quoted with \bt quotes.

Unlike SQL, PartiQL allows the possibility of duplicate attribute names, in the
interest of compatibility with non-strict JSON/Ion datasets. However PartiQL
does not encourage duplicate attribute names; navigation into tuples with the
conventional dot notation (Section~\ref{section:paths}) is tuned to the
assumption that the attribute names are unique. 

A \gn{collection\_value} is either an ordered or unordered
\linevalues{collection}. Both arrays and bags may contain duplicate elements. An
array is ordered (similar to a JSON array or Ion list) and each element is
accessible by its ordinal position. (See specifics of access by position in
Section~\ref{section:paths}.) Arrays are delimited with \gl{[} \gl{]}. For
example, the value of the attribute \texttt{configurationItems} in
Figure~\ref{figure:values:example-value} is an array. Arrays have size, which
is not explicitly denoted but is implied by the number of elements in the array.
For example, the size of the \texttt{configurationItems} in
Figure~\ref{figure:values:example-value} is 2. The first element of an array
corresponds to index 0; the last element corresponds to index size minus one.

In contrast, a bag is unordered (similar to a SQL table) and its elements cannot
be accessed by ordinal position. Bags are denoted with \gt{\ob} and \gt{\cb}.

Finally, note that PartiQL has two kinds of absent values: \NULL and
\MISSING. The motivation is as follows: Unlike SQL, where a query that
refers to a non-existent attribute name is expected to fail during compilation,
in semi-structured data one expects a query to operate even if some of the tuples
do not define some of the attributes that the query's paths mention. Hence
PartiQL contains the special value \MISSING \linevalues{missing}, which is
the path result in cases where navigation fail to bind to any information. The
distinction between \MISSING and \NULL enables retaining the original
distinction between a missing attribute and a null-valued attribute. The utility
of \MISSING (as opposed to just having \NULL) will become further
apparent when navigation into semi-structured data and construction of
semi-structured results is discussed.

The value \MISSING may not appear as an attribute value. Notice that in the
interest of readability, the syntax of Figure~\ref{figure:values:bnf} does
exclude these cases; rather the ``\gn{value} cannot be \MISSING''
restrictions \linevalues{tuplemissing} indicate that \MISSING cannot appear
as an attribute value.

\paragraph{Comparisons to the relational model} In summary, the PartiQL
data model extended the SQL data model in the following ways:

\begin{compact_enum}
\item The elements of an array/bag can be any kind of value---not just tuples.
Furthermore they can be heterogeneous. That is, there are no restrictions
between the elements of an array/bag. For example, the two tuples in
\texttt{configurationItems} array of are \textit{heterogeneous} because: (i)
each tuple has a different set of attributes (for example, the second tuple has
\texttt{configurationStateId} while the first does not), (ii) an attribute of a
first tuple may map to some type while the same attribute in a second tuple may
map to another type.

\item More broadly, unlike SQL where the values are tables that have homogeneous
tuples that have scalars, PartiQL complex values are {\em arbitrary compositions
of arrays, bags and tuples}. E.g., the top level value of
Figure~\ref{figure:values:example-value} is a tuple and the
\texttt{configurationItems} array has two heterogeneous tuples. Note that in
this example, the top-level name refers to a value that is \textit{not} a bag
(e.g. a table).

\item There is a distinction between null-valued attributes and missing
attributes. 

\yannis{OS: it carries over to array elements, right? 
But it makes no sense to carryover to bags and thus we need to mark
the "cannot be MISSING) exception there also.}

\almann{MISSING in a bag and array is fine, the problem is in serializing such 
data.}

\item PartiQL makes an explicit distinction between arrays and bags, where the
former have order and their elements can be addressed by ordinal position.
\footnote{Despite the ``SQL table is a bag'' and ``the results of an SQL query
is a table'' statements of SQL textbooks, SQL also recognizes that the result of
a query that has an \gl{ORDER BY} is a list, i.e., an ordered collection of
tuples.}
\end{compact_enum}

\begin{figure}[htb!]
\begin{lstlisting}
{
  'fileVersion':'1.0',
  'configurationItems':[
    {
        'configurationItemCaptureTime': `2016-08-03T08:56:52.415Z`,
        'resourceId':'h-0337bfe6793cf9e0c',
        'configuration':{
          'hostId':'h-0337bfe6793cf9e0c',
          'hostProperties':{
              'sockets':2,
              'cores':20,
              'totalVCpus':32,
              'instanceType':'m4.medium'
          },
        'tags':{
          'CostCenter':'Prod'
        },
    },
    {
        'configurationItemCaptureTime':`2016-08-03T09:41:56.906Z`,
        'resourceId':'h-0337bfe6793cf9e0c',
        'configurationStateId':3,
        'configuration':{
          'hostId':'h-0337bfe6793cf9e0c',
          'autoPlacement':'off',
          'hostProperties':{
              'sockets':2,
              'cores':20,
              'totalVCpus':32,
              'instanceType':'m3.medium'
          },
        'tags':{
        },
    }
  ]
}
\end{lstlisting}
\caption{An Example of a PartiQL Value}
\label{figure:values:example-value}
\end{figure}

\section{Queries, Environments and Binding Tuples}
\label{section:environment-and-sfw}
 
\begin{figure}[ht!]
\centering
\scriptsize
\begin{tabular}{|@{~}rc@{~}l@{~}|}
\hline
         1  & \multicolumn{2}{@{}l@{~}|}{\gn{query}} \\
         2  & \gp & \gn{sfw\_query} \\
         3  & \gd & \gn{expr\_query} \\ \hline
         4  & \multicolumn{2}{@{}l@{~}|}{\gn{sfw\_query}} \\
         5  & \gp & (\gt{WITH} \gn{query} \gt{AS} \gn{variable})? \\
         6  &     & \gn{select\_clause} \\ %FIXME (see Figure~\ref{figure:select:bnf}) \\
         7  &     & \gn{from\_clause} \\   %FIXME (see Figure~\ref{figure:from:bnf}) \\ 
         8  &     & (\gt{WHERE} \gn{expr\_query})? \\
         9  &     & (\gt{GROUP BY} \gn{expr\_query} (\gt{AS} \gn{variable})? \\ 
        10  &     & ~~~~(\gt{,} \gn{expr\_query} (\gt{AS} \gn{variable})?)*)? \\
        11  &     & ~~~~\gt{GROUP AS} \gn{variable} \\
        12  &     & (\gt{HAVING} \gn{expr\_query})? \\
        13  &     & ((\gt{OUTER})? (\gt{UNION}\gd\gt{INTERSECT}\gd\gt{EXCEPT}) \gt{ALL}? \gn{sfw\_query})? \\
        14  &     & ((\gt{ORDER BY} (\gn{expr\_query} (\gt{ASC}\gd\gt{DESC})? \gs{order\_spec}?  \\
        15  &     & ~~~~~~~~(\gt{,} \gn{expr\_query} (\gt{ASC}\gd\gt{DESC})? \gs{order\_spec}?)*)? ) \\
        16  &     & ~~~~| \gt{PRESERVE})? \\
        17  &     & (\gt{LIMIT} \gn{expr\_query})? \\
        18  &     & (\gt{OFFSET} \gn{expr\_query})? \\ \hline
        19  & \multicolumn{2}{@{}l@{~}|}{\gn{expr\_query}} \\
        20  & \gp & \gt{(} \gn{sfw\_query} \gt{)} \\
        21  & \gd & \gn{path\_expr} \\
        22  & \gd & \gn{function\_name} \gl{(} (\gn{expr\_query} (\gt{,} \gn{expr\_query})*)? \gl{)} \\
        23  & \gd & \gt{\{} (\gs{expr\_query}\gt{:}\gn{expr\_query} (\gt{,} \gs{expr\_query}\gt{:}\gn{expr\_query})*)? \gt{\}} \\
        24  & \gd & \gt{[} (\gn{expr\_query} (\gt{,} \gn{expr\_query})*)? \gt{]} \\
        25  & \gd & \gl{\ob} (\gn{expr\_query} (\gt{,} \gn{expr\_query})*)? \gl{\cb} \\
        26  & \gd & \gs{sql\_scalar\_expr} \\
        27  & \gd & \gs{value\_constant} \\ \hline
        28  & \multicolumn{2}{@{}l@{~}|}{\gn{path\_expr}} \\
        29  & \gp & \gs{variable} \\
%        30  & \gd & \gt{(} \gn{sfw\_query} \gt{)} \\
        30  & \gd & \gt{(} \gn{expr\_query} \gt{)} \\
        31  & \gd & \gn{path\_expr} \gt{.} \gs{attr\_name} \\
        32  & \gd & \gn{path\_expr} \gt{[} \gs{expr\_query} \gt{]} \\
        33  & \gd & \gn{path\_expr} \gt{.} \gl{*} \\
        34  & \gd & \gn{path\_expr} \gt{[} \gl{*} \gt{]} \\
\hline
\end{tabular} 
\caption{BNF Grammar for PartiQL Queries}
\label{figure:query:bnf}
\end{figure}

\newcommand{\linequery}[1]{%
    \IfEqCase*{#1}{%
    {sfw query}{(Figure~\ref{figure:query:bnf}, lines~4--18)}%
    {expression query}{lines~19--34}%
    {function invocation}{(Figure~\ref{figure:query:bnf}, line~22)}%
    {path expressions}{(Figure~\ref{figure:query:bnf}, lines~28--34)}%
    {tuple nav}{(Figure~\ref{figure:query:bnf}, lines~32--33)}%
    {array nav}{(Figure~\ref{figure:query:bnf}, line~32)}%
    {sfw subquery}{Figure~\ref{figure:query:bnf}, line~20}%
    {expression, sql}{(Figure~\ref{figure:query:bnf}, line~26)}%
    {expression, extensions}{(Figure~\ref{figure:query:bnf}, lines~21--25, 28--34)}%
    {constructors}{(Figure~\ref{figure:query:bnf}, lines~23--25)}%
    {group by}{(Figure~\ref{figure:query:bnf}, lines~9--11)}%
    }[\errmessage{Unable to ref #1 for query BNF}]%
}

PartiQL may be seen as a functional programming language with composable
semantics. Three aspects of the PartiQL syntax and semantics are characteristic
of its functional orientation: First, every (sub-)query and every
(sub-)expression input and output PartiQL data. Second, each clause of an SFW
query (\gt{SELECT}-\gt{FROM}-\gt{WHERE}) is itself a function. Third, every
(sub-)query evaluates within an \emph{environment} created by the database names
and the variables of the enclosing queries.

\subsection{Basics of PartiQL Syntax} 
\label{sec:syntax-basics}

A PartiQL query is either an \textit{SFW query} (i.e.
\gt{SELECT}-\gt{FROM}-\gt{WHERE}-$\ldots$, \linequery{sfw query} of the grammar
of Figure \ref{figure:query:bnf}) or an \textit{expression query} (also called
\textit{simple expression} in the rest, \linequery{expression query}) such as a
path expression \linequery{path expressions} or a function invocation. Unlike
SQL expressions, which are restricted to outputting scalar and null values,
PartiQL expressions output arbitrary PartiQL values, and are fully composable
within larger SFW queries and expressions. Indeed, PartiQL allows the top-level
query to also be an expression query, not just a SFW query as in SQL.

An PartiQL (sub)query is evaluated within an environment, which provides
variable bindings (as defined next). 

\subsection{Environments} 
\label{sec:environments-and-bindings}

\begin{figure}[ht!]
\centering
\begin{tabular}{|r|lrl|}
\hline
   1  & \gn{bind\_name}         & \gp   & \gn{global\_name} \\
   2  &                         & \gd   & \gn{variable} \\ \hline
   3  & \gn{qualified\_name}    & \gp   & \gn{identifier} (\gl{.} \gn{identifier})+ \\
   4  & \gn{variable\_name}     & \gp   & \gn{identifier} \\ \hline
   5  & \gn{identifier}         & \gp   & (\gl{\textdollar} | \gl{\_} | \gs{letter})
                                          (\gl{\textdollar} | \gl{\_} | \gs{letter} 
                                            | \gs{digit})* \\
   6  &                         & \gd   & \gl{"} \gs{quoted\_identifier\_body} \gl{"} \\
\hline
\end{tabular}
\caption{BNF Grammar for PartiQL Names}
\label{figure:names:bnf}
\end{figure}

\newcommand{\linenames}[1]{
    \IfEqCase*{#1}{
    {bind name}{(Figure~\ref{figure:names:bnf}, lines~4--18)}
    {qualified name}{(Figure~\ref{figure:names:bnf}, line~3)}
    {variable name}{(Figure~\ref{figure:names:bnf}, line~4)}
    {identifier}{(Figure~\ref{figure:names:bnf}, lines~5--6)}
    }[\errmessage{Unable to ref #1 for identifiers BNF}]
}

Each PartiQL (sub-)query and PartiQL (sub-)expression $q$ is evaluated within
the \textit{database environment} $\db$ created by the database names and the
\textit{variables environment} $\env$ created by the defined query variables.
The pair of these environments, $\benv$ is collectively called the
\textit{bindings environment}.

In either case, an environment is a \textit{binding tuple} $\langle x_1:v_1,
\ldots, x_n:v_n \rangle$, where each $x_i$ is a \textit{bind name}
\linenames{bind name} that is unique and binds to the PartiQL \textit{value}
$v_i$. The two distinct environments may also be thought of as \textit{global}
(the database object names) and \textit{local} (the lexically defined variables
in a particular scope of the query).

% TODO we need a better section around the types

Similarly, for a given $q$ at compile (i.e. planning) time, a \textit{database
type environment}, $\typedb$, and \textit{variables type environment},
$\typeenv$ are defined. The type environment is a \textit{binding tuple}
$\langle x_1:\type_1, \ldots, x_n:\type_n \rangle$, where each $x_i$ is a
\textit{name} that is unique and binds to the PartiQL \textit{type} $\type_i$.
For schema-less values, $\type$ can be considered the union of any possible type
(for which all operations are \textit{potentially} applicable). This is
discussed in more detail in Section~\ref{sec:schema}.

\textit{Qualified names} \linenames{qualified name} only ever appear in the
\textit{database environment}. Lexically defined \textit{variable names}
\linenames{variable name} are always just simple identifiers
\linenames{identifier}. For example, a relational database might define a compound name
\texttt{mydb.log}, where \texttt{mydb} is the schema (and not actually a value)
and \texttt{log} could be a table name within that schema. Note, that a
qualified name is distinct from a quoted identifier that contains a dot. Thus,
the qualified name \texttt{mydb.log} is distinct from the bind name
\texttt{"mydb.log"}.

\begin{example}
Let us assume that we evaluate the following query on the database of
Figure~\ref{figure:values:example-value}, whose top-level value is named
\gt{mydb.log}.

\begin{lstlisting}
SELECT x.resourceId
FROM mydb.log.configurationItems x
\end{lstlisting}

The query is evaluated within the database environment 

\[\db = \langle \gt{mydb.log}:\gt{\{ 'fileversion':'1.0', 'configurationItems':}\ldots \gt{\}} \rangle \]

\noindent and the variables environment $\env_1 = \langle \rangle$. Notice the
database environment $\db$ has a single name/value pair, which corresponds to
the only name (\gt{mydb.log}) of the database of
Figure~\ref{figure:values:example-value}. The variables environment has no
name/value pair because the above query is not a subquery of a larger query. 

Next, consider the subexpression \gt{x.resourceId} of the example's query. This
subexpression will, generally, be evaluated many times - once for each \gt{x}.
Technically, each time it is evaluated within the same database environment
$\db$ and within a variables environment $\env_2 = \langle \gt{x}:\ldots
\rangle$, i.e., a variables environment that defines the variable \gt{x}. 
\end{example}

\noindent \textbf{Remark on relationship of binding tuples to PartiQL tuples}
A binding tuple is similar to a PartiQL tuple, if you think of the bind names as
attribute names. The characterization ``binding'' pertains to its use in the
semantics (e.g. an association of names to types) and the fact that qualified
names are not reified in the PartiQL data model, and we have a representation.
As we will see collections of binding tuples will be homogenous, i.e., they will
all have the same ``attribute'' names. Also important, is that when we represent
binding tuples we explicitly represent a variable with a \MISSING\
value, as opposed to omitting it because the lack of a variable name is distinct
from a variable whose value is \MISSING. For example, we write $\langle
\gt{x}:\gt{1}, \gt{y}:\MISSING \rangle$, instead of $\langle \gt{x}:\gt{1}
\rangle$.$\bullet$

\noindent \textbf{Evaluation in environment} The notation $\benv \vdash q
\rightarrow v$ denotes that the PartiQL query $q$ evaluates to the value $v$
when evaluated within the database environment $\db$ and the variables
environment $\env$, i.e. when every variable of $q$ is instantiated by its
binding in $\env$ and each database name is instantiated to its value in $\db$.
For example, consider the query \gt{x + y / 2}, the database environment $\db =
\langle \gt{x}:\gt{5} \rangle$ and the variables environment $\env = \langle
\gt{y}:\gt{3} \rangle$. Then $\db, \env \vdash \gt{(x + y)/2} \rightarrow
\gt{5+3/2} \rightarrow \gt{4}$.

\subsection{The semantics of each clause of an SFW query explained as
input and output of binding tuples}
\label{sec:clause-semantics}

The semantics of PartiQL are shorter than the semantics of SQL itself---despite
being backwards compatible with SQL. The key reason is that the semantics of
each clause of an SFW query 
in the PartiQL core can be understood in isolation from the other
clauses. A clause is simply a function that inputs and outputs binding tuples.
Thus the specifics of how the binding tuples of a query and of its subqueries
are produced are a central part of the semantics. At a high level (which will be
elaborated upon later) the construction of binding environments proceeds as
follows:

\begin{compact_enum} 
\item When a query is submitted to a database, it is evaluated in an empty
variables environment $\env = \langle \rangle$.

\item The \from\ clause of a SFW query produces new environments by
concatenating bindings of the \gl{FROM} variables to the environment of its SFW
query, as explained below. 

The subqueries that appear in the \gl{WHERE}, \gl{SELECT}, etc clauses are
evaluated in these new environments. The optional \gl{GROUP BY} clause also
produces additional variable bindings, as explained in
Section~\ref{section:group-by}.

\end{compact_enum}

\begin{figure}
\[ 
\begin{array}{l}
\db: \langle \gt{mydb.r}:\gt{[3, 'x' ]}, \gt{mydb.s}:\gt{\ob \{'a':1, 'b':2\}, \{'a':3\} \cb} \rangle \\
\env: \langle \rangle
\end{array}
\]
\begin{tabbing}
\gt{FROM }\=\gt{mydb.r AS x, mydb.s AS y}\\
\>\ $B^{out}_{\gt{FROM}} = B^{in}_{\gt{WHERE}} = \ob$\=$\langle \gt{x:3, y:\{'a':1, 'b':2\}} \rangle$\\
\>\>$\langle \gt{x:3, y:\{'a':3\}} \rangle$\\
\>\>$\langle \gt{x:'x', y:\{'a':1, 'b':2\}} \rangle$\\
\>\>$\langle \gt{x:'x', y:\{'a':3\}} \rangle$\\
\>\>\!\!$\cb$\\
\gt{WHERE x > y.b}\\
\>\ $B^{out}_{\gt{WHERE}} = B^{in}_{\gt{SELECT}} = \ob\langle \gt{x:3, y:\{'a':1, 'b':2\}} \rangle$ \cb\\
\gt{SELECT x AS foo, y.a AS bar}\\
\>Result: $\ob \gt{\{foo:3, bar:1\}} \cb$
\end{tabbing}
\caption{An Example SFW Query with Flow of Binding Tuples}
\label{fig:xmpl:sfw-bindings}
\end{figure}

\paragraph{SFW query clauses as operators that input/output binding tuples}
Similar to SQL semantics, the clauses of an PartiQL SFW query are evaluated in
the following order: \gl{WITH}, \gl{FROM}, \gl{LET}, \gl{WHERE}, \gl{GROUP BY},
\gl{HAVING}, \gl{LETTING} (which is special to PartiQL), \gl{ORDER BY},
\gl{LIMIT / OFFSET} and \gl{SELECT} (or \gl{SELECT VALUE} or \gl{PIVOT}, which
are both special to ion PartiQL).
\footnote{PartiQL also supports a syntax improvement where \gl{SELECT}
is optionally written as the last clause since, anyway, that's the proper
way to read an SQL query.}

% Notice that PartiQL SFW queries may also have a \gl{LET} clause, between
% the \gl{FROM} and \gl{WHERE} clauses and a \gl{LETTING} clause between
% the \gl{HAVING} and \gl{ORDER BY} clauses. 

\yannis{to do: add LET and LETTING in syntax.}

Using the example of Figure~\ref{fig:xmpl:sfw-bindings}, we illustrate how the
clauses of an SFW query input and output binding tuples. In the
Figure~\ref{fig:xmpl:sfw-bindings}, the \gl{FROM}, \gl{WHERE} and \gl{SELECT}
clauses of the example query are displayed apart from each other so that the
example can also illustrate the binding tuples that flow from the one clause to
the next. 

The query is evaluated within the bindings environment $\benv$ shown at the top
of Figure~\ref{fig:xmpl:sfw-bindings}. Consequently, the \gl{FROM} clause is
evaluated in the same environment. Thereafter the \gl{FROM} clause outputs the
bag of binding tuples $B^{out}_{\gl{FROM}}$, which has four binding tuples in
the example. In each binding tuple of $B^{out}_{\gl{FROM}}$, each variable of
the \gl{FROM} clause is bound to a value. There are no restrictions that a
variable binds to homogenous values across binding tuples. In the example,
\gl{x} binds to two values that are heterogeneous: some bindings of \gl{x} bind
to a number, while others to a string. It would also be possible that a variable
binds to, say, a scalar in one binding, while the same variable binds to a
complex value in another binding.

Each subsequent clause inputs a bag of binding tuples, evaluates the constituent
expressions of the clause (which may themselves contain nested SFW queries), and
outputs a bag of binding tuples that is in turn input by the next clause. For
instance, the \gl{WHERE} clause inputs the bag of binding tuples that have been
output by the \gl{FROM} clause ($B^{out}_{\gl{FROM}} = B^{in}_{\gl{WHERE}}$),
and outputs the subset thereof that satisfies the condition expression of the
\gl{WHERE} clause. This subset is the $B^{out}_{\gl{WHERE}} =
B^{in}_{\gl{SELECT}}$. 

In particular, the \gl{WHERE}'s condition is evaluated once for each input
binding tuple $b$ in $B^{in}_{\gl{WHERE}}$. In general, each evaluation is done
within the bindings environment $(\db, \env \| b)$, i.e., the concatenation of
the binding tuple $\env$ (where $\env$ is the binding environment of the SFW
query) with the binding tuple $b$ that has the variables of the \gl{FROM}
clause. In the particular example $\env \| b$ is simply $b$ since $\env =
\langle \rangle$. The condition \gt{x > y.b} is evaluated once for each of the
four input binding tuples of $B^{in}_{\gt{WHERE}}$. The variables environment of
the first evaluation is:

\[ 
\env = \langle \gt{x}:\gt{3}, \gt{y}:\gt{\{'a':1, 'b':2\}} \rangle 
\] 

\noindent The \gl{WHERE} condition evaluates to \gt{true} for the first
binding tuple of  $B^{in}_{\gt{WHERE}}$, since 

\[ \benv \vdash \gt{x > y.b} \rightarrow \gt{3 > \{'a':1, 'b':2\}.b} \rightarrow \gt{true} \]

\noindent Thus the first binding tuple of $B^{in}_{\gt{WHERE}}$ is output from
the \gl{WHERE} and is input to \gl{SELECT}.

The pattern of ``input bag of binding tuples, evaluate constituent expressions,
output bag of binding tuples'' has a few exceptions: First, the \gt{ORDER BY}
clause inputs a bag of binding tuples and outputs an array of binding tuples.
Second, a \gt{LIMIT}/\gt{OFFSET} clause need not evaluate its constituent
expression for each input binding tuple. For example a ``\gt{LIMIT} \texttt{10}"
clause that inputs an array with 100 binding tuples need not access binding
tuples 11-100. 

Finally, the \gt{SELECT} clause is responsible for converting from binding
tuples to collections of arbitrary PartiQL elements. The \gt{SELECT} inputs a
bag (or array, if \gl{ORDER BY} is present) of binding tuples, and outputs
the SFW query's result, which is a bag (resp. array) with exactly one element
for each input binding tuple. In the example, the \gl{SELECT} expressions \gt{x}
and \gt{y.a} are evaluated once for each of the input binding tuples of
$B^{in}_{\gl{SELECT}}$, which in this example happen to be just one binding
tuple.

Finally, notice that the above discussion of SFW queries did not capture the set
operators \gl{UNION}, \gl{INTERSECT} and \gl{EXCEPT}. As is the case with SQL
semantics too, the coordination of \gl{ORDER BY} with the set operators requires
attention, as discussed in Section~\ref{sec:order-with-set}.

\paragraph{PartiQL clauses as operators} 
In summary, each clause of PartiQL is an operator that inputs/outputs binding
tuples. As such, we can (and will) present the semantics of each clause
separately from the semantics of the other clauses. This is not the case in SQL:
Notably, in the presence of aggregation functions the \gl{SELECT}, \gl{HAVING}
and \gl{WHERE} cannot be interpreted in isolation; they can only be interpreted
along with the \gl{GROUP BY} clause. 

\subsection{Scoping Rules of Variables} 
\label{sec:scoping-variables}

As in any programming language, the PartiQL semantics have to deal with issues
of variable scope. For example, how are references to \gt{x} resolved in the
following query:

\begin{lstlisting}
SELECT x.a AS a
FROM db1 AS x
WHERE x.b IN (SELECT x.c FROM db2 AS x)
\end{lstlisting}

Since this is an SQL query and PartiQL is backwards compatible to SQL, it is
easy to tell that the \gt{x} in \gt{x.c} resolves to the variable \gt{x} defined
by the inner query's \gl{FROM} clause.

Technically, this scoping rule is captured by the following handling of binding
tuples. The inner \gl{FROM} clause is evaluated with a variables environment
$\env = \langle \gt{x}: \ldots\rangle$; its \gt{x} is the one defined by the
outer \gl{FROM}. Then the inner \gl{FROM} clause outputs a binding $b = \langle
\gt{x}: \ldots\rangle$; this \gt{x} is defined by thinner \gl{FROM}. Then the
\gt{x.c} is evaluated in the concatenation $\env \| b$ and because \gt{x}
appears in both $\env$ and $b$, the concatenation keeps only the \gt{x} of its
right argument. Essentially by putting $b$ as the right argument of the
concatenation, the semantics indicate that the variables of $b$ have precedence
over synonymous variables in the left argument (which was the $\env$).

Generally, given two binding tuples $b$ and $b'$, their concatenation is a
binding tuple, denoted as $b \| b'$, that has the variable bindings of both $b$
and $b'$. This creates the possibility that both $b$ and $b'$ have the same
variable $x$. In this case, the concatenation $b || b'$ will have the $b'.x$ and
its value; it will not have the $b.x$ and its value.

Note, the above does not resolve scoping issues resulting from conflicts between
the database environment and the variables environment. We resolve these
conflicts by explicit rules.

\section{Path Navigation}
\label{section:paths}

\paragraph{Tuple path navigation} A \textit{tuple path navigation}
$t.a$ from the tuple $t$ to its attribute $a$ \linequery{tuple nav} returns the
value of the attribute $a$. (We discuss below the corner case where a tuple has
multiple attributes \texttt{a}.)  $t$ is an expression but $a$ is always an
identifier \linenames{identifier}. For example:

\begin{lstlisting}
{'a': 1, 'b':2}.a (*$\eqv$*) {'a': 1, 'b':2}."a" (*$\eval$*) 1    
\end{lstlisting}

\noindent Even if there were a variable \gt{a}, bound to \gt{'b'}, the result of
the above expression would still be \gt{1}, because the identifier \gt{a} (or
\gt{"a"}) is interpreted as the ``look for the attribute named \gt{a}'' when it
follows the dot in a tuple path navigation.  The semantics of tuple path
navigation do not depend on whether the tuple is ordered or unordered by schema.

\paragraph{Array navigation} An \textit{array navigation} $a[i]$ returns
the $i$-th element \textit{when} it is applied on an array $a$ \linequery{array
nav} and $i$ is an expression that evaluates into an integer. Both $a$ and $i$
are expressions. For example:

\begin{lstlisting}
[2, 4, 6][1+1] (*$\eval$*) 6. 
\end{lstlisting}

\paragraph{Tuple navigation with array notation} The expression $a[s]$ is
a shorthand for the tuple path navigation $a.s$ when the expression $s$ is
either (a) a string literal or (b) an expression that is explicitly \gl{CAST}
into a string. For example:

\begin{lstlisting}
{'a': 1, 'b': 2}['a'] (*$\eqv$*) {'a': 1, 'b':2}.'a' (*$\eval$*) 1
\end{lstlisting}

\noindent Similarly:

\begin{lstlisting}
{'attr': 1, 'b':2}[CAST('at' || 'tr' AS STRING)] (*$\eval$*) 1
\end{lstlisting}

If $s$ is not a string literal or an expression that is cast into a string, then
$a[s]$ is evaluated as an array path navigation. Notice that in the absence of
an explicit cast, the navigation $a[e]$ evaluates as an array navigation, even
if $e$ ends up evaluating to a string. For example, let us assume that the
variable \gt{v} is bound to \gt{'at'} and the variable \gt{w} is bound to
\gt{'tr'}. Still, the expression:

% TODO determine if cases where static type is known requires a CAST.

\begin{lstlisting}
{'attr': 1, 'b':2}[v || w]
\end{lstlisting}

\noindent does not evaluate to \gt{1}. It is treated as an array navigation with
wrongly typed index and it will return \gt{missing}, for reasons explained
below. 

\paragraph{Composition of navigations} Notice that consecutive
tuple/array navigations (e.g. \texttt{r.no[1]}) navigate deeply into complex
values. Notice further that paths consisting of plain tuple and array path
navigations evaluate to a unique value.

% TODO verify that this is correct for unordered tuples.

\paragraph{Tuple navigation in tuples with duplicate attributes} When the tuple
\gt{t} has multiple attributes \gt{a}, the tuple path navigation \gt{t.a} will
return the first instance of \gt{a}.  Note that for tuples whose order is
defined by schema, this is well-defined, for unordered tuples, it is
implementation defined which attribute is returned in \textit{permissive mode}
or an error in \textit{type checking mode}, which is described in
Section~\ref{sec:tuple-path-on-wrong}.

If one wants to access all instances of \gt{a}, she should use the \gt{UNPIVOT}
feature instead (see Section~\ref{sec:unpivot}). For example, the
following query returns the list of all \gt{a} values in a tuple \gt{t}.

\begin{lstlisting}
SELECT VALUE v
FROM UNPIVOT t AS v AT attr
WHERE attr = 'a'
\end{lstlisting}
 
\subsection{Tuple path evaluation on wrongly typed data}
\label{sec:tuple-path-on-wrong}

In the case of tuple paths, since PartiQL does not assume a schema, the
semantics must also specify the return value when:

\begin{compact_enum}
\item $t$ is not a tuple (i.e., when the expression $t$ does not evaluate into a
tuple), or

\item $t$ is a tuple that does not have an $a$ attribute.
\end{compact_enum}

\paragraph{Permissive mode} PartiQL can operate in a permissive mode or
in a conventional type checking mode, where the query fails once typing errors
(such as the above mentioned ones) happen. In the permissive mode, typing errors
are typically neglected by using the semantics outlined next.

In all of the above cases PartiQL returns the special value \gt{missing}.
Recall, the \gt{missing} is different from \gt{null}. The distinction enables
PartiQL to be able to distinguish between a tuple (JSON object) that lacked an
attribute \gt{a} and a tuple (JSON object) whose \gt{a} attribute was \gt{null}.
This distinction, coupled with appropriate features on how result tuples are
constructed (see \gt{SELECT} clause in Section~\ref{sec:select-values}), enables
PartiQL to easily preserve (when needed) the distinction between absent
attribute and null-valued attribute.

For example, the expression \gt{'not a tuple'.a} and the expression \gt{\{'a':1,
'b':2\}.noSuchAttribute} evaluate to \gt{missing}.

The above semantics apply regardless of whether the tuple navigation is
accomplished via the dot notation or via the array notation. For example, the
expression \gt{\{'a':1, 'b':2\}['noSuchAttribute']} will also evaluate to
\gt{missing}.

\paragraph{Type checking mode} In the type checking mode and in the
absence of schema, PartiQL will fail when tuple path navigation is applied on
wrongly typed data.

\subsubsection{Role of schema in type checking}
\label{sec:schema-in-tuple-path}

In the presence of schema, PartiQL may return a compile-time error when the
query processor can prove that the path expression is guaranteed to
\textit{always} produce \MISSING. The extent of error detection is
implementation-specific. 

For example, in the presence of schema validation, an PartiQL query processor
can throw a compile-time error when given the path expression \gt{\{a:1,
b:2\}.c}. In a more important and common case, an PartiQL implementation can
utilize the input data schema to prove that a path expression \textit{always}
returns \MISSING and thus throw a compile-time error. For example, assume
that \gt{sometable} is an SQL table whose schema does not include an attribute
\gt{c}. Then, an PartiQL implementation may throw a compile-time error when
evaluating the query:

\begin{lstlisting}
SELECT t.a, t.c FROM sometable AS t
\end{lstlisting}

Apparently, such an PartiQL implementation is fully compatible with the behavior
of an SQL processor. Generally, if a rigid schema is explicitly present, a tuple
path navigation error can be caught during compilation time; this is the case in
SQL itself, where referring to a non-existent attribute leads to a compilation
error for the query.

Notice that operating with schema validation may not prevent all tuple path
navigations from being applied to wrongly typed data. The choice between
permissive mode versus type checking mode dictates what happens next in these
cases: If permissive, the tuple path navigation evaluates into \MISSING. If
in type checking mode, the query fails.

\subsection{Array navigation evaluation on wrongly typed data}
\label{sec:array-on-wrong}

In the permissive mode, an array navigation evaluation $a[i]$ will result into
\gt{missing} in each of the following cases:

\begin{itemize}
\item $a$ does not evaluate into an array, or
\item $i$ does not evaluate into a positive integer within the array's bounds.
\end{itemize}

For example, \gt{[1,2,3][1.0]} evaluates to \gt{missing} since \gt{1.0} is not
an integer - even though it is coercible to an integer.
 
In type checking mode, the query will fail in each one of the cases above.

\subsection{Additional Path Syntax}
\label{sec:deep-navigation}

The following additional path functionalities are explained by reduction
to the basic tuple navigation and array navigation.

\paragraph{Wildcard steps} The expression $e[*]$ reduces to (i.e., is
equivalent to):

\begin{lstlisting}
SELECT VALUE (*$v$*) FROM (*$e$*) AS (*$v$*)
\end{lstlisting}

\noindent where $v$ is a \textit{fresh variable}, i.e., a variable that does not
already appear in the query.  Similarly, when the expression $e.*$ is not a
\gl{SELECT} clause item of the form $t.*$, where $t$ is a variable, it reduces
to:

\begin{lstlisting}
SELECT VALUE (*$v$*) FROM UNPIVOT (*$e$*) AS (*$v$*)
\end{lstlisting}

\noindent where $v$ is a fresh variable. An expression $t.*$, where $t$ is a
variable and the expression appears as a \gl{SELECT} clause item, is interpreted
according to the \gl{SELECT} clause \gl{*} semantics
(Section~\ref{sec:sql-star}).

\begin{example} 
The expression:

\begin{lstlisting}
[1,2,3][*] (*$\eqv$*) SELECT VALUE v FROM [1, 2, 3] AS v
    (*$\eval$*) <<1, 2, 3>>
\end{lstlisting}

\noindent The expression:

\begin{lstlisting}
{'a':1, 'b':2}.* (*$\eqv$*) SELECT VALUE v FROM UNPIVOT {'a':1, 'b':2} AS v
    (*$\eval$*) <<1, 2>>
\end{lstlisting}

\noindent Whereas the following query:

\begin{lstlisting}
SELECT t.* FROM <<{'a':1, 'b':1}, {'a':2, 'b':2}>> AS t
    (*$\eval$*) <<{'a':1, 'b':1}, {'a':2, 'b':2}>>
\end{lstlisting}

\noindent does not do the transformation with \gl{UNPIVOT}.  If one does not
want this behavior, \gl{SELECT VALUE} can be used
(Section~\ref{sec:select-values}).
\end{example}

\paragraph{Path Expressions with Wildcards}
PartiQL also provides multi-step path expressions, called \textit{path
collection expressions}. Their semantics is a generalization of the semantics of
a path expression with a single $[*]$ or $.*$. Consider the path collection
expression:

\[ e w_1 p_1 \ldots w_n p_n \]

\noindent where $e$ is any expression; $n>0$; each \textit{wildcard step}
$w_i$ is either $[*]$ or $.*$; each \textit{series of plain path steps} $p_i$ is
a sequence of zero or more tuple path navigations or array navigations
(potentially mixed). 

Then the path collection expression is equivalent to the SFW query 

\begin{lstlisting}
SELECT VALUE (*$v_n.p_n$*)
FROM
    (*$u_1$*) (*$e$*) AS (*$v_1$*),
    (*$u_2$*) @(*$v_1.p_1$*) AS (*$v_1$*),
    (*$\ldots$*),
    (*$u_n$*) @(*$v_{n-1}.p_{n-1}$*) AS (*$v_n$*),
\end{lstlisting}

\noindent where each $v_i$ is a fresh variable and each $u_i$ is \unpivot\ if
$w_i$ is a $.*$ and it is nothing if $w_i$ is a $[*]$. Intuitively $v_i$
corresponds to the $i$-th star.

\begin{example} According to the above, consider the following query:

\begin{lstlisting}
SELECT VALUE foo FROM e.* AS foo
\end{lstlisting}

\noindent reduces to

\begin{lstlisting}
SELECT VALUE foo FROM (SELECT VALUE v FROM UNPIVOT e AS v) AS foo
\end{lstlisting}

\noindent which is equivalent to

\begin{lstlisting}
SELECT VALUE foo FROM UNPIVOT e AS foo
\end{lstlisting}

\noindent Next, consider the path collection expression:

\begin{lstlisting}
tables.items[*].product.*.nest
\end{lstlisting}

\noindent This expression reduces to 

\begin{lstlisting}
SELECT
  VALUE v2.nest
FROM
  tables.items AS v1,
  UNPIVOT @v1.product AS v2
\end{lstlisting}

\end{example}

\section{\from Clause Semantics}
\label{sec:from}

The formal semantics of a \from clause describe the collection of binding
tuples $B^{out}_{\from}$ that is output by the \from clause. The semantics
specify three cases and essentially extend the tuple calculus that underlies the
SQL semantics.

\begin{enumerate}
\item The semantics specify what is the core semantics of a \from clause with a
single \from item (Sections~\ref{sec:single-item-from} and ~\ref{sec:unpivot}).
The term ``semantics of the \from item $f$'' is synonymous to the term
``semantics of a \from clause with the single item $f$''. In either case, we
refer to the specification of the collection of binding tuples $B^{out}_{\from}$
that results from the evaluation of ``\from $f$''.

\item Then the semantics specify how multiple \from items combine, according to
the core semantics, using the join and outerjoin operations
(Sections~\ref{sec:combining-multiple-item-join},
\ref{sec:combining-multiple-item-leftjoin} and
~\ref{sec:combining-multiple-item-full-outerjoin}). 

\item Finally, the semantics specify the syntactic sugar structures that are
overlaid over the core semantics. Their primary purpose is SQL compatibility.
\end{enumerate}


\subsection{Ranging Over Bags and Arrays}
\label{sec:single-item-from}

Next we define the semantics of a \from clause that has a single \from item
and such item ranges over a bag or array. First consider the \from clause:

\begin{lstlisting}
FROM (*$a$*) AS (*$v$*) AT (*$p$*)
\end{lstlisting}

\noindent Let us call $v$ to be the \emph{element variable} and $p$ to be the
\emph{position variable}. In the normal case, $a$ is an array $[ e_0, \ldots,
e_{n-1} ]$. The \from clause outputs a bag of binding tuples. For each $e_i$,
the bag has a binding tuple $\langle v: e_i, p:i \rangle$.

\begin{example}
\label{xmpl:single-from-item-with-order}

Consider the following $\db$ (database environment):

\begin{lstlisting}
(*$\db = \langle$*)
    someOrderedTable:[
        {'a':0, 'b':0},
        {'a':1, 'b':1}
    ]
(*$\rangle$*)
\end{lstlisting}

\noindent then the following \from clause:

\begin{lstlisting}
FROM someOrderedTable AS x AT y
\end{lstlisting}

\noindent outputs the bag of binding tuples:

\begin{tabbing}
\ \ \ \=$B^{out}_{\from} = \ob $\=
    $\langle$ \lstinline|x:{'a':0, 'b':0}, y:0| $\rangle$\\
\>\>$\langle$ \lstinline|x:{'a':1, 'b':1}, y:1| $\rangle$\\
\>\>$\cb$
\end{tabbing}
\end{example}

As in SQL, the keyword \as is optional. The same applies to all cases below
where \as appears. If there is no \at clause, then the binding tuples have only
the element variable. In particular, consider:

\begin{lstlisting}
FROM (*$a$*) AS (*$v$*)
\end{lstlisting}

\noindent Normally $a$ is a collection, i.e, an array
$[ e_0, \ldots, e_{n-1} ]$ or a bag $\ob e_0, \ldots, e_{n-1} \cb$.
In either case, the \from clause outputs a bag. For each $e_i$, the bag
has a binding tuple $\langle v:e_i \rangle$.

\begin{example}
Consider again the database of Example~\ref{xmpl:single-from-item-with-order}
and then the \from clause

\begin{lstlisting}
FROM someOrderedTable AS x
\end{lstlisting}

\noindent this \from clause outputs:

\begin{tabbing}
\ \ \ \=$B^{out}_{\from} = \ob $\=
    $\langle$ \lstinline|x:{'a':0, 'b':0}| $\rangle$\\
\>\>$\langle$ \lstinline|x:{'a':1, 'b':1}| $\rangle$\\
\>\>$\cb$
\end{tabbing}
\end{example}

\subsubsection{Mistyping Cases}
\label{sec:bag-array-mistypings}

In the following cases the expression in the \from clause item has the wrong
type. Under the type checking option, all of these cases raise an error and the
query fails. Under the permissive option, the cases proceed as follows

\begin{itemize}
\item \highlight{Position variable on bags} Consider the clause:

\begin{lstlisting}
FROM (*$b$*) AS (*$v$*) AT (*$p$*)
\end{lstlisting}

\noindent and assume that $b$ is a bag $\ob e_0, \ldots, e_{n-1} \cb$. The
output is a bag with binding tuples $\langle v: e_i, p: \MISSING \rangle$. The
value \MISSING for the variable $p$ indicates that the order of elements in
the bag was meaningless. 

\item \highlight{Iteration over a scalar value} Consider the query:

\begin{lstlisting}
FROM (*$s$*) AS (*$v$*) AT (*$p$*)
\end{lstlisting}

\noindent or the query:

\begin{lstlisting}
FROM (*$s$*) AS (*$v$*)
\end{lstlisting}
 
\noindent where $s$ is a scalar value. Then $s$ coerces into the bag $\ob s
\cb$, i.e., the bag that has a single element, the $s$. The rest of the
semantics is identical to what happens when the lhs of the \from item is a bag.

\begin{example}
Consider again the database of Example~\ref{xmpl:single-from-item-with-order}
and the \from clause: 

\begin{lstlisting}
FROM someOrderedTable[0].a AS x
\end{lstlisting}

The expression \gl{someOrderedTable[0].a} evaluates to \gt{0} and,
consequently, the \from clause outputs a single binding tuple:

\begin{tabbing}
\ \ \ \=$B^{out}_{\from} = \ob
        \langle$ \lstinline|x:0| $\rangle \cb$\\
\end{tabbing}
\end{example}

\item \highlight{Iteration over a tuple value} Consider the query:

\begin{lstlisting}
FROM (*$t$*) AS (*$v$*) AT (*$p$*)
\end{lstlisting}

\noindent or the query:

\begin{lstlisting}
FROM (*$t$*) AS (*$v$*)
\end{lstlisting}
 
\noindent where $t$ is a tuple. Then $t$ coerces into the bag $\ob t \cb$

\item \highlight{Iteration over an absent value} Consider the query

\begin{lstlisting}
FROM (*$a$*) AS (*$v$*) AT (*$p$*)
\end{lstlisting}

\noindent or the query

\begin{lstlisting}
FROM (*$a$*) AS (*$v$*)
\end{lstlisting}

\noindent whereas $a$ evaluates into an \gn{absent\_value} (i.e., either
\MISSING or \NULL). In either case the \gn{absent\_value} $a$ coerces
into the bag $\ob a \cb$. Then the semantics follow the normal case.

\begin{example}
Consider again the database of Example~\ref{xmpl:single-from-item-with-order}
and the \from clause 

\begin{lstlisting}
FROM someOrderedTable[0].c AS x
\end{lstlisting}

The expression \gl{someOrderedTable[0].c} evaluates to \MISSING and,
consequently, the \from clause outputs the binding tuple:

\begin{tabbing}
\ \ \ \=$B^{out}_{\from} = \ob \langle$ \lstinline|x:MISSING| $\rangle \cb$\\
\end{tabbing}
\end{example}

\end{itemize}

\subsection{Ranging over Attribute-Value Pairs}
\label{sec:unpivot}

The \gl{UNPIVOT} clause enables ranging over the attribute-value pairs of a
tuple. The \from clause

\begin{lstlisting}
FROM UNPIVOT (*$t$*) AS (*$v$*) AT (*$a$*)
\end{lstlisting}

\noindent normally expects $t$ to be a tuple, with attribute/value pairs
$a1: v1, \ldots, a_n:v_n$. It does not matter whether the tuple is ordered
or unordered. The \from clause outputs the collection of binding tuples 

\[
    B^{out}_{\from} = \ob
        \langle v:v_1, a:a_1 \rangle 
        \ldots
        \langle v:v_n, a:a_n\rangle
    \cb 
\]

\begin{example}
Consider the $\db$:

\begin{lstlisting}
(*$\db = \langle$*)
    justATuple: {'amzn': 840.05, 'tdc': 31.06}
(*$\rangle$*)
\end{lstlisting}

\noindent The \from clause:

\begin{lstlisting}
FROM UNPIVOT justATuple AS price AT symbol
\end{lstlisting}

\noindent outputs:

\begin{tabbing}
\ \ \ \=$B^{out}_{\from} = \ob $\=
    $\langle $\lstinline|price: 840.05, symbol:'amzn'|$ \rangle$\\
\>\>$\langle $\lstinline|price: 31.06, symbol:'tdc'|$ \rangle$\\
\>\>$\cb$
\end{tabbing}
\end{example}

\subsubsection{Mistyping Cases}
\label{sec:unpivot-mistypings}

In the following cases the expression in the \from \unpivot clause item has the
``wrong" type, i.e., it is not a tuple. Under the type checking option, all of these cases raise an error
and the query fails. Under the permissive option, the cases proceed as follows:

\begin{lstlisting}
FROM UNPIVOT (*$x$*) AS (*$v$*) AT (*$n$*)
\end{lstlisting}

\noindent whereas $x$ is not a tuple and is not \MISSING, is equivalent to:

\begin{lstlisting}
FROM UNPIVOT {'_1': (*$x$*)} AS (*$v$*) AT (*$n$*)
\end{lstlisting}

\noindent Effectively, a tuple is generated for the non-tuple value.  When $x$ is \MISSING
then the above is equivalent to:

\begin{lstlisting}
FROM UNPIVOT {} AS (*$v$*) AT (*$n$*)
\end{lstlisting}

\noindent remember that a tuple cannot contain \MISSING. So the present case is equivalent to the empty tuple case.

\subsection{Combining Multiple \from Items with Comma, \CROSSJOIN, or \JOIN}
\label{sec:combining-multiple-item-join}

The \from clause expressions:

\begin{lstlisting}
(*$l$*) , (*$r$*) (*$\eqv$*)
(*$l$*) CROSS JOIN (*$r$*) (*$\eqv$*)
(*$l$*) JOIN (*$r$*) ON TRUE (*$\eqv$*)
\end{lstlisting}

\noindent have the same semantics. They combine the bag of bindings produced
from the \from item $l$ with the bag of binding tuples produced by the \from
item $r$, whereas the expression $r$ may utilize variables defined by $l$. Again, the term ``the semantics of $l\ \CROSSJOIN\ r$'' is equivalent
to the term ``the semantics of \from $l\ \CROSSJOIN\ r$''. In both cases, the
semantics specify a bag of binding tuples.

\paragraph{Associativity of $\CROSSJOIN$} We explain the $\CROSSJOIN$ and
``$,$'' as if they are left associative binary operators, despite the fact that
one can write more than two \from items without specifying grouping with
parenthesis. Since the ``$,$'' and $\CROSSJOIN$ operators are associative, we
may write (as is common in SQL):

\begin{lstlisting}
(*$f_1$*), (*$f_2$*), (*$f_3$*) (*$\eqv$*)
(*$f_1$*) CROSS JOIN (*$f_2$*) CROSS JOIN (*$f_3$*) (*$\eqv$*)
(*$f_1$*) JOIN (*$f_2$*) ON TRUE JOIN (*$f_3$*) ON TRUE (*$\eqv$*)
((*$f_1$*), (*$f_2$*)), (*$f_3$*) (*$\eqv$*)
((*$f_1$*) CROSS JOIN (*$f_2$*)) CROSS JOIN (*$f_3$*) (*$\eqv$*)
((*$f_1$*) JOIN (*$f_2$*) ON TRUE) JOIN (*$f_3$*) ON TRUE (*$\eqv$*)
\end{lstlisting}

\paragraph{Semantics} Consider the following:

\begin{lstlisting}
(*$l$*) CROSS JOIN (*r*)
\end{lstlisting}

\noindent unlike SQL, the rhs $r$ of the expression may use variables defined by
the lhs item $l$.  The result of this expression for a database environment
$\db$ and variables environment $\env$ is the bag of binding tuples produced by
the following pseudo-code. The pseudo-code uses the function $\evalf(\db, \env,
e)$ that evaluates the expression $e$ within the environments $\db$ and $\env$,
i.e., $\db, \env \vdash e \rightarrow \evalf(\db, \env, e)$.

\begin{tabbing}
\ \ \ \=for \=each binding tuple $b^l$ in $\evalf(\db, \env, l)$\\
\>\>for \=each binding $b^r$ in $\evalf(\db, (\env \| b^l), r)$\\
\>\>\>add $b^l \| b^r$ to the output bag
\end{tabbing}

In other words, the ``$l\ \CROSSJOIN\ r$'' outputs all binding tuples $b = b^l
\| b^r$, where $b^l \in \evalf(\db, \env, l)$ and $b^r \in \evalf(\db, (\env \|
b^l), r)$. The key extension to SQL is that $r$ is evaluated in the variables
environment $\env \| b^l$, i.e., it can use the variables that were defined by
$l$. The details of the variable scoping aspects are described in
Section~\ref{sec:scoping-variables}.

\begin{example}
This example simply reminds the tuple calculus explanation of the
\from SQL semantics. It does not yet endeavor into special aspects
of PartiQL. Consider the following database, which is conventional SQL:

\begin{lstlisting}
(*$\db = \langle$*)
    customers: [
        {'id': 5, 'name': 'Joe'},
        {'id': 7, 'name': 'Mary'}
    ],
    orders: [
        {'custId': 7, 'productId': 101},
        {'custId': 7, 'productId': 523}
    ]
(*$\rangle$*)
\end{lstlisting}

\noindent Then consider the following \from clause, which could be coming from
a conventional SQL query:

\begin{lstlisting}
FROM customers AS c, orders AS o
\end{lstlisting}

\noindent Note that in PartiQL this could also be written using the \CROSSJOIN
keyword, and presumably, one would put the sensible equality condition
\gt{c.id=o.custId} in the \gl{WHERE} clause. At any rate, this \from clause
outputs the bag of binding tuples:

\begin{tabbing}
\ \ \ \=$B^{out}_{\from} = \ob $\=
    $\langle$ \lstinline|c: {'id': 5, 'name': 'Joe'}, o: {'custId': 7, 'productId': 101}| $\rangle$\\
\>\>$\langle$ \lstinline|c: {'id': 5, 'name': 'Joe'}, o: {'custId': 7, 'productId': 523}| $\rangle$\\
\>\>$\langle$ \lstinline|c: {'id': 7, 'name': 'Mary'}, o: {'custId': 7, 'productId': 101}| $\rangle$\\
\>\>$\langle$ \lstinline|c: {'id': 7, 'name': 'Mary'}, o: {'custId': 7, 'productId': 523}| $\rangle$\\
\>\>$\cb$
\end{tabbing}

\end{example}

\noindent Due to scoping rules that will be justified and elaborated in
Section~\ref{sec:variable-scoping}, when the rhs of a $\CROSSJOIN$ is a path or a
function that uses a variable named $n$, such variable must be referred as
$\gt{@}n$. 

\begin{example}
Consider the database:

\begin{lstlisting}
(*$\db = \langle$*)
    sensors: [
        {'readings': [{'v': 1.3}, {'v': 2}]},
        {'readings': [{'v': 0.7}, {'v': 0.8}, {'v': 0.9}]}
    ]
(*$\rangle$*)
\end{lstlisting}

\noindent Intuitively, the following \from clause unnests the tuples that are
nested within the \gt{readings}.

\begin{lstlisting}
FROM sensors AS s, s.readings AS r
\end{lstlisting}

\begin{tabbing}
\ \ \ $B^{out}_{\from} = \ob$\=
  $\langle$ \lstinline|s: {'readings': [{'v': 1.3}, {'v': 2}]}, r: {v:1.3}| $\rangle$,\\
\>$\langle$ \lstinline|s: {'readings': [{'v': 1.3}, {'v': 2}]}, r: {v:2}| $\rangle$,\\
\>$\langle$ \lstinline|s: {'readings': [{'v': 0.7}, {'v': 0.8}, {'v': 0.9}]}, r: {'v':0.7}| $\rangle$,\\
\>$\langle$ \lstinline|s: {'readings': [{'v': 0.7}, {'v': 0.8}, {'v': 0.9}]}, r: {'v':0.8}| $\rangle$,\\
\>$\langle$ \lstinline|s: {'readings': [{'v': 0.7}, {'v': 0.8}, {'v': 0.9}]}, r: {'v':0.9}| $\rangle$,\\
\>$\cb$
\end{tabbing}
\end{example}

\subsection{Combining Multiple \from Items with \LEFTJOIN}
\label{sec:combining-multiple-item-leftjoin}

The \from clause expression:

\begin{lstlisting}
(*$l$*) LEFT CROSS JOIN (*$r$*) (*$\eqv$*)
(*$l$*) LEFT JOIN (*$r$*) ON TRUE
\end{lstlisting}

\noindent replicates SQL's \gl{LEFT JOIN} functionality and, in addition, it
also works for the case where the lhs of $r$ uses variables defined from $l$.

\yannis{OS: Do you really mean to make the CROSS necessary in the absence of ON?}

\almann{Yes, there is a parser problem if we don't, this was raised by Redshift
a while ago.}

Let's assume that the variables defined by $r$ are $v^r_1, \ldots, v^r_n$. The
result of evaluating $l\ \LEFTCJOIN\ r$ in environments $\db$ and $\env$ is the
bag of binding tuples produced by the following pseudocode, which also uses the
$\evalf$ function (See Section~\ref{sec:combining-multiple-item-join}). 

\begin{tabbing}
\ \ \ \=for \=each binding $b^l$ in $\evalf(\db, \env, l)$\\
\>\>$B^r \leftarrow \evalf(\db, (\env \| b^l), r)$\\
\>\>if \=$B^r$ is the empty bag\\
\>\>\>add $b^l \| \langle v^r_1:\NULL \ldots v^r_n:\NULL \rangle$ to the output bag \\
\>\>else\\
\>\>\>for \=each binding $b^r$ in $B^r$\\
\>\>\>\>add $b^l \| b^r$ to the output bag
\end{tabbing}

\begin{example}
Consider the database:

\begin{lstlisting}
(*$\db = \langle$*)
    sensors: [
        {'readings': [{'v':1.3}, {'v':2}]}
        {'readings': [{'v':0.7}, {'v':0.8}, {'v':0.9}]},
        {'readings': []}
      ]
(*$\rangle$*)
\end{lstlisting}

\noindent Notice that the value of the last tuple's \gt{readings} attribute is the empty
array. The following \from clause unnests the tuples that are nested within
the \gt{readings} but will also keep around the tuple with the empty
\gt{readings}. (See the last binding tuple.)

\begin{lstlisting}
FROM sensors AS s LEFT CROSS JOIN s.readings AS r
\end{lstlisting}

\begin{tabbing}
\ \ \ $B^{out}_{\from} = \ob$\=
  $\langle$ \lstinline|s: {'readings': [{'v':1.3}, {'v':2}]}, r: {'v':1.3}| $\rangle$,\\
\>$\langle$ \lstinline|s: {'readings': [{'v':1.3}, {'v':2}]}, r: {'v':2}| $\rangle$,\\
\>$\langle$ \lstinline|s: {'readings': [{'v':0.7}, {'v':0.8}, {'v':0.9}]}, r: {'v':0.7}| $\rangle$,\\
\>$\langle$ \lstinline|s: {'readings': [{'v':0.7}, {'v':0.8}, {'v':0.9}]}, r: {'v':0.8}| $\rangle$,\\
\>$\langle$ \lstinline|s: {'readings': [{'v':0.7}, {'v':0.8}, {'v':0.9}]}, r: {'v':0.9}| $\rangle$,\\
\>$\langle$ \lstinline|s: {'readings': []}, r: NULL| $\rangle$,\\
\>$\cb$
\end{tabbing}
\end{example}

\subsection{Combining Multiple \from Items with \FULLJOIN}
\label{sec:combining-multiple-item-full-outerjoin}

The \from clause expression:

\begin{lstlisting}
(*$l$*) FULL JOIN (*$r$*) ON *$c$*
\end{lstlisting}

\noindent replicates SQL's \gl{FULL JOIN} functionality. It
assumes that (alike SQL) the lhs of $r$ does not use variables defined from $l$. 
Thus, we do not discuss further.

\subsection{Expanding \JOIN and \LEFTJOIN with \on}
\label{sec:rewriting-on}

In compliance to SQL, the \JOIN and \LEFTJOIN have an optional \on
clause. The semantics of \gl{ON} can be explained as syntactic sugar over the core
PartiQL. They can also be explained by a simple extension of the semantics of
Sections~\ref{sec:combining-multiple-item-join},
~\ref{sec:combining-multiple-item-leftjoin}, and
~\ref{sec:combining-multiple-item-full-outerjoin}. The semantics of:

\begin{lstlisting}
(*$l$*) JOIN (*$r$*) ON (*$c$*)
\end{lstlisting}

\noindent are the following modification of the pseudocode of
Section~\ref{sec:combining-multiple-item-join}. The modification is the
inclusion of the underlined line.

\begin{tabbing}
for \=each binding $b^l$ in $\evalf(\db, \env, l)$\\
\>for \=each binding $b^r$ in $\evalf(\db, (\env \| b^l), r)$\\
\>\>\underline{if $\evalf(\db, (\env \| b^l \| b^r), c)$ is true}\\
\>\>\ \ \ add $b^l \| b^r$ to the output bag
\end{tabbing}

\noindent The semantics of:

\begin{lstlisting}
(*$l$*) LEFT JOIN (*$r$*) ON (*$c$*)
\end{lstlisting}

\noindent are the following modification of the pseudocode of
Section~\ref{sec:combining-multiple-item-leftjoin}. In essence, the \LEFTJOIN
\on outputs a tuple padded with \NULL whenever there is no binding of $r$
that satisfies the condition $c$.

\begin{tabbing}
for \=each binding $b^l$ in $\evalf(\db, \env, l)$\\
\>$B^r \leftarrow \evalf(\db, (\env \| b^l), r)$\\
\>$Q^r \leftarrow \ob \cb$\\
\>for \=each binding $b^r$ in $B^r$\\
\>\>if \=$\evalf(\db, (\env \| b^l \| b^r), c)$ is true\\
\>\>\>add $b^r$ in $Q^r$\\
\>if \=$Q^r$ is the empty bag\\
\>\>add $b^l \| \langle v^r_1:\NULL \ldots v^r_n:\NULL\rangle$ to the output bag \\
\>else\\
\>\>for \=each binding $b^r$ in $Q^r$\\
\>\>\>add $b^l \| b^r$ to the output bag
\end{tabbing}

\subsection{SQL's \gl{LATERAL}}
\label{sec:lateral}

SQL 2003 used the \gl{LATERAL} keyword to correlate \from clause items. In
the interest of compatibility with SQL, PartiQL also allows the use of the
keyword \gl{LATERAL}, though it does not do anything more than the comma itself
would do. That is ``$l\ \gl{, LATERAL}\ r$'' is equivalent to ``$l\ \gl{,}\
r$''.


\section{\select\ clauses}
\label{sec:select-values}

Core PartiQL SFW queries have a \gl{SELECT VALUE} clause (in lieu of SQL's
\gl{SELECT} clause) that can create outputs that are collections of anything
(e.g., collections of tuples, collections of scalars, collections of arrays,
collections of mixed type elements, etc.) Section~\ref{sec:select-values-core}
describes the \gl{SELECT VALUE} clause.

SQL's well-known \gl{SELECT} clause can be used as a mere syntactic sugar over
\gl{SELECT VALUE}, when we consider the top-level query. In particular,
Section~\ref{sec:sql-select} shows that SQL's \gl{SELECT} is the special case
where the \gl{SELECT VALUE} produces collections of tuples. Furthermore, when
\select is used as a subquery it is coerced into a scalar or a tuple, in the
ways that SQL coerces the results of subqueries.

Section~\ref{sec:pivot} describes \gl{PIVOT}, which can be used
instead of \gl{SELECT VALUE}. \gl{PIVOT} creates a tuple, with a data
dependent number of attribute/value pairs, where not only the values but the
attributes as well could be originating from the data found in the binding
tuples.

\subsection{\select \values core clause}
\label{sec:select-values-core}
The \select \values clause inputs a bag of binding tuples or an array of
binding tuples (from the other clauses of the SQL query) and outputs a bag or an
array. For example, if the query only has \select \values, \from, and
\gl{WHERE} clauses, then the bindings that are output by the \gl{WHERE} clause
are input by the \select \values clause. Unlike SQL, the output of a \select
\values clause need not be a bag or array of tuples. It is a bag or array of
any kind of PartiQL values. For example, it may be a bag of integers, or a
bag of arrays, etc. Indeed, the values may be heterogeneous. For example, the
output may even be a bag that has both integers and arrays. 

The core PartiQL clause:

\begin{lstlisting}
SELECT VALUE (*$e$*)
\end{lstlisting}

\noindent inputs a bag or an array (depending on the presence or non-presence of
\gl{ORDER BY}) of binding tuples and outputs respectively a bag or an array of
values. Let $\db$ and $\env$ be the environments of the SFW query. For each
input binding tuple $b \in B^{in}_{\gl{SELECT}}$, \gl{SELECT VALUE} outputs a
value $v$, where $\db, (\env \| b) \vdash e \rightarrow v$. Notice that PartiQL
expressions $e$ (Figure~\ref{figure:query:bnf} \linequery{expression query}) will typically be
tuple or array or bag constructors \linequery{constructors}, 
which enable the construction of respective results.
In general $e$ can be any expression.

\begin{example}
This example illustrates a \gl{SELECT VALUE} that creates a collection of
numbers.

\begin{lstlisting}
SELECT VALUE 2*x.a
FROM [{'a':1}, {'a':2}, {'a':3}] as x
\end{lstlisting}

The result is \lstinline|<<2, 4, 6>>|.
\end{example}

\subsubsection{Tuple constructors} 
\label{sec:tuple-constructor}

A \textit{tuple constructor} is of the form

\begin{lstlisting}
{(*$a_1$*):(*$e_1$*), (*$\ldots$*), (*$a_n$*):(*$e_n$*)}
\end{lstlisting}

\noindent whereas $a_1 \ldots a_n, e_1 \ldots e_n$ are expressions, potentially
being themselves constructors.

\begin{example}
The query:

\begin{lstlisting}
SELECT VALUE {'a':v.a, 'b':v.b}
FROM [{'a':1, 'b':1}, {'a':2, 'b':2}] AS v
\end{lstlisting}

\noindent results into \lstinline|<<{'a':1, 'b':1}, {'a':2, 'b':2}>>|.
\end{example}

\paragraph{Treatment of mistyped attribute names}

It is possible that an expression $a_i$ that computes an attribute name results
into a non-string, i.e., a value that is not a legitimate attribute name. In
such cases, under the permissive mode the attribute-value pair will be
dismissed. Under the type checking mode the query will fail.

\begin{example}
In the permissive mode, the query:

\begin{lstlisting}
SELECT VALUE {v.a: v.b}
FROM [{'a':'legit', 'b':1}, {'a':400, 'b':2}] AS v
\end{lstlisting}

\noindent results into \lstinline|<<{'legit':1}, {}>>|. Notice that the attempt
to create an attribute named \gt{400} failed, thus leading to a tuple with no
attributes.
\end{example}

\paragraph{Treatment of duplicate attribute names}

It is possible that the constructed tuples contain twice or more the same
attribute name. 

\begin{example}
The query:

\begin{lstlisting}
SELECT VALUE {v.a: v.b,  v.c: v.d}
FROM [{'a':'same', 'b':1, 'c':'same', 'd':2}] AS v
\end{lstlisting}

\noindent results into \lstinline|<<{'same':1, 'same':2}>>|. 
Recall, a \texttt{same} path will only pick one of the two values. 
\end{example}

\subsubsection{Array Constructors} 
\label{sec:array-constructor}

An array constructor has the form:

\begin{lstlisting}
[(*$e_0$*), (*$\ldots$*), (*$e_{n-1}$*)]
\end{lstlisting}

\noindent where $e_1 \ldots e_{n-1}$ are expressions. Notice that the arrays
produced by such constructor will always have size $n$.

\begin{example}
The query:

\begin{lstlisting}
SELECT VALUE [v.a, v.b]
FROM [{'a':1, 'b':1}, {'a':2, 'b':2}] AS V
\end{lstlisting}

\noindent results into \lstinline|<<[1, 1], [2, 2]>>| 
\end{example}

In the interest of compatibility to SQL, PartiQL also allows array constructors
to be denoted with parentheses instead of brackets, when there are at least two
elements in the array, i.e., $n \geq 2$:

\begin{lstlisting}
((*$e_0$*), (*$\ldots$*), (*$e_{n-1}$*))
\end{lstlisting}

See Section~\ref{sec:select-coercion-array} for uses of this feature in SQL
compatibility.

\subsubsection{Bag Constructors}
A bag constructor has the form:

\begin{lstlisting}
<<(*$e_0$*), (*$\ldots$*), (*$e_{n-1}$*)>>
\end{lstlisting}

\noindent where $e_1 \ldots e_n$ are expressions.

\begin{example}
The query:

\begin{lstlisting}
SELECT VALUE <<v.a, v.b>>
FROM [{'a':1, 'b':1}, {'a':2, 'b':2}] AS v
\end{lstlisting}

\noindent results into \lstinline|<< <<1, 1>>, <<2, 2>> >>|. 
\end{example}

\subsubsection{Treatment of \MISSING in \gl{SELECT VALUE}}
\label{sec:treatment-missing-select-value}

\MISSING may behave differently from \NULL and differently from scalars.
The following itemizes the behavior of \MISSING in a number of cases:

\begin{itemize}
\item \textbf{when constructing tuples} Whenever during tuple construction an
attribute value evaluates to \MISSING, then the particular attribute/value is
omitted from the constructed tuple.

\begin{example} The query

\begin{lstlisting}
SELECT VALUE {'a':v.a, 'b':v.b}
FROM [{'a':1, 'b':1}, {'a':2}]
\end{lstlisting}

\noindent results into \lstinline|<<{'a':1, 'b':1}, {'a':2}>>|.
\end{example}

\item \textbf{when constructing arrays} Whenever an array element evaluates to
\MISSING, the resulting array will contain a \MISSING.

\begin{example}
The query

\begin{lstlisting}
SELECT VALUE [v.a, v.b]
FROM [{'a':1, 'b':1}, {'a':2}]
\end{lstlisting}

\noindent results into \lstinline|<<[1, 1], [2, MISSING]>>|.
\end{example}

Upon output serialization the \MISSING will convert to the symbol that the
serialization has chosen for serializing \MISSING.

\item \textbf{when constructing bags} Whenever an element of a bag evaluates to
\MISSING, the resulting bag will contain a corresponding \MISSING. 

\begin{example}
The query

\begin{lstlisting}
SELECT VALUE v.b
FROM [{'a':1, 'b':1}, {'a':2}]
\end{lstlisting}

\noindent results into \lstinline|<<1, MISSING>>| because \lstinline|{'a':2}.b|
evaluated to \MISSING.
\end{example}

\begin{example}
The query

\begin{lstlisting}
SELECT VALUE <<v.a, v.b>>
FROM [{'a':1, 'b':1}, {'a':2}]
\end{lstlisting}

\noindent results into \lstinline|<< <<1, 1>>, <<2, MISSING>> >>|.
\end{example}
\end{itemize}

\subsection{Pivoting a Collection into a Variable-Width Tuple}
\label{sec:pivot}
The \gl{PIVOT} clause may appear in lieu of \gl{SELECT VALUE}. The
\gl{PIVOT} clause outputs a tuple; in contrast, a \gl{SELECT VALUE}
outputs a collection (bag or array). The syntax is

\begin{lstlisting}
PIVOT (*$e_v$*) AT (*$e_a$*) (*$c$*)
\end{lstlisting}

\noindent where the other clauses, $c$, are the usual \gl{FROM}, \gl{WHERE},
etc. The semantics are similar to \gl{SELECT VALUE}. Let $\db$ and $\env$ be the
environments of the SFW query. For each input binding tuple $b \in
B^{in}_{\gl{PIVOT}}$, \gl{PIVOT} outputs an attribute name/value pair $a, v$,
where the name $a$ is the result of $e_a$ and the value $v$ is the result of
$e_v$. (Technically, $\db, (\env \| b) \vdash e_a \mapsto a$ and $\db, (\env \|
b) \vdash e_v \mapsto v$.) Regardless of whether $B^{in}_{\gl{PIVOT}}$ is a bag
(i.e., the SFW query did not have an \gl{ORDER BY}) or an array (i.e., the SFW
query had an \gl{ORDER BY}), the output tuple is unordered. Schema may be
applied extantly to obtain an ordered tuple.

\begin{example}
The query:

\begin{lstlisting}
PIVOT t.price AT t.symbol
FROM [{'symbol':'tdc', 'price': 31.52}, {'symbol': 'amzn', 'price': 840.05}] AS t
\end{lstlisting}

\noindent results into the tuple \lstinline|{'tdc':31.52, 'amzn':840.05}|.
\end{example}

The treatment of \MISSING is same to the treatment of \MISSING by \select
\values (Section~ref{{sec:tuple-constructor}}). Namely, whenever an attribute
name or attribute value evaluates to \MISSING, the corresponding attribute
name/value pair will not appear in the tuple.

\begin{example}
The query

\begin{lstlisting}
PIVOT t.price AT t.symbol
FROM [{'symbol':25, 'price':31.52}, {'symbol':'amzn', 'price':840.05}] AS t
\end{lstlisting}

\noindent results into the tuple \lstinline|{'amzn': 840.05}| since
\lstinline|25| is not a legitimate attribute name.
\end{example}

\subsection{SQL \select list as Syntactic Sugar of \select \values}
\label{sec:sql-select}

\subsubsection{\select Without \gl{*}}
\label{sec:select-without-star}

The SQL syntax:

\begin{lstlisting}
SELECT (*$e_1$*) AS (*$a_1$*), (*$\ldots$*), (*$e_n$*) AS (*$a_n$*)
\end{lstlisting}

\noindent is syntactic sugar for:

\begin{lstlisting}
SELECT VALUE {(*$a'_1$*):(*$e_1$*), (*$\ldots$*), (*$a'_n$*):(*$e_n$*)}
\end{lstlisting}    

\noindent whereas if the attribute name $a_i$ is written as an identifier (e.g.,
\lstinline|a| or \lstinline|"a"|) it is replaced by a single-quoted form $a'_i$
(e.g., \lstinline|'a'|).

When the expression $e_i$ is of the form $e_i'\gl{.}n$ (i.e. a path that
navigates into tuple attribute $n$), PartiQL follows SQL in allowing the
attribute name to be optional. In this case, 

\begin{lstlisting}
SELECT (*$\ldots e_i$*).(*$n \ldots$*)
\end{lstlisting}

\noindent is equivalent to 

\begin{lstlisting}
SELECT (*$\ldots e_i$*).(*$n \ldots$*) AS (*$n$*)
\end{lstlisting}

In the case that the expression $e_i$ is not of the form $e_i'\gl{.}n$ the
clause:

\begin{lstlisting}
SELECT (*$\ldots e_i$ \ldots*)
\end{lstlisting}

\noindent is equivalent to 

\begin{lstlisting}
SELECT (*$\ldots e_i$*) AS (*$a_i \ldots$*)
\end{lstlisting}

\noindent where $a_i$ is a system-generated name. SQL and PartiQL do not
provide a standard convention. 

\subsubsection{SQL's \gl{*}} 
\label{sec:sql-star}

Consider a query whose \from\ defines a variable $x$ that has no schema and the
\select\ clause includes at least one $x.\gl{*}$. Let us first consider the
simpler case where the \gl{SELECT} clause is a single item $x$\gl{.*}. Then the
clause

\begin{lstlisting}
SELECT (*$x$*).*
\end{lstlisting}

\noindent reduces to

\begin{lstlisting}
SELECT VALUE CASE WHEN NOT (*$x$*) IS TUPLE THEN {'_1': (*$x$*)} ELSE (*$x$*) END 
\end{lstlisting}

\noindent Notice that PartiQL extends the \gl{.*} to also operate on $x$
bindings that are not tuples. These are converted to singleton tuples with a
synthetic name.

\begin{example}
The query

\begin{lstlisting}
SELECT x.*
FROM [{'a':1, 'b':1}, {'a':2}, 'foo'] AS x
\end{lstlisting}

\noindent results into \lstinline|<< {'a':1, 'b':1}, {'a':2}, {'_1':'foo'} >>|.
Notice that the input has a non-tuple that was converted to a tuple with a
synthetic attribute name \lstinline|_1|, this is because the result of a
traditional \gl{SELECT} is always a container of tuples.
\end{example}

We generalize the semantics of a \gl{SELECT} list, where at least one of the
items is a \gl{.*} item, we use the function \tupleunion. When all of $t_1, t_2,
\ldots, t_n$ are tuples $\tupleunion(t_1, t_2,\ldots,t_n)$ outputs a tuple $t$
such that for each attribute name/value pair $n:v$ of any $t_i$, the tuple $t$
has a respective $n:v$. Notice the possibility that the output $t$ has duplicate
attribute names because either (i) two different inputs $t_i$ and $t_j$ had the
same attribute name, or (ii) because an input $t_i$ already had a duplicate
attribute name. 

Using \tupleunion, we rewrite the \select clause as illustrated by the following
example, which has two \gl{.*} items and one conventional item. The
generalization to more items, of either kind should be obvious. Notice that if
$v_1$ (resp. $v_3$) is bound to a non-tuple value $v$, then it is treated as if
it were the tuple \gl{\{'\_1':$v_1$\}} (resp. \gl{\{'\_2':$v_3$\}}. 

\begin{lstlisting}
SELECT (*$v_1$*).*, (*$e_2$*) AS (*$a$*), (*$v_3$*).*
\end{lstlisting}

\noindent is equivalent to

\begin{lstlisting}
SELECT VALUE TUPLEUNION(
    CASE WHEN (*$v_1$*) IS TUPLE THEN (*$v_1$*) ELSE {'_1': (*$v_1$*)} END,
    {'(*$a$*)':(*$e_2$*)},
    CASE WHEN (*$v_3$*) IS TUPLE THEN (*$v_3$*) ELSE {'_2': (*$v_3$*)} END
)
\end{lstlisting}

Notice that the attribute names \gl{\_1}, \gl{\_2} have been invented.


\subsection{Examples with combinations of multiple features}

\begin{example} 
\label{xmpl:nesting-readings}
A SFW subquery may appear in the \gl{SELECT VALUE} clause of a query, enabling
the creation of nested results.  

Consider the database

\begin{tabbing}
\ \ \ \=\gt{sensors : [}\=\gt{\{'sensor':1\},}\\
\>\>\gt{\{'sensor':2\}}\\
\>\>\gt{]}\\
\ \ \ \=\gt{logs: [}\=\gt{\{'sensor':1, 'co':0.4\},}\\
\>\>\gt{\{'sensor':1, 'co':0.2\},}\\
\>\>\gt{\{'sensor':2, 'co':0.3\}}\\
\>\>\gt{]}
\end{tabbing}

The query

\begin{tabbing}
\ \ \ \=\gt{SELECT VALUE \{}\=\gt{'sensor': s.sensor,}\\
\>\>\gt{'readings': (}\=\gt{SELECT VALUE l.co}\\
\>\>\>\gt{FROM logs AS l}\\         
\>\>\>\gt{WHERE l.sensor = s.sensor}\\ 
\>\>\>\gt{)}\\
\>\>\!\!\gt{\}}\\
\>\gt{FROM sensors AS s}
\end{tabbing}

\noindent results into

\begin{tabbing}
\ \ \ \=\gt{\ob }\=\gt{\{'sensor':1, 'readings':\ob 0.4, 0.2 \cb\},}\\
\>\>\gt{\{'sensor':2, 'readings':\ob 0.3 \cb\}}\\
\>\gt{\cb}
\end{tabbing}

\noindent Notice that each tuple of the result has a nested array, which has
been created by the inner \gl{SELECT VALUE}. 

The query could also have been written using \gl{SELECT} (instead of \gl{SELECT
VALUE}) for the outer query, as follows:

\begin{tabbing}
\ \ \ \=\gt{SELECT }\=\gt{s.sensor AS sensor,}\\
\>\>\gt{(}\=\gt{SELECT VALUE l.co}\\
\>\>\>\gt{FROM logs AS l}\\         
\>\>\>\gt{WHERE l.sensor = s.sensor}\\ 
\>\>\>\gt{) AS readings}\\
\>\gt{FROM sensors AS s}
\end{tabbing}

Furthermore, the \gt{AS sensor} could be ommitted (as in SQL).
\end{example}

\begin{example}
This example shows how the combined action of \gl{UNPIVOT} and \gl{PIVOT}
enables to analyze the attribute names. Consider the following database that has
a sequence of measurements of various gases.

\begin{tabbing}
\ \ \ \gt{sensors : [}\=\gt{\{'no2':0.6, 'co':0.7, 'co2':0.5\},}\\
\>\gt{\{'no2':0.5, 'co':0.4, 'co2':1.3\}}\\
\>\gt{]}
\end{tabbing}

The following query keeps only the carbon oxides.
\footnote{The query author is pretty weak in chemistry and cannot enumerate the
carbon oxides explicitly in her query.}

\begin{tabbing}
\ \ \ \=\gt{SELECT VALUE (}\=\gt{PIVOT v AT g}\\
\>\>\gt{FROM UNPIVOT r AS v AT g}\\
\>\>\gt{WHERE g LIKE 'co\%')}\\
\>\gt{FROM sensors AS r}
\end{tabbing}

The result is 

\begin{tabbing}
\ \ \ \gt{[}\=\gt{\{'co':0.7, 'co2':0.5\},}\\
\>\gt{\{'co':0.4, 'co2':1.3\}}\\
\>\gt{]}
\end{tabbing}

Intuitively, the \gl{UNPIVOT} turns every instance of the tuple \gt{t} into a
collection. The \gl{WHERE} filters the collections. The \gl{PIVOT} pivots the
filtered collections back into tuples.
\end{example}

\section{Functions}
\label{sec:preds-and-fns}

The semantics of predicates (i.e., functions returning booleans) and
(non-aggregate) functions in PartiQL are identical to those of SQL when their
inputs are those that are allowed by SQL. PartiQL makes the following extensions
for the cases where the inputs are beyond those allowed by SQL.
 
\subsection{Inputs with wrong types:}
\label{sec:fns-with-wrong-inputs}

\almann{TODO confirm that this is implemented correctly in the implementation}

Unlike SQL where typing issues can be detected during query compilation, the
permissive option of PartiQL has to define semantics for the case where the
inputs of a function are not compatible with the function/predicate arguments.
Furthermore, PartiQL facilitates propagating missing input attributes to
respective missing output attributes.

Alike SQL, all functions have input argument types that they conform to. For example, the
function \gl{log} expects numbers. All functions return \MISSING when they
input data whose types do not conform to the input argument types. Since no
function (other than \gl{IS MISSING}) has \MISSING as an input argument
type, it follows that all functions return \MISSING when one of their inputs
is \MISSING.

\begin{example} 
\label{xmpl:missing-on-wrong-types}

The query

\begin{tabbing}
\gl{SELECT VALUE \{'a':3*v.a, 'b':3*(CAST v.b AS INTEGER)\}}\\
\gl{FROM [\{'a':1, 'b':'1'\}, \{'a':2\}] v}
\end{tabbing}

\noindent results into \gl{\ob \{'a':3, 'b':3\}, \{'a':6\} \cb}. Notice how the
missing \gt{b} attribute in the input leads to a respective missing attribute in
the output.
\end{example}

\begin{example}
Each one of these expressions returns \MISSING: \gt{5 + missing}, \gt{5 >
'a'}, \gt{NOT \{a:1\}}.
\end{example}

\subsubsection{Equality}
\label{sec:equality}
Equality never fails in the type-checking mode and never returns \MISSING in
the permissive mode. Instead, it can compare values of any two types, according
to the rules of the PartiQL type system. For example, \gt{5 = 'a'} is
\gl{false}.

Since PartiQL variables may bind to composite values (collections, tuples),
PartiQL extends the semantics of equality for these cases. In particular,
equality in PartiQL is {\em deep equality}, defined as follows: 

\begin{enumerate}
\item Given two arrays $x$ and $y$ that have the same length $l$, the result of
$x=y$ is the result of 
\[\gt{eqg}(x[0], y[0])\ \gt{AND}\ \ldots\ \gt{AND}\ \gt{eqg}(x[l], y[l])\]
The \gt{eqg}, unlike the \gt{=}, returns true when a \NULL is compared to a \NULL 
or a \MISSING to a \MISSING. When the arrays $x$ and
$y$ do not have the same length, the $x=y$ is \gt{false}.

\almann{Yannis, it is a bit of a tragedy, that we cannot use equality for
nested nulls/missing--the implementation does this for arrays/tuples/bags.\\
I see no reason to make \NULL and \MISSING short-circuit in recursive
equality--yes it is less orthogonal, but really not useful.\\
Yannis: Fixed. I changed to using the \gl{eqg}. Fixed the following examples also.}

\item A similar straightforward equality applies to tuples: 
They have to have the same attributes. Then equality $t_1 =
t_2$ is true if 
\[\gt{eqg}(t_1.a_, t_2.a_1)\ \gl{AND}\ \ldots\ \gl{AND}\ \gt{eqg}(t_1.a_n, t_2.a_n)\]
where $a_1, \ldots, a_n$ are the attributes that appear in $t_1$ and
$t_2$.%

\almann{TODO define this in terms of duplicate attribute names/values.
Yannis: Are we going too far with duplicate attribute names?
I thought the idea was to let them happen, let them propagate
but do not redefine the language primitives to accomodate this possibility.
I can see hard time supporting it beyond pass-thru.
}
 
\item Equality for bags is similarly straightforward: Two bags $x$ and $y$ are
equal if and only if every element $e$ of $x$ that appears $n$ times in $x$ also
appears $n$ times in $y$.
\end{enumerate}

\begin{example}

The following are true:

\begin{tabbing}
\ \ \ \=\gt{\ob 3, 2, 4, 2 \cb = \ob 2, 2, 3, 4 \cb}\\
\>\gt{\{'a':1, 'b':2\} = \{'b':2, 'a':1\}}\\
\>\gt{\{'a':[0,1], 'b':2\} = \{'b':2, 'a':[0,1]\}}\\
\end{tabbing}

The following are false:

\begin{tabbing}
\ \ \ \=\gt{\ob 3, 4, 2 \cb = \ob 2, 2, 3, 4 \cb}\\
\>\gt{\{'a':1, 'b':2\} = \{'a':1\}}\\
\>\gt{\{'a':[0,1], 'b':2\} = \{'b':2, 'a':[0,1,2]\}}\\
\end{tabbing}

The following are also false. 

\begin{tabbing}
\ \ \ \=\gt{\{'a':1, 'b':2\} = \{'a':1\}}\\
\>\gt{\{'a':1, 'b':2\} = \{'a':1, 'b':null\}}\\
\>\gt{\{'a':[0,1], 'b':2\} = \{'b':2, 'a':[null,1]\}}\\
\end{tabbing}
\end{example}


\section{\gt{WHERE} clause}
\label{sec:where}
 
The \gl{WHERE} clause inputs the bindings that have been produced from the
\gl{FROM} clause and outputs the ones that satisfy its condition.

The boolean predicates follow SQL's 3-valued logic. Recall, PartiQL has two
kinds of absent values: \gl{NULL} and \gl{MISSING}. As far as the boolean
connectives and \gl{IS NULL} are concerned a \gl{NULL} input and a \gl{MISSING}
input behave identically. For example, \gl{MISSING AND TRUE} is equivalent to
\gl{NULL AND TRUE}: they both result into \gl{NULL}.

For the semantics of equality and of other functions, see
Section~\ref{sec:preds-and-fns}.

Alike SQL, when the expression of the \gt{WHERE} clause expression evaluates to
an absent value or a value that is not a Boolean, PartiQL eliminates the
corresponding binding. 

\begin{example} The result of
\begin{tabbing}
\gt{SELECT VALUES v.a}\\
\gt{FROM [\{'a':1, 'b':true\}, \{'a':2, 'b':null\}, \{'a':3\}] v}\\
\gt{WHERE v.b}
\end{tabbing}
\noindent is \texttt{\ob 1 \cb}.
\end{example}

The predicate \gl{IS MISSING} allows distinguishing between \gt{NULL} and
\gt{MISSING}: \gt{NULL IS MISSING} results to false; \gt{MISSING IS MISSING}
results to true.

\eat{
\subsection{Semantics of \gl{IN}}
\label{sec:in}
\yannis{TO DO.}
}
\section{Coercion of subqueries} 
\label{sec:subquery-coercion}

In PartiQL, as is the case with SQL as well, expressions may involve SFW
subqueries (\linequery{sfw subquery}). PartiQL SFW subqueries are enclosed in
parentheses (i.e., identical to SQL). For compatibility with SQL, a SFW subquery
starting with a \gt{SELECT} clause (as opposed to a subquery starting with
\gt{SELECT VALUE} or \gl{PIVOT}) coerces into a scalar or into an array,
depending on the context. The following cases replicate SQL's coercing behavior
and analyze in which cases the result of a subquery coerces into scalar and in
which cases they coerce into arrays.

An PartiQL extension with respect to SQL is that, in the permissive mode,
subqueries that fail to coerce to the required type (scalar or tuple) still run,
as opposed to failing. They simply omit from the results the data that
correspond to the coercion failures.

\subsection{Coercion of a \gl{SELECT} subquery into a scalar}
\label{sec:select-coercion-scalar}
In each of the following cases a SFW subquery coerces into a scalar
\begin{itemize}
\item if it appears as the rhs of a comparison operator (\gt{=}, \gt{>}, etc)
where the lhs is not an array literal. And, vice versa, if it appears as the lhs
of a comparison operator where the rhs is not an array literal. (If it is the
lhs of a comparison operator where the lhs is an array literal, it coerces into
array, per Section~\ref{sec:select-coercion-array}.)
\item if it is an SFW subquery expression that (a) is not the collection
expression of a \gl{FROM} clause item and (b) is not the rhs of an \gl{IN}. (If
it is the rhs of an \gl{IN} then it should not be coerced; see note on semantics
of \gl{IN}, Section~\ref{sec:in}.)
\end{itemize}

Essentially, a subquery that is coerced may appear in all clauses except the
\gl{FROM}. For example, it may be a \gt{SELECT} subquery $s$ that appears as an
item of a \gl{SELECT}, \gl{SELECT VALUE} or \gl{PIVOT} clause. Or it may
be a subexpression of an expression that appears in \gl{SELECT}, \gl{SELECT
VALUE} or \gl{PIVOT} clause. Or it may be a subexpression of the
\gl{WHERE} clause expression, as long as it is not the rhs of an \gl{IN}. In any
of these cases the result of the subquery $s$ is cast into a scalar. 

Technically, the subquery $s$ (which uses \gl{SELECT}) is rewritten into an
equivalent subquery $s'$ that utilizes \gl{SELECT VALUE}, by following the steps
of Section~\ref{sec:sql-select}. Then the result of $s'$ is cast into a scalar
by applying the function \gl{COLL\_TO\_SCALAR($s'$)}. 

\begin{example} The SQL query
\begin{tabbing}
\ \ \ \=\gt{SELECT }\=\gt{v.foo,}\\ 
\>\>\gt{(SELECT w.bar}\\
\>\>\gt{\ FROM someDataSet w}\\
\>\>\gt{\ WHERE w.sth = v.sthelse) AS bar}\\
\>\gt{FROM anotherDataSet v}
\end{tabbing}
\noindent is rewritten into 
\begin{tabbing}
\ \ \ \=\gt{SELECT VALUE \{}\=\gt{'foo': v.foo}\\
\>\>\gt{'bar': COLL\_TO\_SCALAR(}\=\gt{SELECT VALUE \{'bar': w.bar\}}\\
\>\>\>\gt{FROM someDataSet w}\\
\>\>\>\gt{WHERE w.sth = v.sthelse)\}}\\
\>\gt{FROM anotherDataSet v}
\end{tabbing}
\end{example}

As is the common semantics of PartiQL in the permissive mode, when
\gl{COLL\_TO\_SCALAR} fails to cast the subquery into a scalar, it outputs
\MISSING. The inputs that are coerced into scalars are the ones that SQL
prescribes: When the input is a collection consisting of a single tuple with a
single attribute, the input is coerced into a scalar. All other inputs to
\gl{COLL\_TO\_SCALAR} lead to \MISSING.

\begin{example} 
\label{xmpl:coercion-failure}
In this example, in one instance the inner \gt{SELECT} evaluates to a collection
with more than one element. Because the \gl{COLL\_TO\_SCALAR} function produces a
\MISSING instead of failing, the query works. 

Consider the tables
\begin{tabbing}
\ \ \ \=\gt{customers : [}\=\gt{\{'id':1, 'name':'Mary'\},}\\
\>\>\gt{\{'id':2, 'name':'Helen'\},}\\
\>\>\gt{\{'id':1, 'name':'John'\}}\\
\>\>\gt{]}\\
\>\gt{orders : [}\=\gt{\{'custId':1, 'name':'foo'\},}\\
\>\>\gt{\{'custId':2, 'name':'bar'\}}\\
\>\>\gt{]}
\end{tabbing}

The following query would fail in SQL, because there are two customer tuples
with the same id. Of course, in a well-designed SQL database that has a primary
key or uniqueness constraint on the id, there would not be two customers with
the same id. However, lack of constraints is typical in the data targeted by
PartiQL. This query runs in the permissive mode of PartiQL.

\begin{tabbing}
\ \ \ \=\gt{SELECT }\=\gt{o.name AS orderName, }\\
\>\>\gt{(SELECT c.name FROM customers c WHERE c.id=o.custId) AS customerName}\\ 
\>\gt{FROM orders o}
\end{tabbing}
The result is
\begin{tabbing}
\gt{\ob\ \{'orderName':'foo'\}, \{'orderName':'bar', 'customerName':'Helen'\}\ \cb}
\end{tabbing}
\noindent Notice the missing \gt{'customerName'} in the first tuple.
\end{example} 

As in SQL, an implementation with static type checks will be able to detect and
warn that, in certain cases, a coercion will always fail and produce
\gt{missing}. 

\begin{example}
The following \gl{SELECT} clause is guaranteed to produce tuples with \gt{bar1}
and \gt{bar2}. Thus it cannot coerce into scalar. 
\begin{tabbing}
\ \ \ \=\gt{SELECT w.bar1 AS bar1, w.bar2 AS bar2}\\
\>\gt{FROM someDataSet w}
\end{tabbing}
Static type analysis can infer that the nested query above will deliver tuples
consisting of \gt{bar1} and \gt{bar2}. Thus, even before accessing any data, it
can warn the user that this query is erroneous.
\end{example}

\subsection{Coercion of a \gt{SELECT} subquery into an array}
\label{sec:select-coercion-array}

An \gt{SELECT} SFW subquery coerces into an array when it is the rhs
(respectively, lhs) of a comparison operator whose other argument is an array.

\footnote{Recall, in the interest of compatibility to SQL, PartiQL allows array
literals to be denoted with parentheses instead of brackets (see
Section~\ref{sec:array-constructor}).}

\eat{
Technically, a function \gl{COLL\_TO\_ARRAY} coerces the singleton collection of
ordered tuples produced by the \gl{SELECT} subquery into an array that has the
values of the single tuple of the collection, in the same order that they
appeared in the tuple. If the subquery outputs anything but a singleton
collection of ordered tuples, \gl{COLL\_TO\_ARRAY} returns a \MISSING  in the
permissive mode and fails in the type-checking mode.
}

The reduction of a \gl{SELECT} subquery to the PartiQL is exhibited by the
following example.
 
\begin{example} The SQL query
\begin{tabbing}
\ \ \ \=\gt{SELECT }\=\gt{v.foo}\\ 
\>\gt{FROM anotherDataSet v}\\
\>\gt{WHERE (v.a, v.b) = (}\=\gt{SELECT w.c, w,d}\\
\>\>\gt{FROM someDataSet w}\\
\>\>\gt{WHERE w.sth = v.sthelse)}
\end{tabbing}
\noindent is rewritten into 
\begin{tabbing}
\ \ \ \=\gt{SELECT VALUE \{'foo': v.foo\}}\\
\>\gt{FROM anotherDataSet v}\\
\>\gt{WHERE (v.a, v.b) = (}\=\gt{SELECT VALUE [w.c, w,d]}\\
\>\>\gt{FROM someDataSet w}\\
\>\>\gt{WHERE w.sth = v.sthelse)}
\end{tabbing}
\end{example}

\eat{
\[ \gt{SELECT}\ r \mapsto \gt{collToCollArrays(SELECT\ VALUES}\ r\gt{)} \]

The function \gt{collToCollArrays} coerces the collection produced by \gt{SELECT
VALUES} into a collection of arrays. In particular, if an element is an ordered
tuple, the tuple is coerced into an array by losing the attribute names. If an
element is anything but an ordered tuple or an array, it is dismissed by the
coercion, i.e., it does not appear in the result of the coercion.

 Scalar lhs: $l$\ \gt{IN (SELECT}\ $r$\gt{)}, when $l$ is an expression. 

When the lhs is an expression, the nested \gt{SELECT} is transformed into a
\gt{SELECT VALUE} that produces a collection of scalars, as it would be in SQL.
Technically

\[\gt{SELECT}\ s\ (\gt{AS}\ a) \]

is transformed into 

\[ \gt{SELECT VALUE }s \]

\reminder{Open Issue: Transformation in the presence of *, though it is almost
silly in this case.}

When the lhs is an array, the nested \gt{SELECT} in the rhs is transformed into
a \gt{SELECT VALUE} that produces a collection of arrays. In the case that the
\gt{SELECT} does not involve \gt{*} the transformation is based on treating the
\gt{SELECT} clause items as the elements of the array, potentially ignoring any
\gt{AS} clause. That is

 Array lhs: \gt{(}$l_1, l_2, \ldots$\gt{) IN (SELECT}\ $r$\gt{)}, where the lhs has two or more expressions $l_1, l_2, \ldots$. As in SQL, when there are two or more expressions the notation \gt{(}$l_1, l_2, \ldots$\gt{)} stands for an array whose $i$th element is the expression $l_i$. In contrast, \gt{)}$l$\gt{)} stands for a (scalar) expression.
}

\section{Scoping rules}
\label{sec:variable-scoping}

As far as the variables environment is concerned, the scoping rules are
identical to those of SQL. Section~\ref{sec:scoping-variables} explained how the
resolution of variable naming conflicts favors the variables defined by the
inner queries.

The scoping rules discussed in the present section discuss the resolution of
naming conflicts between names defined in the database environment and the
variables of the environment variables. The potential for such naming conflicts
is driven by the nested data of PartiQL, as illustrated next.

Notice there are a few more naming conventions, pertaining to the use of
attribute names defined in the \gl{SELECT} clause into the \gl{GROUP BY} and
\gl{ORDER BY} clause. These conventions are explained in along with the
semantics of the respective clauses (see Sections~\ref{section:group-by}
and~\ref{section:order-by}).

\almann{TODO: Clean up this terminology with respect to
Section~\ref{sec:environments-and-bindings}}

\begin{example}
The following example illustrates how SQL compatibility issues and the needs of
navigating into nested data need to be carefully merged together. Consider the
following database that has a table \gt{c}, i.e. a collection of tuples, and
also named data \gt{x.n} and \gt{y}.

\begin{tabbing}
\ \ \ \=\gt{t.c: \ob\ }\=\gt{\{'a':1, 'n':[\{'b':11, 'c':12\}]\},}\\
\>\>\gt{\{'a':2, 'n':[\{'b':21, 'c':22\}]\}}\\
\>\>\gt{\cb}\\
\>\gt{x.n : \ob\ \{'b':3\}\ \cb}\\
\>\gt{y: \{'a':1, 'b':2\}}
\end{tabbing}

Then consider the query
\begin{tabbing}
\ \ \ \=\gl{SELECT t.a}\\
\>\gl{FROM t.c AS x}\\
\>\gl{WHERE x.a IN (SELECT y.b FROM x.n AS y)}
\end{tabbing}

This query poses many scoping issues:
\begin{enumerate}
\item Does \gt{x.n} refer to the named value \gt{x.n} or to the \gt{n} attribute
of the variable \gt{x}? For SQL compatibility purposes it refers to the named
value \gt{x.n}. Read below how to refer to the variable \gt{x}.
\item Does \gt{y.b} refer to the \gt{b} attribute of the \gt{y} attribute or to
the \gt{b} attribute of the named value \gt{y}? For SQL compatibility purposes
it refers to the \gt{b} attribute of the variable \gt{y}.
\end{enumerate}
Notice how SQL compatibility required the database environment to take priority
over the variables environment in the \gl{FROM} clause and then, vice versa, the
variables environment to take priority over the database environment in the
\gl{SELECT} clause.
\end{example}

\highlight{Scoping rules resolving naming conflicts between variables and database names}
Since the rules are easier to express when all database names are a single
identifier, such as \gt{thedb} or \gt{"the db"} (as opposed to paths, such as
\gt{somedb.sometable}), we first specify the scoping rules under the assumption
that all database names are a single identifier. We remove the assumption and
generalize later.

In the absence of schema the following rules apply
\begin{enumerate}
\item \gl{@}\gn{identifier} refers to the environment variable named
\gn{identifier}; if there is no such environment variable, the \gn{identifier}
refers to the database name \gn{identifier}; if there is no such database name
either, the query fails compilation.
\item in a \gl{FROM} clause path that starts with \gn{identifier}, the
\gn{identifier} refers to the database name \gn{identifier}; if there is no such
database name, the \gn{identifier} refers to a variable; otherwise query fails
compilation. \footnote{A path is a \from\ clause path if it appears in the \from
\ clause of the SFW query in which it is \textit{immediately} nested.}
\item in a non-\from\ clause path that starts with \gn{identifier}, the
\gn{identifier} refers to the environment variable named \gn{identifier}; if
there is no such environment variable, the \gn{identifier} refers to the
database name \gn{identifier}; if there is no such database name either, the
query fails compilation.
\end{enumerate}

Next, we
generalize to also allow for the possibility of database names of the form
$\gn{identifier}\gl{.}\gn{indentifier}\gl{.}\ldots$. The following rules apply
regarding the semantics of $i_1 \gl{.} i_2 \gl{.} \ldots \gl{.} i_n$, where
$i_1, i_2, \ldots i_n$ are identifiers.
\begin{itemize}
\item \gl{@}$i_1 \gl{.} i_2 \gl{.} \ldots \gl{.} i_n$ always refers to the
environment variable named $i_1$; if there is no such variable and $i_1 \gl{.}
i_2 \gl{.} \ldots \gl{.} i_m, m\leq n$ is a database name then $i_1 \gl{.} i_2
\gl{.} \ldots \gl{.} i_m$ refers to such named database name. Again, if there is
a choice, choose the largest $m$. If both the resolution to variable and the
resolution to database name, fail the query during compilation.
%
\item if  $i_1 \gl{.} i_2 \gl{.} \ldots \gl{.} i_n$ is a \from\ path and $i_1
\gl{.} i_2 \gl{.} \ldots \gl{.} i_m, m\leq n$ is a database name then $i_1
\gl{.} i_2 \gl{.} \ldots \gl{.} i_m$ refers to such named database name and
$i_{m+1} \gl{.} \ldots \gl{.} i_n$ is a series of tuple path navigations
starting from the database name $i_1 \gl{.} i_2 \gl{.} \ldots \gl{.} i_m$. If
there is a choice, choose the largest $m$, i.e., the longest database name.
%
\item if $i_1 \gl{.} i_2 \gl{.} \ldots \gl{.} i_n$ is a non-\from\ clause
expression and $i_1$ is an environment variable then $i_1$ refers to such
variable; if there is no such variable and $i_1 \gl{.} i_2 \gl{.} \ldots \gl{.}
i_m, m\leq n$ is a database name then $i_1 \gl{.} i_2 \gl{.} \ldots \gl{.} i_m$
refers to such named database name. Again, if there is a choice, choose the
largest $m$. If both the resolution to variable and the resolution to database
name, fail the query during compilation.
\end{itemize}

\begin{example}
Assume database names \gt{coll}, \gt{v.foo}, \gt {w}. Then in the query

\begin{verbatim}
1 SELECT v.foo
2 FROM coll AS v, @v.foo AS w, 
3          (SELECT w.a, u.b FROM @w.bar AS u)
4          AS x
\end{verbatim}

\noindent  \gt{coll} refers to the database name. The \gt{v} in \gt{@v.foo} refers to the variable \gt{v}. If the \gt{@} were not there, \gt{v.foo} would refer to the database name \gt{v.foo}. The \gt{w} in \gt{w.a} refers to the variable defined in line~2. 

Note, the expressions \gt{coll} and \gt{@v.foo} are \from\ clause expressions because they appear in the \from\ clause of the \gn{sfw\_query} of lines~1-4, in which they are immediately nested. Similarly, the expression \gt{@w.bar} is a \from\ clause expression because it appears in the \from\ clause of the \gn{sfw\_query} of line~3, in which it is immediately nested. In contrast, the expressions \gt{w.a} and \gt{u.b} are not \from\ clause expressions. Though they are nested into the \from\ clause of the query of lines~1-4, they are not immediately nested into the query of lines~1-4. 
\end{example}

\section{\gl{GROUP BY} clause}
\label{section:group-by} 
The PartiQL \gl{GROUP BY} clause expands SQL's grouping. Unlike SQL, the PartiQL
\gl{GROUP BY} can be thought of as a standalone operator that inputs a
collection of binding tuples and outputs a collection of binding tuples. 

As is typical in many clauses, the semantics proceed in two steps:
\begin{itemize}
\item Section~\ref{sec:group-variable} explains the core PartiQL \gl{GROUP BY}
structure.
\item Section~\ref{sec:sql-groupby} shows that SQL's \gl{GROUP BY} can be
explained over the core \gl{GROUP BY}.
\end{itemize}

\subsection{PartiQL \gl{GROUP BY} core: Grouping into a Group Variable}
The \gl{GROUP BY} clause \linequery{group by} 
\label{sec:group-variable}

%
\[ \gl{GROUP BY}\ e_1\ \gl{AS}\ x_1\ \gl{,} \ldots \gl{,}\ e_m\ \gl{AS}\ x_m\ \gl{GROUP AS}\ g\] 
%
\noindent creates a group. Each $e_i$ is a \textit{grouping expression}, each
$x_i$ is a \textit{grouping variable}
\footnote{Grouping variables is an extension of SQL by PartiQL, which
interestingly simplifies dramatically the explanation of SQL semantics, as it
enables the \gl{GROUP BY} to be seen as a standalone function.}
and $g$ is the \textit{group variable}. 

\almann{TODO define the semantics when the group expression is a constant
positive integer--specifically the relation to the \gl{SELECT} projection
items.
Yannis: Interestingly, this is a deprecated feature of the SQL standard as of 1998.
Unfortunately, Postgres carried the torch but I think that all we have to do 
is to say ``avoid using positive constants" as many implementations 
use them to mean the first column of SELECT. 
Recall, this was Don's March feedback, that
we cannot hold Couchbase for SQL incompatibility on what is not SQL :) }

As in SQL, the bag of input binding tuples $B^{in}_{\gl{GROUP}}$ is partitioned
into the minimal number of equivalence groups $B_1 \ldots B_n$, such that any
two binding tuples $b, b' \in B^{in}_{\gl{GROUP}}$ are in the same equivalence
group if and only if every grouping expression $e_i$ evaluates to equivalent
values $v_i$ (when evaluated on $b$) and $v'_i$ (when evaluated on $b'$). More
precisely, as in SQL, there is an equivalence function \gl{eqg}, used by the
\gl{GROUP BY} to determine if two values $v_i$ and $v_i'$ are equivalent for
grouping purposes. The equivalence function $\gl{eqg(}v_i, v'_i\gl{)}$ returns
only true or false; true meaning that the values are equivalent for grouping
purposes. See Section~\ref{sec:eqg} for specifics of \gl{eqg}. If a grouping
expression evaluates to \MISSING, it is first coerced into \NULL,
thus bringing \MISSING and \NULL in the same group.

Unlike SQL, for each group $B_j ~ (1 \leq j \leq n)$, the \gl{GROUP BY} clause
outputs a binding tuple $b_j = \langle x_1 : v_1, ~ \ldots, ~ x_m : v_m, ~ g :
B_j \rangle$ that has the full group $B_j$. Notice:
\begin{enumerate}
\item the binding tuples that appear in the $g$ collection have one attribute
for each of the variables defined in the \gl{FROM} clause,
since these binding tuples come as-is from $B^{in}_{\gl{GROUP}}$.
\item even if the bag $B^{in}_{\gl{GROUP}}$ is flat binding tuples, the output
bag $B^{out}_{\gl{GROUP}}$ is not just flat binding tuples, since $g$ has nested
binding tuples. Note, we have been explicitly denoting binding attributes
with \MISSING values in the binding tuples. However, once these binding tuples
become the tuples of the PartiQL data model, any binding attribute with 
\MISSING value will not appear.
\end{enumerate}

\begin{example}
\label{sec:grouping-readings}
Consider again the \gt{logs} data of Example~\ref{xmpl:nesting-readings} and
assume that we want to group the \gt{co} readings by sensor. The following query
solves the problem using only core features. 

\begin{tabbing}
\ \ \ \=\gl{SELECT VALUE \{}\=\gl{'sensor': sensor,}\\
\>\>\gl{'readings': (SELECT VALUE v.l.co FROM g AS v) \}}\\
\>\gl{FROM logs AS l}\\
\>\gl{GROUP BY l.sensor AS sensor GROUP AS g}
\end{tabbing}
The \gl{GROUP BY} outputs the collection of binding tuples
\begin{tabbing}
\ \ \ \=$B^{out}_{\gl{GROUP}} = B^{in}_{\gl{SELECT}} =$\=\gt{\ob }\=\gt{$\langle$ sensor:1, g: \ob}\=\gt{$\langle$l:\{'sensor':1, 'co':0.4\}$\rangle$,}\\ 
\>\>\>\>\gt{$\langle$l:\{'sensor':1, 'co':0.2\}$\rangle$}\\
\>\>\>\>\gt{\cb $\rangle$,}\\
\>\>\>\gt{$\langle$ sensor:2, g:\ob $\langle$l:\{'sensor':2, 'co':0.3\}$\rangle$ \cb $\rangle$}\\
\>\>\gt{\cb}
\end{tabbing}

Notice that the collection \gt{g} has tuples with a single attribute \gt{l},
since this is the single variable of the \gl{FROM} clause in this example.

Consequently the \gl{SELECT} clause outputs
\begin{tabbing}
\ \ \ \=\gt{\ob }\=\gt{\{'sensor':1, 'readings':\ob 0.4, 0.2\cb\},}\\
\>\>\gt{\{'sensor':2, 'readings':\ob 0.3 \cb\}}\\
\>\gt{\cb}
\end{tabbing}
Notice that the query of Example~\ref{xmpl:nesting-readings} and the query of
the present example do not always produce the same result. For example, if there
were no readings for a sensor, the query of Example~\ref{xmpl:nesting-readings}
would still have this sensor in the result (and its \gt{readings} would be
empty). In contrast, the query of the present example will not have this sensor
in the result.

Here is a shorter equivalent query that uses PartiQL collection paths and SQL's
aliases.
\begin{tabbing}
\ \ \ \=\gl{SELECT VALUE \{}\=\gl{'sensor': sensor,}\\
\>\>\gl{'readings': g[*].l.co \}}\\
\>\gl{FROM logs AS l}\\
\>\gl{GROUP BY l.sensor AS sensor GROUP AS g}
\end{tabbing}
\end{example}

Notice, the output binding tuple provides the partitioned input binding tuples
in the group variable $g$, which can be explicitly utilized in subsequent
\gl{HAVING}, \gl{ORDER BY} and \gl{SELECT} clauses. Thus, an PartiQL query can
perform complex computations on the groups, leading to results of any type (e.g.
collections nested within collections). The explicit presence of groups in
PartiQL, while more general than SQL, also leads to simpler semantics than those
of SQL, since the \gl{GROUP BY} clause semantics are independent of the presence
of subsequent functions in \gl{HAVING}, \gl{ORDER BY} and \gl{SELECT}.

\begin{example}
\label{xmpl:groupby-avg-count}

The following PartiQL query counts and averages the readings of each sensor. It
also refers to the \gt{logs} of Example~\ref{xmpl:nesting-readings}. The
\gl{COLL\_COUNT} function is simply given the group variable and counts how many
elements are in that collection.
\begin{tabbing}
\ \ \ \=\gl{SELECT VALUE \{}\=\gl{'sensor': sensor,}\\
\>\>\gl{'avg': COLL\_AVG(SELECT VALUE v.l.co FROM g AS v), }\\
\>\>\gl{'count': COLL\_COUNT(g) \}}\\
\>\gl{FROM logs AS l}\\
\>\gl{GROUP BY l.sensor AS sensor GROUP AS g}
\end{tabbing}
\end{example}

Notice, the aggregate functions \gl{COLL\_AVG} and \gl{COLL\_COUNT} (and for
that matter, by convention, any function starting with \gl{COLL}) can be thought
of as general-purpose functions. Generally, they do not have to be fed by the
result of a grouping operation - unlike SQL's \gl{COUNT} and \gl{AVG} that are
being fed exclusively from the results of grouping operations. (Furthermore, the
SQL \gl{COUNT} and \gl{AVG} make use of SQL's syntactic sugar, where there is no
explicit use of group variable, as explained in
Section~\ref{sec:implicit-group-variable}.)

\begin{example} 
This is a legitimate PartiQL expression:
\begin{tabbing} 
\ \ \ \gl{COLL\_COUNT(}\gt{[5, \{a:2, b:3\}]}\gl{)}
\end{tabbing}
The result is \gt{2}, since the input to \gl{COLL\_COUNT} is an array with two
elements. 
\end{example}
Similarly, it is fine to include in any clause an aggregate function fed by the
result of a (sub)query.
\begin{example}
In the following expression \gl{COLL\_COUNT} inputs the result of a query
\begin{tabbing}
\ \ \ \gl{COLL\_COUNT(SELECT VALUE x FROM logs x WHERE x.sensor=1)}
\end{tabbing}
\end{example}

\highlight{Remark} An efficient implementation will often avoid materializing
the group variable. In many cases, like the ones of the above examples, the
group can be streamed into the aggregate function.

\highlight{Remark} The semantics of the \gl{SELECT} and \gl{HAVING} clauses do
not need to be aware of the presence of \gl{GROUP BY} and treat differently (as
SQL would do) these classes of functions: 
\begin{itemize}
\item scalar functions (e.g. \gl{+}) that input scalars and output scalars 
\item SQL aggregation functions (e.g. \gl{SUM}) that input bags and output
scalars
\end{itemize}
Indeed, \gl{HAVING} behaves identical to a \gl{WHERE}, once groups
are already formulated earlier.

The PartiQL approach provides two benefits: First, it leads to shorter, modular
semantics. Second, it enables \gl{GROUPY} to address use cases that would
otherwise need knowledge and non-trivial SQL programming of window functions.
See Example~\ref{xmpl:windows-by-grouping}.

\subsubsection{Equivalence function used by grouping; grouping of \NULL and \MISSING}
\label{sec:eqg}

The equivalence function \gl{eqg} extends SQL's respective function. In
particular, it behaves as follows:
\begin{itemize}
\item $\gl{eqg}(\NULL, \NULL)$ is true, despite $\NULL=\NULL$ not being true.
\item for any two non-null values $x$ and $y$, $\gl{eqg}(x,y)$ returns the same
with $x=y$. As is the case generally for $=$, while SQL's $=$ will error when
given incompatible types, while the PartiQL $=$ will return \gl{false}.
\end{itemize}

Notice that PartiQL will group together the \NULL and the \MISSING grouping
expressions, since any grouping expression resulting to \MISSING has been
coerced into \NULL before \gl{eqg} does comparisons for grouping.
Example~\ref{xmpl:grouping-null-missing} shows the repercussions of coercing
\NULL into \MISSING and also shows how to discriminate between \NULL and
\MISSING, if so desired.

\begin{example}
\label{xmpl:grouping-null-missing}
The query of Example~\ref{sec:grouping-readings} will group together any log
readings where the \gl{sensor} attribute is either \NULL or is altogether
\MISSING. For example, if \gl{logs} is
\begin{tabbing}
\gl{logs:}\=\gl{[ }\=\gl{\{'sensor': 1, 'co':0.4\},}\\
\>\>\gl{\{'sensor': 2, 'co':0.3\},}\\
\>\>\gl{\{'sensor': null, 'co':0.1\},}\\
\>\>\gl{\{'sensor': 1, 'co':0.2\},}\\
\>\>\gl{\{'co':0.5\}}\\
\>\gl{]}
\end{tabbing}
\noindent then the \gl{GROUP BY} will output the collection of binding tuples
\begin{tabbing}
\ \ \ \=$B^{out}_{\gl{GROUP}} = B^{in}_{\gl{SELECT}} =$\=\gt{\ob }\=\gt{$\langle$ sensor:1, g: \ob}\=\gt{$\langle$l:\{'sensor':1, 'co':0.4\}$\rangle$,}\\ 
\>\>\>\>\gt{$\langle$l:\{'sensor':1, 'co':0.2\}$\rangle$}\\
\>\>\>\>\gt{\cb $\rangle$,}\\
\>\>\>\gt{$\langle$ sensor:2, g:\ob $\langle$l:\{'sensor':2, 'co':0.3\}$\rangle$ \cb $\rangle$,}\\
\>\>\>\gt{$\langle$ sensor:null, g: \ob}\=\gt{$\langle$l:\{'sensor':null, 'co':0.1\}$\rangle$,}\\
\>\>\>\>\gt{$\langle$l:\{'co':0.5\}$\rangle$}\\
\>\>\>\>\gt{\cb $\rangle$} \\
\>\>\gt{\cb}\\
\end{tabbing}
\noindent Notice that both the 3rd and 5th tuples of \gl{logs} were grouped
under the \gl{sensor:null} group, despite the \gl{sensor} of the 3rd being \NULL
while the \gl{sensor} of the 5th being \MISSING. The query result is 
\begin{tabbing}
\ \ \ \=\gt{\ob }\=\gt{\{'sensor':1, 'readings':\ob 0.4, 0.2\cb\},}\\
\>\>\gt{\{'sensor':2, 'readings':\ob 0.3\cb\},}\\
\>\>\gt{\{'sensor':null, 'readings':\ob 0.1, 0.5\cb \}}\\
\>\gt{\cb}
\end{tabbing}

If we wanted to discriminate the \NULL from the \MISSING we could write the
following query
\begin{tabbing}
\ \ \ \=\gl{SELECT VALUE \{}\=\gl{'sensor': CASE WHEN missingFlag THEN MISSING ELSE sensor END,}\\
\>\>\gl{'readings': (SELECT VALUE v.l.co FROM g AS v) \}}\\
\>\gl{FROM logs AS l}\\
\>\gl{GROUP BY l.sensor IS MISSING AS missingFlag, l.sensor AS sensor GROUP AS g}
\end{tabbing}
\noindent In this case the \gl{GROUP BY} would output the collection of binding tuples
\begin{tabbing}
\ \ \ \=$B^{out}_{\gl{GROUP}} = B^{in}_{\gl{SELECT}} =$\=\gt{\ob }\=\gt{$\langle$}\=\gt{missingFlag:false,}\\
\>\>\>\>\gt{sensor:1, g: \ob}\=\gt{$\langle$l:\{'sensor':1, 'co':0.4\}$\rangle$,}\\ 
\>\>\>\>\>\gt{$\langle$l:\{'sensor':1, 'co':0.2\}$\rangle$}\\
\>\>\>\>\gt{\cb $\rangle$,}\\
\>\>\>\gt{$\langle$ missingFlag: false, sensor:2, g:\ob $\langle$l:\{'sensor':2, 'co':0.3\}$\rangle$ \cb $\rangle$}\\
\>\>\>\gt{$\langle$ missingFlag: false, sensor:null, g:\ob $\langle$l:\{'sensor':null, 'co':0.1\}$\rangle$ \cb $\rangle$}\\
\>\>\>\gt{$\langle$ missingFlag: true, sensor:null, g:\ob $\langle$l:\{'co':0.5\}$\rangle$ \cb $\rangle$}\\
\>\>\gt{\cb}\\
\end{tabbing}
\noindent and the query result would be
\begin{tabbing}
\ \ \ \=\gt{\ob }\=\gt{\{'sensor':1, 'readings':\ob 0.4, 0.2\cb\},}\\
\>\>\gt{\{'sensor':2, 'readings':\ob 0.3\cb\},}\\
\>\>\gt{\{'sensor':null, 'readings':\ob 0.1\cb\},}\\
\>\>\gt{\{'readings':\ob 0.5\cb\}}\\
\>\gt{\cb}
\end{tabbing}
\end{example}

\subsubsection{The \gl{GROUP ALL} variant}
\label{sec:group-all}

The \gl{GROUP ALL} variant of \gl{GROUP BY} outputs a single binding tuple,
regardless of whether the \gl{FROM}/\gl{WHERE} produced any tuples, i.e.,
regardless of whether its input $B^{in}_{\gl{GROUP}}$ is empty or not.

The \gl{GROUP ALL} is not increasing the expressiveness of PartiQL.
Example~\ref{xmpl:group-all-core} shows how to achieve without \gl{GROUP ALL},
what the \gl{GROUP ALL} can do. However, we include \gl{GROUP ALL} for
facilitating the reduction of SQL's aggregation into the core PartiQL (see
Section~\ref{sec:implicit-group-variable}). 
%
\begin{example}
\label{xmpl:group-all-core}
Consider again the \gt{logs} data of Example~\ref{xmpl:nesting-readings} and
assume that we want to count the total number of readings that are above
\gt{1.5} with a core PartiQL query. (Example~\ref{xmpl:group-by-nothing-sql}
does the same with SQL.) 
\begin{tabbing}
\ \ \ \=\gl{SELECT VALUE \{'largeco': COLL\_COUNT(g)\}}\\
\>\gl{FROM logs AS l}\\
\>\gl{WHERE l.co > 1.5}\\
\>\gl{GROUP ALL AS g} 
\end{tabbing}
Notice, there are no readings above \gt{1.5} in the example data. Since there is
no tuple that satisfies the \gl{WHERE} clause
\[\begin{array}{l}
B^{out}_{\gl{WHERE}} = B^{in}_{\gl{GROUP}} = \ob\ \cb \\
B^{out}_{\gl{GROUP}} = B^{in}_{\gl{SELECT}} = \ob \langle \gt{g}: \ob\ \cb \rangle \cb
\end{array}
\]
Since $\gl{COLL\_COUNT(}\ob\ \cb\gl{)}$ is \gt{0}, the query result is the collection 
\[\ob \gt{\{'largeco': 0 \}} \cb \]
Therefore the PartiQL query is equivalent to the plain SQL query
\begin{tabbing}
\ \ \ \=\gl{SELECT COUNT(*) AS largeco}\\
\>\gl{FROM logs AS l}\\
\>\gl{WHERE l.co > 1.5}\\
\end{tabbing}
The following core PartiQL also accomplishes the same computation, without using
\gl{GROUP ALL}.
\begin{tabbing}
\ \ \ \=\gl{\{'largeco': COLL\_COUNT(}\=\gl{SELECT VALUE l}\\
\>\>\gl{FROM logs AS l}\\
\>\>\gl{WHERE l.co > 1.5}\\
\>\>\gl{)}\\
\>\gl{\}}
\end{tabbing}
\end{example}

 
\subsection{SQL compatibility features} 
\label{sec:sql-groupby}
The group-by and aggregation of PartiQL is backwards compatible to SQL.

\subsubsection{Grouping Attributes and Direct Use of Grouping Expressions}
\label{sec:grouping-attributes}
For SQL compatibility PartiQL allows

\[\gl{GROUP BY}\ \ldots\gl{,} e\gl{,} \ldots\]

\noindent i.e., a grouping expression $e$ that is not associated with a grouping
variable $x$. (In core PartiQL, one would write $e\ \gl{AS}\ x$.)

For SQL compatibility, PartiQL supports using the grouping expression $e$ in
\gl{HAVING}, \gl{ORDER BY} and \gl{SELECT} clauses. The SQL form (S) is
syntactic sugar for the core PartiQL (C).

\begin{tabular}{@{}l@{~}l@{~}l@{~}l@{}}
(S) & \texttt{FROM ~~~} \ldots                          & (C)   & \texttt{FROM ~~~} \ldots \\
    & \texttt{GROUP BY} $e \texttt{,} \ldots$         &       & \texttt{GROUP BY} $e \texttt{ AS } x \texttt{,} \ldots$ \\
    & \texttt{HAVING ~} $f(e, \ldots)$                &       & \texttt{HAVING ~} $f(x, \ldots)$ \\
    & \texttt{ORDER BY} $f'(e) \texttt{,} \ldots$     &       & \texttt{ORDER BY} $f'(x) \texttt{,} \ldots$  \\
    & \texttt{SELECT ~} $f''(e) \texttt{,} \ldots$    &       & \texttt{SELECT ~} $f''(x) \texttt{,} \ldots$ \\
\end{tabular}

\begin{example}
The SQL-compatible query 
\begin{tabbing}
\ \ \ \=\gl{SELECT v.a+1 AS bar}\\
\>\gl{FROM foo AS v}\\
\>\gl{GROUP BY v.a+1}
\end{tabbing}
\noindent is written in core PartiQL as
\begin{tabbing}
\ \ \ \=\gl{SELECT VALUE \{'bar': x\}}\\
\>\gl{FROM foo AS v}\\
\>\gl{GROUP BY v.a+1 AS x GROUP AS dontcare}
\end{tabbing}
\end{example}

\highlight{Remark: What is ``same expression"?} An open question in the
equivalence of (C) and (S) is the exact meaning of ``same expression $e$ in
\gl{GROUP BY} and \gl{SELECT} (or \gl{HAVING}, \gl{ORDER BY})". Is \gt{v.a + 1}
the same with \gt{1 + v.a}? Is \gt{v.a + 1} the same with \gt{a + 1} in the
presence of a schema that dictates that the variable \gt{v} is a tuple with an
attribute \gt{a}? Both SQL and PartiQL answer ``no" and ``yes" respectively to
the two questions. In particular:

An expression $e$ that appears in the \gl{GROUP BY} clause and an expression
$e'$ that appears in the \gl{SELECT} or \gl{HAVING} or \gl{ORDER BY} are
considered the same expression if they are syntactically identical after
performing the schema-based rewritings of Section~\ref{sec:schema}.

\subsubsection{SQL's Implicit Use of the Group Variable in SQL Aggregate Functions} 
\label{sec:implicit-group-variable}

SQL does not have explicit group variables. For SQL compatibility, PartiQL
allows the SQL aggregation functions to be fed by expressions that do not
explicitly say that there is iteration over the group variable. Suppose that a
query

\begin{enumerate}
\item is a \gl{SELECT} query,
\item lacks a \gl{GROUP AS} clause, and
\item any of the \gl{SELECT}, \gl{HAVING} and/or \gl{ORDER BY} clauses contains
a function call $f(e)$, where $f$ is a \textit{SQL aggregation function} such as
\gl{SUM} and \gl{AVG}. (See Section~\ref{sec:SQL-aggregation-functions} 
\end{enumerate}

Then, the query is rewritten as follows:
\begin{itemize}
\item if the query has a \gl{GROUP BY} clause, add to it 
\begin{tabbing}
\ \ \ \gl{GROUP AS }$g$
\end{tabbing}

\noindent where $g$ is a fresh variable, i.e., a variable that is not a database
name nor a variable of the query or a variable of the queries within which it is
nested.

\item if the query has no \gl{GROUP BY} clause, add to it 
\begin{tabbing}
\ \ \ \gl{GROUP ALL GROUP AS }$g$
\end{tabbing}
\noindent where $g$ is a fresh variable.

\item if the aggregation function call is \gl{COUNT(*)}, then rewrite into
$\gl{COUNT}(g)$

\item otherwise, rewrite $f(e)$ into 

\[ f(\gl{SELECT VALUE}\ e'\ \gl{FROM}\ g\ \gl{AS}\ p) \]

\noindent where $e'$ is produced from $e$ as follows: Consider the variables
$v_1, \ldots, v_n$  that appear in $B^{in}_{\gl{GROUP}}$ (i.e., the variables
defined by the query's \gl{FROM} and \gl{LET} clauses) and are not grouping
attributes. Substitute each identifier $v_i$ (that does not stand for attribute
name) in $e$ with $p.v_i$.
\end{itemize}

\yannis{to self: Do we first dereference attribute names to variable.attribute
names and then rewrite the GROUP BY? Or vice versa? Or all together?}

\yannis{to self: Add ``database name" in query syntax under path}

\begin{example}
\label{xmpl:groupby-sql}

Consider again the query of Example~\ref{xmpl:groupby-avg-count}. It can be
written in an SQL compatible way as

\begin{tabbing}
\ \ \ \=\gl{SELECT }\=\gl{l.sensor AS sensor,}\\
\>\>\gl{AVG(l.co) AS avg, }\\
\>\>\gl{COUNT(*) AS count}\\
\>\gl{FROM logs AS l}\\
\>\gl{GROUP BY l.sensor}
\end{tabbing}
\end{example}

\begin{example}
\label{xmpl:group-by-nothing-sql}
The query of Example~\ref{xmpl:group-all-core} can be written in standard SQL
syntax as
\begin{tabbing}
\ \ \ \=\gl{SELECT  COUNT(g) AS largeco}\\
\>\gl{FROM logs AS l}\\
\>\gl{WHERE l.co > 1.5}
\end{tabbing}
\end{example}

Notice that SQL does not allow nested aggregate functions. Respectively, PartiQL
does not allow one to write queries that lack a \gl{GROUP AS} or \gl{GROUP ALL}
clause and have nested aggregate SQL functions.

\subsubsection{Designation of SQL aggregate functions}
\label{sec:SQL-aggregation-functions}

Each implementation will have a list of SQL aggregate functions, which are not
necessarily just the ones prescribed by the standard (\gl{COUNT}, \gl{SUM},
\gl{AVG}, etc). (Recall from Section~\ref{sec:implicit-group-variable} that SQL
aggregate functions do not use an explicit group variable.)

Furthermore, it is required that for each SQL aggregate function $f$, if an
implementation offers a corresponding core PartiQL aggregate function, the
PartiQL function is named \gl{COLL\_$f$}. For example, the core PartiQL
aggregate \gl{COLL\_AVG} corresponds to the SQL aggregate \gl{AVG}.
Nevertheless, it is possible that an implementation offers only \gl{COLL\_AVG}
or offers only \gl{AVG}. The semantic relationship between the SQL aggregate
function and the corresponding core PartiQL aggregate function is the one
explained in Section~\ref{sec:implicit-group-variable}: The SQL aggregate
functions do not input explicit group variables and, thus, their semantics are
explained by the reduction to the corresponding core PartiQL aggregate.

\subsubsection{Aliases from \gl{SELECT} clause}
\label{sec:select-aliases-groupby}

In SQL, a grouping expression may be an alias that is defined by the \gl{SELECT}
clause. For compatibility purposes, PartiQL adopts the same behavior. 

The query (S), which uses the \gl{SELECT}-defined alias feature, is syntactic
sugar for the query (C). Notice that the grouping expression $a$ is simply a
shorthand for $e$.

\begin{tabular}{@{}l@{~}l@{~}l@{~}l@{}}
(S) & \gl{SELECT ~} $\ldots\gl{,} e\ \gl{AS}\ a\gl{,} \ldots$    & \ \ \ (C)   & \gl{SELECT ~} $\ldots\gl{,} e\ \gl{AS}\ a \gl{,} \ldots$ \\
    & \gl{FROM ~~~} \ldots                          &    & \gl{FROM ~~~} \ldots \\
    & \gl{GROUP BY} $\ldots\gl{,} a\gl{,} \ldots$         &       & \gl{GROUP BY} $\ldots\gl{,} e\gl{,} \ldots$ \\
\end{tabular}

In the case that the grouping expression is a constant positive integer literal
$n$, then it stands for the $n$th attribute of the \gl{SELECT} clause. However,
this requires that the tuples produced by the \gl{SELECT} have schema and they
are ordered tuples. The relevant examples will be provided in the schema
section.

\begin{example}
\label{xmpl:select-aliases-groupby}
Consider the database 
\begin{tabbing}
\ \ \ \gt{people: \ob}\=\gt{\{'name': 'zoe',  'age': 10, 'tag': 'child'\},}\\
\>\gt{\{'name': 'zoe', 'age': 20, 'tag': 'adult'\},}\\
\>\gt{\{'name': 'bill', 'age': 30, 'tag': 'adult'\}}\\
\>\gt{\cb}
\end{tabbing}
The query
\begin{tabbing}
\ \ \ \=\gl{SELECT p.tag || ':' || p.name AS tagname, AVG(p.age) AS average}\\
\>\gl{FROM people AS p}\\
\>\gl{GROUP BY tagname}
\end{tabbing}
\noindent is equivalent to the query 
\begin{tabbing}
\ \ \ \=\gl{SELECT p.tag || ':' || p.name AS tagname, AVG(p.age) AS average}\\
\>\gl{FROM people AS p}\\
\>\gl{GROUP BY p.tag || ':' || p.name}
\end{tabbing}
Either query results into 
\begin{tabbing}
\ \ \ \gt{people: \ob}\=\gt{\{'tagname': 'child:zoe',  'average': 10\},}\\
\>\gt{\{'tagname': 'adult:zoe', 'average': 20\},}\\
\>\gt{\{'tagname': 'adult:bill', 'average': 30\}}\\
\>\gt{\cb}
\end{tabbing}
\end{example}
 
\subsection{Windowing cases simplified by the PartiQL grouping}
\label{sec:windows-by-grouping}

\begin{example}
\label{xmpl:windows-by-grouping}
Consider again a collection of sensor readings, this time with a timestamp. 

\begin{tabbing}
\ \ \ \=\gt{logs: [}\=\gt{\{'sensor':1, 'co':0.4, 'timestamp':04:05:06\},}\\
\>\>\gt{\{'sensor':1, 'co':0.2, 'timestamp':04:05:07\},}\\
\>\>\gt{\{'sensor':1, 'co':0.5, 'timestamp':04:05:10\},}\\
\>\>\gt{\{'sensor':2, 'co':0.3\}}\\
\>\>\gt{]}
\end{tabbing}

We look for the ``jump" readings that are more than 2x the previous reading at
the same sensor. The following query solves the problem using \gl{GROUP BY}.

\begin{tabbing}
\ \ \ \=\gl{SELECT }\=\gl{sensor AS sensor,}\\
                          \>\>\gl{(}\=\gl{WITH }\=\gl{orderedReadings}\\
                          \>\>\>\>\gl{AS (SELECT v FROM oneSensorsReadings v ORDER BY v.timestamp)}\\
                          \>\>\>\gl{SELECT r.co, r.timestamp }\\
                          \>\>\>\gl{FROM orderedReadings r AT p}\\
                          \>\>\>\gl{WHERE r.co > 2*orderedReadings[p-1].co}\\
                          \>\>\>\gl{ORDER BY p}\\
                          \>\>\>\gl{) AS jumpReadings}\\
       \>\gl{FROM logs l}\\
       \>\gl{GROUP BY l.sensor AS sensor GROUP AS oneSensorsReadings}
\end{tabbing}

The result is
\begin{tabbing}
\ \ \ \=\gt{\ob}\=\gt{\{'sensor':1, 'jumpReadings':[\{'co':0.4, 'timestamp':04:05:06\}]\},}\\
\>\>\gt{\{'sensor':2, 'jumpReadings':[]\}}\\
\>\>\gt{\cb}
\end{tabbing}

\end{example}

\section{\gl{ORDER BY} clause}
\label{section:order-by}
SQL's \gl{ORDER BY} orders the output data. Similarly, the PartiQL \gl{ORDER BY}
is responsible for turning its input bag into an array. In the following
aspects, PartiQL extends the SQL semantics to resolve issues that are not
relevant in SQL but emerge when working on Ion data.
%
\begin{enumerate}
%
\item SQL's \gl{ORDER BY} clause orders its input using an expanded version of
the less-than function, which we call the {\em order-by less-than} and denote by
$\order$. The PartiQL $\order$ semantics (Section~\ref{sec:order-by-less-than})
also specify an order among values of heterogeneous types, including complex
values.
%
\item The interaction of \gl{ORDER BY} with a \gl{UNION} (or any other set
operator) of SFW queries requires attention since, unlike SQL, in PartiQL there
are no binding tuples (or any tuples at all for that matter) after a \gl{SELECT
VALUE} clause. Section~\ref{sec:order-by-and-setops} elaborates on this aspect
of PartiQL.
%
\item Unlike SQL, the input of an PartiQL query may also have order, because it
is an array. The user may want to preserve the order of the input into the
output. In this case, the \gl{AT} structure in the \gl{FROM} clause (recall,
Section~\ref{sec:single-item-from}) can capture the input order and the
\gl{ORDER BY} can recreate it. However, this order preservation mechanism is
tedious for the user. Thus, \gl{ORDER BY} also offers an order preservation
directive (Section~\ref{sec:order-preservation}).
%
\end{enumerate}

Sections~\ref{sec:orderby-sql-compatibility}
and~\ref{sec:select-variables-in-order} discuss SQL compatibility issues.


\subsection{PartiQL \gl{ORDER BY} Syntax}
\label{sec:orderby-syntax}
Similar to SQL, the PartiQL \gl{ORDER BY} clause syntax is: 
\[
\begin{array}{ll}
\gl{ORDER BY }( & e_1\ [\gl{ASC} | \gl{DESC}]?\ [\gl{NULLS FIRST}|\gl{NULLS LAST}]? \\
 & \vdots \\
 & e_m\ [\gl{ASC} | \gl{DESC}]?\ [\gl{NULLS FIRST}|\gl{NULLS LAST}]? \\
 &) \\
 &| \gl{PRESERVE}
\end{array}
\] 
\noindent (Figure~\ref{figure:query:bnf}), where $e_1 \ldots e_m$ is a list of
\textit{ordering expressions}.  In PartiQL a SFW query with \gl{ORDER BY}
outputs an array, whereas a SFW query without \gl{ORDER BY} outputs a bag. 

Alike SQL's \gl{ORDER BY} clause, the \gl{NULLS FIRST} and \gl{NULLS LAST}
keywords indicate whether \NULL and \MISSING values are ordered before
or after all other values. Notice that in PartiQL, the \gl{NULLS FIRST} and
\gl{NULLS LAST} refer to both \NULL and \MISSING.


\subsection{The PartiQL order-by less-than function}
\label{sec:order-by-less-than}
The \gl{ORDER BY} clause sorts its input using the \textit{order-by less-than
function} $\order$, which is able to compare values of different types (unlike
SQL). In particular:

\almann{Yannis, what about SEXP type here, it exists in Ion and in our PartiQL
implementation?  Also, I checked, this is exactly aligned with the definition I
implemented in the codebase so that is good.}

\begin{enumerate}
\item \gl{NULL} and \gl{MISSING} are always first or last and compare equally
according to $\order$. In other words, $\order$ cannot distinguish between
\gl{NULL} and \gl{MISSING}.
\item The boolean values are coming first among the non-absent values (i.e., $b
\order x$ is always true if $b$ is boolean and $x$ is not a \gl{NULL} or a
\gl{MISSING} or a boolean).  \gl{false} comes before \gl{true}.
\item The numbers come next. The comparisons between number values do not depend
on precision or specific type. Given two numbers $x$ and $y$, the PartiQL $x
\order y$ behaves identical to the SQL order-by less-than function. Namely, if
$x$ and $y$ are not the special values \gl{`-inf'}, \gl{`inf'} or \gl{`nan'},
then $x \order y$ is the same with $x \gl{<} y$. The special value \gl{`nan'}
comes before \gl{`-inf'}, which comes before all normal numeric values, which
are followed by \gl{`+inf'}.
\item Timestamp values follow and are compared by the absolute point of time
irrespective of precision or local UTC offset.
\item The text types come next ordered by their lexicographical ordering by
Unicode scalar irrespective of their specific type.
\item The LOB types follow and are ordered by their lexicographical ordering by
octet.
\item Arrays come next, and their values compare lexicographically based on the
comparison of their elements, recursively. Notice that given an array $[e_1,
\ldots, e_m]$ and a longer array $[e_1, \ldots, e_m, e_{m+1}, \ldots, e_n]$ that
has the same first $m$ values, the former array comes first.
\item Tuple values follow and compare lexicographically based on the sorted
attributes (as defined recursively), first by the attribute name, and secondly
by the attribute values themselves.
\item Bag values come last (except, of course, when \gl{NULLS LAST} is
specified) and their values compare by first reducing them to arrays by sorting
their elements and then comparing the resulting arrays.
\end{enumerate}

\subsection{Dependency of \gl{ORDER BY} semantics on the Presence of Set Operators}
\label{sec:order-by-and-setops}
Coming up...
\eat{
Similar to SQL, the presence of bag/set operators (such as \gl{UNION} and
\gl{UNION ALL}) in a SFW query results in different semantics for \gl{ORDER BY}
and, hence, PartiQL itemizes the semantics depending on the presence or absence
of set operators.

\subsubsection{SFW without bag/set operators} 
\label{sec:orderby-without-setops}
The \gl{ORDER BY} clause inputs a bag of binding tuples $B_{\gl{ORDER}}^{in}$,
thus each ordering expression $e_i$ can utilize variables from the preceding
\gl{FROM} or \gl{GROUP BY} clauses. The \gl{ORDER BY} clause outputs an array of
sorted binding tuples $B_{\gl{ORDER}}^{out}$.  The semantics are similar to
SQL's. Formally: 

Let $b_1, \ldots, b_n$ be the binding tuples in $B_{\gl{ORDER}}^{in}$, and for
each $b_j \in B_{\gl{ORDER}}^{in}$, let $v_{j,1} \ldots v_{j,m}$ be the results
of evaluating of the ordering expressions $e_1 \ldots e_m$. The order of the
tuples in the output depends on the comparison according to $\order$. Formally,
given two input binding tuples $b_j$ and $b_k$, $b_j$ appears before $b_k$ in
$B_{\gl{ORDER}}^{out}$ if $[v_{j,1} \ldots v_{j,m}] \order [v_{k,1} \ldots
v_{k,m}]$. If neither $[v_{j,1} \ldots v_{j,m} \order [v_{k,1} \ldots v_{k,m}]$
nor $[v_{k,1} \ldots v_{k,m} \order [v_{j,1} \ldots v_{j,m}]$ PartiQL
nondeterministically decides the order between $b_j$ and $b_k$. Sorting in
descending order is defined analogously.


\subsubsection{\gl{ORDER BY} with bag/set operators} 
\label{sec:order-with-set}
Consider SFW queries of the form
\[ q_1 \gl{ UNION } q_2 \gl{ ORDER BY } e_1 \ldots e_m \] \noindent where $q_1$
and $q_2$ are also SFW queries. (The specification generalizes in the obvious
way to include \gl{NULLS LAST} and/or \gl{DESC}.) As expected from SQL (and will
be further detailed for PartiQL in Section~\ref{section:setop}), the \gl{UNION}
clause is evaluated after the individual SFW queries $q_1$ and $q_2$. It outputs
a bag of values, which is in turn input by \gl{ORDER BY}. Thus, in the presence
of bag/set operators, instead of binding tuples, the \gl{ORDER BY} clause inputs
a bag of values $C = \gl{<<} u_1 \gl{, } \ldots \gl{, } u_n \gl{>>}$ and
outputs an array of sorted values. 

The ordering expressions $e_1 \ldots e_m$ use the special \gl{CURRENT} variable
to range over the elements (values) of their input bag $C$. Formally, suppose
the query is evaluated within environment $\env$. For each $u \in C$, let $v_1
\ldots v_m$ be the evaluation of the ordering expressions $e_1 \ldots e_m$
within environment $\env' = \langle \gl{CURRENT} : u \rangle || \env$. Given two
input element values $u_j$ and $u_k$, the \gl{ORDER BY} clause orders them by
comparing the results of the ordering expressions when evaluated in the
environment $\langle \gl{CURRENT} : u_j \rangle || \env$ with the results of the
ordering expressions in $\langle \gl{CURRENT} : u_k \rangle || \env$.

\begin{example}
The following query unions the results of two queries and then orders them
according to modulo 10.

\begin{tabbing}
\ \ \=\gl{(}\=\gl{SELECT VALUE }\texttt{x}\\
\>\>\gl{FROM }\texttt{[25, 11, 9] x }\gl{)} \\
\>\gl{UNION}\\
\>\>\gl{(}\=\gl{SELECT VALUE }\texttt{y}\\
\>\>\gl{FROM }\texttt{[3, 10] y }\gl{)} \\
\>\gl{ORDER BY mod(CURRENT, }\texttt{10}\gl{)}\\ 
\end{tabbing}

\noindent The result is the array \texttt{[10, 11, 3, 25, 9]}.
\end{example}


\subsection{Automatic order preservation}
\label{sec:order-preservation}
PartiQL also provides the \gl{ORDER BY PRESERVE} clause in lieu of the usual
\gl{ORDER BY} that uses expressions to order. The \gl{ORDER BY PRESERVE} is
applicable only when there is no \gl{GROUP BY} clause.

It orders the output elements according to the order of the respective input
elements. Technically it is the syntactic sugar \textbf{(S)} that is reduced to
core PartiQL \textbf{(C)} as shown below. Only the \gl{FROM} and \gl{ORDER}
clauses are shown, as the \gl{SELECT} and \gl{WHERE} are identical and \gl{GROUP
BY} is inapplicable. ~\\
\begin{tabular}{@{}l@{~}l@{~}l@{~}l@{~}l@{~}l@{}}
\textbf{(C)}    & \gl{FROM}     & $e_1$ \gl{AS} $v_1$ \gl{AT} $p_1,$&
\textbf{(S)}   & \gl{FROM}     & $e_1$ \gl{AS} $v_1,$  \\
    &                   & $\ldots$                                  &       &
    & $\ldots$                  \\
    &                   & $e_n$ \gl{AS} $v_n$ \gl{AT} $p_n$ &       & \gl{FROM}
    & $e_n$ \gl{AS} $v_n$   \\
    & \gl{ORDER BY} & $p_1, \ldots, p_n$                        &       &
    \gl{ORDER BY}    & \gl{PRESERVE}         \\
\end{tabular} \\

\highlight{Mistyping cases} It is possible that an \gl{AT} is introduced while
the \gl{FROM} clause iterates over one or more bags, i.e., over collections
whose elements have no order. In this case, the \gl{ORDER BY} will produce a
non-deterministic order. 

\begin{example}
Consider the following \gl{FROM} clause, where \gl{x} iterates over an array but
\gl{y} iterates over a bag.
\begin{tabbing}
\ \ \ \=\gl{SELECT x.a AS a, y.b AS b}\\
\>\gl{FROM [3, 2] AS x, \ob 4, 5 \cb\ AS y}\\
\>\gl{ORDER BY PRESERVE}
\end{tabbing}
It is equivalent to 
\begin{tabbing}
\ \ \ \=\gl{SELECT VALUE [x, y]}\\
\>\gl{FROM [3, 2] AS x AT xp, \ob 4, 5 \cb AS y AT yp}\\
\>\gl{ORDER BY xp, yp}
\end{tabbing}
The result is an array. One possible result is
\begin{tabbing}
\ \ \ \=\gl{[ [3, 4], [3, 5], [2, 5], [2, 4] ]}
\end{tabbing}
Another possible result is 
\begin{tabbing}
\ \ \ \=\gl{[ [3, 4], [3, 5], [2, 4], [2, 5] ]}
\end{tabbing}
Indeed, there are four possible results: The 3's are always ahead of the 2's;
all possible permutations of 4's and 5's are possible.
\end{example}
}

\subsection{SQL Compatibility \gl{ORDER BY} clauses}
\label{sec:orderby-sql-compatibility}

For SQL-compatibility, PartiQL allows the \gl{CURRENT} variable to be omitted
from ordering expressions. Then when the \gl{CURRENT} variable binds tuples, the
ordering expressions can refer directly to the attributes of those tuples.

\yannis{We have generally triggered rewritings like this by the presence of
schema. However, the ORDER BY is guaranteed to have single input (the bag $C$)
and thus it won't be plagued by the semantics, efficiency and stability problems
that led us to restrict the expansion of the environment in the case of FROM
clause items to schemaful variables. Thus I allowed it to not be
schema-specific. Plus, ORDER BY without this syntactic sugar would require us to
teach ``CURRENT" to even first time users.}

\almann{This is true except in the case of nested scopes, should we restrict
this similarly to the single FROM clause rewrites?\\
Put another way, I think this has to be schema-ful, but fortunately this is not
an issue with typical use of \lstinline|(*q1*) UNION (*q2*) ORDER BY (*e*)|
since $q1$ and $q2$ if they are SFW will have implicit schema and thus can be
easily explained.}

The complete scoping rules are as follows. When all of the following conditions
are satisfied:
\begin{enumerate}
\item an PartiQL path expression ordering expression $as$ appears in the
\gl{ORDER BY} of a \gl{UNION ... ORDER BY} query, where $a$ is an identifier and
$s$ is the potentially empty suffix of the path.
\item the expression $as$ is evaluated in database environment $\db$ and
variables' environment $\env$, which defines variables $v_1,\ldots, v_n$ and
none of them is named $a$. 
\item none of the variables $v_1,\ldots, v_n$ may bind to a tuple that has an
attribute $a$.
\end{enumerate}
\noindent then the path expression $as$ resolves to $\gl{CURRENT}.as$. 

The most common and useful way to have the 3rd condition be satisfied is when
the \gl{UNION ... ORDER BY} is a top-level query and, thus, the variables
environment $\env$ is empty.

\subsection{Use of \gl{SELECT} variables in \gl{ORDER BY} for SQL compatibility}
\label{sec:select-variables-in-order}
Recall from Section~\ref{section:environment-and-sfw} that \gl{ORDER BY} is
evaluated before \gl{SELECT}. For SQL-compatibility, given $\gl{SELECT } e_i
\texttt{ AS } a_i$, PartiQL also supports the syntactic sugar of using $a_i$ in
lieu of $e_i$ in the \gl{ORDER BY} clause. Therefore, both SFW queries below are
equivalent: ~\\
\begin{tabular}{@{}l@{~}l@{~}l@{~}l@{~}l@{~}l@{}}
(1) & \texttt{SELECT}   & $e_i$ \texttt{ AS } $a_i$   & (2)   & \texttt{SELECT}
& $e_i$ \texttt{ AS } $a_i$   \\
    & \texttt{FROM}     & \ldots                      &       & \texttt{FROM}
    & \ldots                      \\
    & \texttt{ORDER BY} & $a_i$                       &       & \texttt{ORDER
    BY} & $e_i$                       \\
\end{tabular} \\

\subsection{Coercion of literals for SQL compatibility}
\label{sec:literal-conversion}
\yannis{TO DO: move to comparisons. It's not an ORDER BY issue.}

Notice that definition of \gl{<} dismissed the SQL coercions. In SQL, given
explicit literals in a query, coercions may happen.

\begin{example}
The query
\begin{verbatim} 
SELECT * FROM foo WHERE 9 < '10'
\end{verbatim}
is equivalent to
\begin{verbatim} 
SELECT * FROM foo WHERE 9 < 10
\end{verbatim}
because an automatic coercion of string to number will be introduced. 
\end{example}

This aspect of SQL compatibility is introduced by rewriting. Namely, given a
query with incompatible types 

\section{\gl{UNION} / \gl{INTERSECT} / \gl{EXCEPT} clauses}
\label{section:setop}
Coming up...
\eat{
We describe next the PartiQL \texttt{UNION ALL}, \texttt{INTERSECT ALL} and
\texttt{EXCEPT ALL} bag operators, as well as their duplicate-eliminating
counterparts the \texttt{UNION}, \texttt{INTERSECT} and \texttt{EXCEPT}. While
SQL bag operators always input/output bags of flat, homogeneous tuples, PartiQL
queries have to face the following issues:
\begin{itemize}
\item the inputs may not be bags. Thus PartiQL semantics specify which cases
coerce and how Vs which cases fail.
\item the elements of the inputs may not be homogenous and they may not be
atomic values. Thus PartiQL semantics specify how comparisons between elements
are accomplished. 
\end{itemize}

\subsection{PartiQL Bag Operator Basics}
\label{sec:bagops-basics}
As in SQL, a bag operator%

\footnote{SQL's bag operators are often referred to as set operators, while the
term set is mathematically incorrect.}

is specified with: $q~S~q'$ (Figure \ref{figure:query:bnf}, lines 9-10), which
comprises a left query $q$, the bag operator $S$ and a right query $q'$. The bag
operator $S$ may be \texttt{UNION}, \texttt{INTERSECT} or \texttt{EXCEPT} and
may be optionally suffixed with \texttt{ALL}. Let $q \rightarrow v$ and $q'
\rightarrow v'$, where $v$ and $v'$ are arbitrary values. The \texttt{ALL}
variations input bags and output a bag of values, which may have duplicate
values even if the input does not. Presence of \texttt{ALL} specifies that the
output may have duplicate elements, while absence requires duplicate
elimination, in which case the output is an implicit set.

\subsection{Equivalence function used by Bag Operators}
\label{sec:equiv-bagops}
The bag operators utilize the function \gl{eqg} to decide the equivalence of
elements of their input. Notice it is the same function that is used by the
\gl{GROUP BY}, as described in Section~\ref{sec:eqg}. Furthermore, alike the
\gl{GROUP BY}, the \gl{UNION} and \gl{INTERSECT} operators retain the \gl{NULL}
whenever they compare a \gl{NULL} and a \gl{MISSING}.

\subsection{Operations on Non-Bags}
\label{sec:setops-on-non-bags}
Mistyping situations may emerge when one or both of the inputs to a set operator
is not a bag. In all cases, mistypings are treated identically to mistyping in
the \gl{FROM} clause, as described in Section~\ref{sec:bag-array-mistypings}. In
particular, if an input argument is any of the following, it is coerced
according to the following rules:

\begin{enumerate}
\item A scalar value $s$ is coerced into the bag $\ob s \cb$.
\item A tuple $t$ is coerced into the bag $\ob t \cb$.
\item An absent value $a$ is coerced into the bag $\ob a \cb$.
\item An array $r$ is coerced into the bag that dismisses the order of elements
in $r$.
\end{enumerate}

\subsection{Array versions of the Set Operators}
\label{sec:array-versions-of-setops}
SQL does not allow the set operators applied on ordered subqueries (i.e.,
queries with \gl{ORDER BY}). Thus there is no compatibility issue when we extend
the set operators of PartiQL to input arrays instead of bags. PartiQL allows its
users to easily express versions of the set operators where the inputs have
order and such order is easily preserved in the output.%

\footnote{The described functionality can also be accomplished without the array
operators that preserve input order. However, it is a tedious exercise to do the
job by using the \gl{AT} and \gl{ORDER BY} operators.}

\subsubsection{CONCAT} 
\label{sec:concat}
The query 
\[e_1\ \gl{CONCAT}\ e_2\] \noindent normally expects that $e_1$ and $e_2$ are
arrays. Its output is the concatenation of $e_1$ and $e_2$. Thus, it can be
thought of as the array version of \gl{UNION ALL}.

If $e_1$ is a bag and $e_2$ is an array, then the elements of $e_1$ appear
before the elements of $e_2$ in the output and the elements of $e_2$ maintain
their order. (Vice versa if $e_1$ is an array and $e_2$ is a bag.) If both $e_1$
and $e_2$ are bags, then the elements of $e_1$ appear before the elements of
$e_2$ in the output.

If $e_1$ or $e_2$ is neither an array nor a bag, it is conceptually coerced into
a singleton array (or singleton bag, since there is no difference) according to
the rules of Section~\ref{sec:setops-on-non-bags} and then proceed according to
the above semantics.

\subsubsection{ORDERED INTERSECT ALL}
\label{sec:ordered-intersect}
The query 
\[e_1\ \gl{ORDERED INTERSECT ALL}\ e_2\] \noindent normally expects that $e_1$
is an array and $e_2$ is either bag or array. It outputs the elements of $e_1$
that are found in $e_2$ and the qualified elements retain their order.

If $e_1$ is a bag the semantics are identical to \gl{INTERSECT ALL}.

If $e_1$ or $e_2$ is neither an array nor a bag, it is conceptually coerced into
a singleton array (or singleton bag, since there is no difference) according to
the rules of Section~\ref{sec:setops-on-non-bags} and then proceed according to
the above semantics.

\subsubsection{ORDERED EXCEPT ALL}
\label{sec:ordered-except}
The query 
\[e_1\ \gl{ORDERED EXCEPT ALL}\ e_2\] \noindent normally expects that $e_1$ is
an array and $e_2$ is either bag or array. It outputs the elements of $e_1$ that
are \textit{not} found in $e_2$ and the qualified elements retain their order.

If $e_1$ is a bag the semantics are identical to \gl{EXCEPT ALL}.

If $e_1$ or $e_2$ is neither an array nor a bag, it is conceptually coerced into
a singleton array (or singleton bag, since there is no difference) according to
the rules of Section~\ref{sec:setops-on-non-bags} and then proceed according to
the above semantics.

\yannis{to do: fix the grammar to (Figure~\ref{figure:query:bnf}.} 
}

%\section{\gl{WITH} and \gl{LET}}
\label{sec:with-let-letting}

\textit{TODO}

\section{\pivot Clause Semantics}
\label{sec:pivot}
The \pivot clause inputs a bag of binding tuples or an array of binding tuples.
Semantically, it is similar to \select \values but whereas the latter creates a
collection of values, \pivot constructs a tuple where the each input binding is
evaluated to an attribute value pair in the tuple.

The clause:

\begin{lstlisting}
PIVOT (*$v$*) AT (*$a$*)
\end{lstlisting}

\noindent inputs a bag or an array of binding tuples and outputs a single tuple
where each evaluation of $v$ and $a$ generate an attribute in the tuple.

\begin{example}
This example illustrates a \pivot that creates a tuple from a collection
of tuples.

\begin{lstlisting}
PIVOT x.v AT x.a
FROM << {'a': 'first', 'v': 'john'}, {'a': 'last', 'v': 'doe'} >>
\end{lstlisting}

\noindent The result is \lstinline|{'first':'john', 'last':'doe'}|.
\end{example}

The expression $a$ is expected to evaluate into a string value.  In strict mode,
it is an error if this evaluates to a non-string value.  In permissive mode, the
attribute is considered \MISSING and does not appear in the output.  The
expression $v$ can be any PartiQL value, but if it is \MISSING it will not be
generated in the resulting tuple.
\section{Structural Types and Type-related Query Syntax and Semantics (WIP)}
\label{sec:schema}
The input data generally conform to a \textit{structural type}, also often
called \textit{schema}. The SQL semantics make extensive use of the structural
types in order to assign meaning to queries, which would not have a meaning in
the absence of such structural types.

\almann{TODO: Add reference to static environment.}

In the interest of SQL compatibility and user convenience, PartiQL also allows
structural types to assign meaning to queries that would not have a meaning
otherwise.

We will soon specify the precise rules that provide SQL compatibility, 
while keeping the schema optional and the query results stable with respect to schema addition.

\eat{
\begin{example}
\label{xmpl:reldb-with-schema}
Consider an example relational database with two tables \gt{sensors} and
\gt{logs}, with respective schemas requiring that each sensors' tuple has an
integer attribute \gt{sensor} and each logs' tuple has an integer \gt{sensor}
attribute and a decimal \gt{co} attribute.%
\footnote{It is the same database used in Example~\ref{xmpl:nesting-readings}.}
\begin{tabbing}
\ \ \ \=\gt{sensors : [}\=\gt{\{'sensor':1\},}\\
\>\>\gt{\{'sensor':2\}}\\
\>\>\gt{]}\\
\ \ \ \=\gt{logs: [}\=\gt{\{'sensor':1, 'co':0.4\},}\\
\>\>\gt{\{'sensor':1, 'co':0.2\},}\\
\>\>\gt{\{'sensor':2, 'co':0.3\}}\\
\>\>\gt{]}
\end{tabbing}
Then the SQL query
\begin{tabbing}
\ \ \ \=\gl{SELECT s.sensor AS sensor, co AS reading}\\
\>\gl{FROM sensors AS s, logs AS l}\\
\>\gl{WHERE s.sensor = l.sensor}
\end{tabbing}
\noindent makes sense semantically, despite the expression \gt{co} not
specifying whether it is \gt{l.co} or \gt{s.co}, since the schema dictates that
only \gt{l} has a \gt{co} attribute. Thus \gt{co} means \gt{l.co}.
\end{example} 

\highlight{Remark} The term ``schema" has different meanings and different roles
in various contexts. In relational databases, the schema declarations are often
overloaded with declarations that dictate the specifics of storage. In querying
S3-based data, the PartiQL schema-as-view proposal overloads
schema-as-constraint to also become a data extraction and cleaning tool. As far
as the semantics of querying schemaful data are concerned, the only aspect of
schema that PartiQL querying cares about, is the constraints that the schema
induces on the input data.

\subsection{Structural Types}
\label{sec:structural-types}
The PartiQL structural type (henceforth called just type) syntax is identical to
the PartiQL literal syntax with the following additions. (We do not provide a
formal BNF syntax and semantics.)

\begin{enumerate}
\item Allow the name of an PartiQL type in lieu of a scalar literal.
\begin{example}
\label{xmpl:optional-date1-attr}
The type 
\begin{tabbing}
\ \ \ \gt{\ob \{'date1': DATE\} \cb} 
\end{tabbing}
describes a collection of tuples that have an optional \gt{date1} attribute and
this attribute has \gt{DATE} values. 
\end{example}

\begin{example}
A type may describe nested data. For example, the following describes a table
where the sales attribute is an array of tuples with an integer \gt{product} and
an integer \gt{amount}.
\begin{tabbing}
\ \ \ \=\gt{\ob \{}\=\gt{date1: DATE,}\\
\>\>\gt{sales:[\{product:INTEGER, amount:DECIMAL\}]}\\
\>\>\gt{\}\cb} 
\end{tabbing}
\end{example}

\item Allow a union type $t_1 | t_2 | \ldots | t_n$ in lieu of a scalar literal.
\begin{example}
\gt{date1} is either a \gt{DATE} or a \gt{INTEGER} according to the following
\begin{tabbing}
\ \ \ \gt{\ob \{date1: DATE | INTEGER\} \cb}
\end{tabbing}
\end{example}

\item Allow a constraint \gl{NOT MISSING} applied on an attribute.
\begin{example}
\label{xmpl:required-date1-attr}
The following type describes a collection of tuples that always have a
\gt{date1} attribute, which may nevertheless have \gl{NULL} value.
\begin{tabbing}
\ \ \ \gt{\ob \{'date1': DATE NOT MISSING\} \cb} 
\end{tabbing}
Contrast with Example~\ref{xmpl:optional-date1-attr}.
\end{example}

\item SQL's \gl{NOT NULL} is applied on any value - this includes attributes
values, as well as bag elements and array elements. Such value cannot be missing
and cannot be null.

\yannis{to almann and chris: This semantics has one downside: It makes it very
difficult to express (and impossible until we introduce named types) to express
that ``attribute $x$ may be missing but if it is there, then it is not null".
The reason I prefer this semantics for \gl{NOT NULL} is the straight
correspondence to SQL's \gl{NOT NULL}.}

\begin{example}
The following type describes a collection of tuples that always have a
\gt{date1} attribute and this attribute is never \gl{NULL}.
\begin{tabbing}
\ \ \ \gt{\ob \{'date1': DATE NOT NULL\} \cb} 
\end{tabbing}
Contrast with Examples~\ref{xmpl:optional-date1-attr}
and~\ref{xmpl:required-date1-attr}.
\end{example}

\item Allow \gl{OPEN} tuple types $\{a_1:t_1,\ldots,a_n:t_n,\gl{OPEN}\}$. The
open tuple type may have more attributes than the attributes $a_1,\ldots,a_n$
that are explicitly listed within its definition.
\begin{example}
The following type is a collection of tuples that absolutely have a \gt{date1}
attribute but they may also have other attributes.
\begin{tabbing}
\ \ \ \gt{\ob \{'date1': DATE NOT NULL, OPEN\} \cb} 
\end{tabbing}
In all prior examples, tuple types were \textit{closed} in the sense that they
explicitly specified the attributes they contain.
\end{example}

\end{enumerate}

\highlight{Future Extensions} Schema and schema-as-view will be expanded in
multiple directions, such as inclusion of \gl{CHECK}, \gl{DEFAULT},
\gl{REFERENCES}, etc. Nevertheless, the present definition of schema captures
any aspect that pertains to PartiQL query semantics.

\subsubsection{SQL's \gl{TABLE} as a special case of tuple collections}
\label{sec:create-table}
SQL's \gl{TABLE} is syntactic sugar for collection of closed tuples. The
following two are equivalent:
\[ \gl{TABLE(}a_1:t_1, \ldots, a_n:t_n\gl{)} \] \noindent and
\[ \ob \{ a_1:t_1, \ldots, a_n:t_n \} \cb \]

The keyword \gl{CREATE} is irrelevant in the context of this schema discussion,
since PartiQL does not necessitate that types specify storage. Eg, the data may
already exist.

\subsection{Tuple Navigations that Omit the Starting Variable}
\label{sec:paths-with-no-starting-variable}
For SQL compatibility, PartiQL allows tuple path navigations that do not
explicitly initiate from a variable; rather, they start from an attribute of the
tuples that bind to the variable. 

\begin{example}
\label{xmpl:sql-attribute-dereferencing}
Consider the SQL table \gt{R} that has an attribute \gt{a} and the table \gt{S}
that does not have an attribute \gt{a}. Then the SQL query
\begin{tabbing}
\ \ \ \=\gl{SELECT a, sv.b}\\
\>\gl{FROM R AS rv, S AS sv}
\end{tabbing}
\noindent is syntactic sugar for
\begin{tabbing}
\ \ \ \=\gl{SELECT rv.a, sv.b}\\
\>\gl{FROM R AS rv, S AS sv}
\end{tabbing}
That is, the path \gt{a} in the \gl{SELECT} clause is dereferenced by SQL into
\gt{rv.a}.
\end{example}

We carry over and generalize this feature for PartiQL. Notice that in PartiQL
not all variables have closed tuple types, as in SQL. This will require a few
extra conditions for the rewriting.

Formally: Consider 
\begin{enumerate}
\item an PartiQL path expression $a s$, where $a$ is an identifier and $s$ is
the (potentially empty) suffix of the path. 

\item this path expression is either (a) not a \gl{FROM} clause expression, or
(b) it is an $\gl{@}as$ expression (most likely in the \gl{FROM} clause).

\item this expression is evaluated in database environment $\db$ and variables
environment $\env$, which defines variables $v_1,\ldots,v_n$
\item a variable $v_{i_a}$ (among $v_1,\ldots,v_n$) has closed tuple type and
the tuple type has a (potentially missing and potentially null) attribute $a$.
Then $as$ will be rewritten into $v_{i_a}.as$ when there is none of the
conflicts described in the next items
\item none of the variables of $\env$ is named $a$ and is at the same scope with
$v_{i_a}$; if there is, then the identifier $a$ of $as$ resolves to the variable
named $a$. That is, the variable takes priority over the dereferenced attribute
when they are both in the same level. (This behavior is unlike Postgres, where a
compilation error would happen.)
\item none of the variables of $\env$ is named $a$ and is within the scope of
$v_{i_a}$, i.e., more local to the expression than the $v_{i_a}$. If it were,
then (as it happens in Postgres), the identifier $a$ of $as$ refers to the
variable $a$.
\item there is no other variable $v_{i'_a}$ that is (a) at the scope of
$v_{i_a}$, (b) is closed tuple-typed and (c) has an attribute named $a$. If
there is, the query cannot compile since we cannot tell whether $a$ should
become $v_{i_a}.a$ or $v_{i'_a}.a$. 
\item there is no other variable $v_{i'_a}$ that is (a) is within the scope of
$v_{i_a}$, (b) is closed tuple-typed and (c) has an attribute named $a$.  That
is, there is no variable that satisfies the conditions of $v_{i_a}$ but is more
local to the expression. (Same with SQL.) 
\item\label{item:stability-req} there is no variable $v'$ within the scope of
$v_{i_a}$ or at the same scope with $v_{i_a}$, such that $v$ can bind to a tuple
that has an attribute $a$. If this condition is not satisfied, we have a
compilation error. 
\end{enumerate}
\noindent If all of the above conditions hold, then $as$ is rewritten into
$v_{i_a}.as$. Notice that dereferencing to $v_{i_a}.as$ also takes priority over
a potential name $a$ in the database environment $\db$. 

\yannis{Item~\ref{item:stability-req} is introduced for ensuring stability upon the imposition of schema.}

\almann{Yannis, I don't understand the above comment.}

Notice that the variable $v'$ of Item~\ref{item:stability-req} has not been
required to have a tuple-type schema with attribut $a$. If the type of variable
$v'$ is any of the following cases, then $v'$ can bind to a tuple that has an
attribute $a$.

\begin{enumerate}
\item type \gl{ANY}
\item open tuple type
\item closed tuple type and this closed tuple type has attribute $a$
\item\label{item:union-cond} union type $t_1 | t_2 | \ldots | t_n$ and, recursively, at least one of the types $t_1, \ldots, t_n$ satisfies a condition from 1-\ref{item:union-cond}
\end{enumerate}

\begin{example} 
Consider the following slight modification of
Example~\ref{xmpl:sql-attribute-dereferencing}: \gt{R} and \gt{S} are still
collections of tuples, but these tuples need not necessarily have scalar values
only. Then the PartiQL query
\begin{tabbing}
\ \ \ \=\gl{SELECT a[5], sv.b}\\
\>\gl{FROM R AS rv, S AS sv}
\end{tabbing}
\noindent is syntactic sugar for
\begin{tabbing}
\ \ \ \=\gl{SELECT rv.a[5], sv.b}\\
\>\gl{FROM R AS rv, S AS sv}
\end{tabbing}
Notice, the path \gt{a[5]} had the identifier \gt{a}, which matched with the
schema of \gt{rv}, and the suffix \gt{[5]}, which is carried over in the
rewriting.
\end{example}

\begin{example}
\label{xmpl:variable-attribute-conflict}
When variable names conflict with identifiers that can be dereferenced to
``variable.attribute" paths, the variable names have priority. This behavior is
unlike Postgres, where the same issue can happen due to allowing variables that
range over arrays of scalars but Postgres would throw a compilation error.

The bag \gt{R} has closed tuples that have an attribute \gt{a}, among others. In
the following query the \gt{a} in the \gt{SELECT} clause stands for the variable
\gt{a} defined in the \gt{FROM} clause.
\begin{tabbing}
\ \ \ \=\gl{SELECT a}\\
\>\gl{FROM R AS rv, @rv.somearray AS a}
\end{tabbing}
\end{example}

\eat{
\begin{example}
The bag \gt{R} has again (as in Example~\ref{xmpl:variable-attribute-conflict})
closed tuples that have an attribute \gt{a}, among others. The following query
compiles, because we cannot tell whether \gt{a} stands for \gt{rv.a} or whether
it stands for the variable \gt{a} in the \gt{FROM} clause.
\begin{tabbing}
\ \ \ \=\gl{SELECT a}\\
\>\gl{FROM R AS rv}\\
\>\gl{WHERE EXISTS(}\=\gl{SELECT *}\\
\>\>\gl{FROM @rv.somearray AS a}\\
\>\>\gl{WHERE a > 5}
\end{tabbing}
The \gt{a} of the \gt{WHERE} clause is the variable \gt{a}, since it is more
local than \gt{rv.a}. The \gt{a} of the \gt{SELECT} clause is rewritten to
\gt{rv.a}.
\end{example}
}

\begin{example}
This example illustrates the stability provided by
Item~\ref{item:stability-req}.

Consider the following SQL query

\begin{tabbing}
\ \ \ \=\gl{SELECT *}\\
\>\gl{FROM gtable AS g}\\
\>\gl{WHERE EXISTS (SELECT * FROM ftable AS f WHERE x = f.y)}
\end{tabbing}

\noindent where \gt{gtable} is a Redshift table with schema \gt{\ob \{x:
INTEGER\} \cb} and \gt{ftable} is an S3 dataset that has no schema, i.e.,
technically its schema is \gl{ANY}. The \gt{x} in the \gl{WHERE} clause will not
be rewritten into \gt{g.x} and this query fail compilation due to the
unresolveable reference \gt{x}. Technically, the reason that it will not be
rewritten to \gt{g.x} is because there is the possibility that \gt{f} can also
bind to tuples that have \gt{x} (Item~~\ref{item:stability-req}).

Intuitively, the reason for not rewriting is the instability that would have
happened, had we allowed the rewriting. Then \gt{x} would be rewritten into
\gt{g.x}. Then the query would be equivalent to 

\begin{tabbing}
\ \ \ \=\gl{SELECT *}\\
\>\gl{FROM gtable AS g}\\
\>\gl{WHERE EXISTS (SELECT * FROM ftable AS f WHERE g.x = f.y)}
\end{tabbing}

Now, someone (different guy from the one who wrote the query) inspects the
\gt{ftable} and declares the schema \gt{\ob \{x: INTEGER\} \cb} for \gt{ftable}.
Now \gt{x} would resolve to \gt{f.x} and the query would be rewritten into

\begin{tabbing}
\ \ \ \=\gl{SELECT *}\\
\>\gl{FROM gtable AS g}\\
\>\gl{WHERE EXISTS (SELECT * FROM ftable AS f WHERE f.x = f.y)}
\end{tabbing}

Therefore, such query semantics are unstable in the sense that they change by
the mere definition of a schema. 
\end{example}

\highlight{Remark} The collective path expression rewriting
(Section~\ref{sec:deep-navigation}) preceeds the schema-based rewriting,
described in the present section.

\subsection{Attribute Ordinals in lieu of Attribute Names}
\label{sec:ordinals-in-lieu-of-names}
In certain SQL operations, an expression may refer to an attribute via its
ordinal position, as defined by the schema (\gl{CREATE TABLE}). PartiQL achieves
the same by utilizing schemas in order to statically map ordinal positions into
names. PartiQL does not allow the mapping in the absence of schemas, in order to
avoid unstable and expensive-to-execute semantics.

Consider a collection of tuples, where each tuple follows the closed tuple
schema $\{a_1:t_1,\ldots,a_n:t_m\}$. Such a collection is called a \textit{table
collection}. The following ``syntactic sugar" semantics apply to table
collections.

\subsubsection{Set Operations on Table Collections}
\label{sec:setops-on-tables}
Consider any set operation $\circ$ (\gl{UNION}, \gl{UNION ALL}, \gl{CONCAT},
etc) applied on two table collections $c^1$ and $c^2$ that \textit{both} have
schema. Furthermore, assume the schema of the tuples of $c^1$ is
$\{a_1^1:t_1^1,\ldots,a_n^1:t_n^1\}$ and the schema of the tuples of $c^2$ is
$\{a_1^2:t_1^2,\ldots,a_n^2:t_n^2\}$. Then the user should think of the
schemaful tuples as fixed-arity arrays, where the attribute names define the
order of the values in the array. In particular, $c^1 \circ c^2$ is equivalent
to

\begin{tabbing}
\ \ \ \=\gl{SELECT VALUE \{}$x[0]$ \gl{AS} $a_1^1\gl{,}\ldots\gl{,} x[n-1]$
\gl{AS} $a_n^1$\gl{\}}\\
\>\gl{FROM (}\=\gl{SELECT VALUE [}$v.a_1^1\gl{,} \ldots\gl{,} v.a_n^1$\gl{] FROM
}$c^1$ \gl{AS }$v$\\
\>\>$\circ$\\
\>\>\gl{SELECT VALUE [}$w.a_1^2\gl{,} \ldots\gl{,} w.a_n^2$\gl{] FROM }$c^2$
\gl{AS }$w$\\
\>\>\gl{) AS }$x$\\
\end{tabbing}
 
We can also rewrite $c^1 \circ c^2$ as
\[ c^1 \circ (\gl{SELECT}\ v.a_1^2\ \gl{AS}\ a_1^1, \ldots, v.a_n^2\ \gl{AS}\
a_n^1\ \gl{FROM}\ c^2 \gl{ AS } v) \] \noindent i.e., the set operation that
results from renaming the attribute names of the second argument into the
attribute names of the first argument. The two rewritings are equivalent.

Note that the collection schemas may not necessarily be declared; rather, they
may be inferred, as illustrated in the following example.

\begin{example}
Consider the query
\begin{tabbing}
\ \ \ \=\gl{SELECT v.a AS a, v.b AS b FROM c1 v}\\
\>\gl{UNION}\\
\>\gl{SELECT w.b AS b, w.a AS a FROM c2 w}
\end{tabbing} 
Assume there is no schema for \gl{c1} and \gl{c2}. Still, the first argument of
the \gl{UNION} has schema \gl{\ob \{a:ANY, b:ANY\}\cb} and the second one has
schema \gl{\ob \{b:ANY, a:ANY\}\cb}. Thus they are both collection tables.
Consequently, the above query is rewritten into
\begin{tabbing}
\ \ \ \=\gl{SELECT v.a AS a, v.b AS b FROM c1 v}\\
\>\gl{UNION}\\
\>\gl{SELECT w.b AS a, w.a AS b FROM c2 w}
\end{tabbing} 
\end{example}

\yannis{I believe this stability issue that emerges from the combination of  is
unavoidable.} Notice that SQL (and, for compatibility purposes, PartiQL) uses
the ordinals, even if the two collections' tuples have the same attributes. The
alteration of semantics by the presence of schemas can cause stability issues in
the sense that the introduction of a schema may alter the semantics of a query
that does not explicitly enumerate the attributes in its result, as illustrated
by the following example: 

\begin{example}
Consider two collections \gl{c1} and \gl{c2} with no schema and the following
data:
\begin{tabbing}
\ \ \ \=\gl{c1: \ob \{a:1, b:2\} \cb}\\
\ \ \ \=\gl{c1: \ob \{a:1, b:2\} \cb}
\end{tabbing}
The following query uses \gl{*} and thus does not explicitly enumerate the
attributes in the output
\begin{tabbing}
\ \ \ \=\gl{SELECT * FROM c1 v}\\
\>\gl{UNION}\\
\>\gl{SELECT * FROM c2 w}
\end{tabbing} 
\noindent It results into \gl{\{a:1, b:2\}}.

\yannis{Explain step by step that the * was expanded in the presence of schema}
Then, we give the following schemas to the two collections:
\begin{tabbing}
\ \ \ \=\gl{c1: \ob \{a:ANY, b:ANY\} \cb}\\
\ \ \ \=\gl{c2: \ob \{b:ANY, a:ANY\} \cb}
\end{tabbing}
Notice that the tuples are now ordered. Thus it matters that in \gl{c1} the
attribute \gl{a} is first and the attribute \gl{b} is second, while in \gl{c2}
the attribute order is vice versa. The result of the same \gl{UNION} query is
\begin{tabbing}
\ \ \ \=\gl{\ob }\=\gl{\{a:1, b:2\}}\\
\ \ \ \>\>\gl{\{a:1, b:2\} \cb}
\end{tabbing}
\noindent because it has the schema of the left argument of the \gl{UNION} 
\end{example}

}

















\end{document}
