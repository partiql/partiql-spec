\documentclass{article}


%------------------------------------------------------------------------------- 
% Math packages
%------------------------------------------------------------------------------- 

% The "amsmath" package provides advanced math extensions.
\usepackage{amsmath}

% The "amssymb" package adds new symbols to be used in math mode.
\usepackage{amssymb}

% The "amsthm" package adds the "proof" environment and "theoremstyle" command.
\usepackage{amsthm}
\theoremstyle{definition}

% The "faktor" package adds the "faktor" macro for variable substitution (i.e. vulgar fractions). This depends on the "amssymb" package.
\usepackage{faktor}

% The "semantic" package adds new macros for (PL-style) inference rules.
\usepackage[inference]{semantic}

%------------------------------------------------------------------------------- 
% Figure packages
%------------------------------------------------------------------------------- 

% The "fancyvrb" package provides advanced customization of verbatim environments, such as font families, numbering lines, box borders etc.
\usepackage{fancyvrb}

% The "graphicx" package allows including external graphic files.
%\usepackage{graphicx}

% The "subfig" package allows multiple sub-figures within a single figure, where sub-figures can be separately captioned and labeled, e.g. Figure % 1.2(a). This is a replacement for the older "subfigure" package.
\usepackage{subfig}

% HACK: The caption package (included by the subfig package) requires a counter for ACM's copyright box.
\newcounter{copyrightbox}

% The "float" package allows the "H" option for figures, which places a float % at a precise location.
\usepackage{float}

% The "caption" package allows captions for figures that are not actually in a floating environment (e.g. framed environment).
%\usepackage{caption}

% The "mdframed" package creates framed regions that can break across pages.
\usepackage{mdframed}

% The "algorithm2e" package provides keywords for typesetting algorithms. The "noend" option disables the printing of the "end" keywords. Use "algomargin" to decrease the margins for all algorithms.


% The "multirow" package allows table cells to span more than one row.
\usepackage{multirow}

% The "balance" package allows columns of the last page to be of equal height.
\usepackage{balance}

% The "fixltx2e" package prevents two-column figures from being placed out-of-order wrt regular (one-column) figures.
%\usepackage{fixltx2e}

% The "beramono" package provides Bitstream Vera Mono, which has a bold typewritter fontface.
\usepackage[scaled]{beramono}
\usepackage[T1]{fontenc}

% The "courier" package provides Courier, which has a bold typewritter fontface.
%\usepackage{courier}


%------------------------------------------------------------------------------- 
% Misc packages
%------------------------------------------------------------------------------- 

% The "optional" package allows multiple versions of the document via optional text.
%\usepackage{optional}

% The "lmodern" package allows for better font size support https://ctan.org/pkg/lm
\usepackage{lmodern}

% The "upquote" package formats quotes appropriately in verbatim
\usepackage{upquote}

% The "xstring" package allows switch/case conditionals.
\usepackage{xstring}

% The "xcolor" package allows colored text and backgrounds.
\usepackage[table]{xcolor}

% The "xspace" package allows us to write unit macros without protected spacing
\usepackage{xspace}

% The "soul" package allows highlighting.
\usepackage{soul}

% The "ulem" package allows highlighting.
\usepackage[normalem]{ulem}

% The "tocloft" packages allows generating custom lists that are similar to table of contents, list of figures etc.
% \usepackage[subfigure]{tocloft}

% The "hyperref" package allows creating hyperlinks. Note that it must be the last package loaded, and will automatically includes the "url" package.
\usepackage[colorlinks]{hyperref}
\hypersetup{ 
  linkcolor=blue
} 

% The "hypcap" package fixes "hyperref" so that hyperlinks go to the top of a float (as opposed to its caption).
\usepackage[all]{hypcap}

% Adjust whitespace before/after floats
\setlength{\textfloatsep}{6pt plus 1.0pt minus 2.0pt}
\setlength{\floatsep}{6pt plus 1.0pt minus 1.0pt}

%%%%
%% Use better margins
%%%%
\usepackage[margin=0.7in]{geometry}

%%%%
%% Control header and footer 
%%%%
\usepackage{lastpage} % used in customized footer
\usepackage{fancyhdr}
\setlength{\headheight}{16pt} 
\pagestyle{fancy}
\fancyhf{}
\lhead{--- DRAFT ---}
\rhead{\rightmark}
\lfoot{\today}
\rfoot{Page \thepage\ of \pageref{LastPage}}

%%%%
%% Formatting source code 
%% 
%% Use Cases: 
%% 1. Include a file with source code 
%%    \lstinputlisting[language=Java]{HelloWorld.java}
%% 
%% 2. Environment 
%%    \begin{lstlisting}[language=Scheme] 
%%    (foldl + 0 '(1 2 3))
%%    \end{lstlisting}
%%
%% 3. Inline with English text
%%      \lstinline{int i;}
%%%%
\usepackage{listings}

\lstset{
language=SQL,                           % Code langugage
basicstyle=\ttfamily,                   % Code font, Examples: \footnotesize, \ttfamily
keywordstyle=\textbf,                   % Keywords font ('*' = uppercase)
escapeinside={(*}{*)},                  % Escape syntax e.g. (*$\Longrightarrow$*)
% commentstyle=\color{gray},              % Comments font
% numbers=left,                           % Line nums position
% numberstyle=\tiny,                      % Line-numbers fonts
% stepnumber=1,                           % Step between two line-numbers
% numbersep=5pt,                          % How far are line-numbers from code
% tabsize=2,                              % Default tab size
% captionpos=b,                           % Caption-position = bottom
% breaklines=true,                        % Automatic line breaking?
% breakatwhitespace=false,                % Automatic breaks only at whitespace?
% showspaces=false,                       % Dont make spaces visible
% showtabs=false,                         % Dont make tabs visible
frame=none,                             % A frame around the code
aboveskip=0.25in,                       % Top margin
belowskip=0.25in,                       % Bottom margin
xleftmargin=0.25in,                     % Left margin
xrightmargin=0.25in,                    % Right margin
columns=flexible,                       % Column format
morekeywords={                          % PartiQL additional keywords
  MISSING,
  VALUE, PIVOT, UNPIVOT,
  STRING, STRUCT, TUPLE, DECIMAL, INT, BOOL,
},
}

%------------------------------------------------------------------------------- 
% Macros
%------------------------------------------------------------------------------- 

% Define our own compact enumerate
\newenvironment{compact_enum}
{\setlength{\leftmargini}{1em}
\begin{enumerate}
  \setlength{\labelsep}{.3em} 
  \setlength{\itemsep}{.4em}
  \setlength{\parskip}{0pt}
  \setlength{\parsep}{0pt}}
{\end{enumerate}}

% Define our own compact itemize
\newenvironment{compact_item}
{\setlength{\leftmargini}{1em}
\begin{itemize}
  \setlength{\labelsep}{.3em} 
  \setlength{\itemsep}{.4em}
  \setlength{\parskip}{0pt}
  \setlength{\parsep}{0pt}}
{\end{itemize}}


%------------------------------------------------------------------------------- 
% Database symbols
%------------------------------------------------------------------------------- 

\def\join{$\bowtie$}
\def\ojoin{\setbox0=\hbox{$\bowtie$}%
  \rule[-.02ex]{.25em}{.4pt}\llap{\rule[\ht0]{.25em}{.4pt}}}
\def\leftouterjoin{\mathbin{\ojoin\mkern-5.8mu\bowtie}}
\def\rightouterjoin{\mathbin{\bowtie\mkern-5.8mu\ojoin}}
\def\fullouterjoin{\mathbin{\ojoin\mkern-5.8mu\bowtie\mkern-5.8mu\ojoin}}
\def\semijoin{\mbox{$\mathrel{\raise1pt\hbox{\vrule height5pt depth0pt\hskip-1.5pt$>$\hskip -2.5pt$<$}}$}}
\def\antisemijoin{\overline{\semijoin}}

%------------------------------------------------------------------------------- 
% Grammar symbols (BNFs and Tree Grammars) 
%------------------------------------------------------------------------------- 

% Formatting commands for tree grammars
\newcommand{\gn}[1]  {\textit{#1}}           % (N)on-terminal
\newcommand{\gt}[1]  {\texttt{\textbf{#1}}}  % (T)erminal
\newcommand{\gl}[1]  {\texttt{\textbf{#1}}}  % (L)iteral
\newcommand{\gs}[1]  {\textit{\textbf{#1}}}  % (S)pecial construct
\newcommand{\gob}[0] {[}                     % (O)ptional begin
\newcommand{\goe}[0] {]}                     % (O)ptional end
\newcommand{\gr}[0]  {...}                   % (R)epeat
\newcommand{\gp}[0]  {$\rightarrow$}         % (P)roduction rule
\newcommand{\gd}[0]  {$|$}                   % (D)isjunction

%------------------------------------------------------------------------------- 
% Misc symbols
%-------------------------------------------------------------------------------

% Undirected single quote mark
\chardef\singlequote=13




\newcommand{\TRUE}{\gl{TRUE}\xspace}
\newcommand{\FALSE}{\gl{FALSE}\xspace}
\newcommand{\NULL}{\gl{NULL}\xspace}
\newcommand{\MISSING}{\gl{MISSING}\xspace}


\newcommand{\ionquote}[1]{\gl{\bt} #1 \gl{\bt}}


% MODEL definitions
\newcommand{\bt}{\texttt{\`}}
\newcommand{\ob}{<\!\!<}
\newcommand{\cb}{>\!\!>}
\newcommand{\out}{\{-}
\newcommand{\cut}{-\}}



% Let's make sure each section starts on its own page
\let\oldsection\section
\renewcommand\section{\clearpage\oldsection}

\newcounter{issue}

\newcommand{\reminder}[1]{\textbf{OPEN~\refstepcounter{issue}: \emph{#1}}}
\renewcommand{\reminder}[1]{} % comment this line to enable \reminder
%\newcommand{\reminder}[1]{{\refstepcounter{issue}}}
\newcommand{\reminderclosed}[1]{\textbf{RESOLVED~\refstepcounter{issue}: \emph{#1}}}
\renewcommand{\reminderclosed}[1]{} % comment this line to enable \reminderclosed
%\newcommand{\reminderclosed}[1]{{\refstepcounter{issue}}}
\newcommand{\reminderfuture}[1]{\textbf{FUTURE~\refstepcounter{issue}: \emph{#1}}}
\renewcommand{\reminderfuture}[1]{} % comment this line to enable \reminderfuture
%\newcommand{\reminderfuture}[1]{{\refstepcounter{issue}}}


\newcommand{\almann}[1]{\reminder{Almann: #1}}
\newcommand{\yannis}[1]{\reminder{Yannis: #1}}
\newcommand{\yannisclosed}[1]{\reminderclosed{Yannis: #1}}
\newcommand{\almannclosed}[1]{\reminderclosed{Almann: #1}}	
\newcommand{\yannisfuture}[1]{\reminderfuture{Yannis: #1}}
\newcommand{\almannfuture}[1]{\reminderfuture{Almann: #1}}

\newcommand{\eat}[1]{}


\def\ground{\bot}
\def\collscan{S^C}
\def\outercollscan{O^C}
\def\tuplescan{S^T}
\def\outertuplescan{O^T}
\def\tuplenav{N^T}
\def\arraynav{N^A}
\def\constructcoll{C^C}
\def\corr{R}
\def\flat{F}

\def\evalto{\leadsto}
\def\group{f_\gl{GROUP}}
\def\order{<^o}
%\def\order{f_\gl{ORDER}}x

\def\arr{\operatorname{array}}
\def\bag{\operatorname{bag}}
\def\tuple{\operatorname{tuple}}
\def\map{\operatorname{map}}
\def\fst{\operatorname{fst}}
\def\sub{\operatorname{sub}}
\def\sort{\operatorname{sort}}
\def\setop{\operatorname{set\_op}}
\def\unionall{\operatorname{union\_all}}
\def\intersectall{\operatorname{intersect\_all}}
\def\exceptall{\operatorname{except\_all}}
\def\setopeq{\overset{@}{=}}


% Operators
\newcommand{\g}{\{\!\{\langle \rangle\}\!\}}
\newcommand{\term}[1]{\ensuremath{\ddot{#1}}}
\newcommand{\op}[1]{\ensuremath{\operatorname{#1}}}
%\newcommand{\scancoll}{\ensuremath{\ggg^C}}
\newcommand{\scanbag}{\ensuremath{\ggg^{\{\!\{\}\!\}}}}
\newcommand{\leftscanbag}{\ensuremath{{\stackrel{\circ}{\ggg}}^{\{\!\{\}\!\}}}}
\newcommand{\scanarray}{\ensuremath{\ggg^{[]}}}
\newcommand{\scantuple}{\ensuremath{\ggg^{\{\}}}}
%\newcommand{\leafscancoll}{\ensuremath{\bar{\ggg}^C}}
\newcommand{\leafscanbag}{\ensuremath{\bar{\ggg}^{\{\!\{\}\!\}}}}
\newcommand{\leafscanarray}{\ensuremath{\bar{\ggg}^{[]}}}
\newcommand{\leafscantuple}{\ensuremath{\bar{\ggg}^{\{\}}}}
\newcommand{\navarray}{\ensuremath{\gl{[]}}}
\newcommand{\navtuple}{\ensuremath{\bullet}}
\newcommand{\functioncall}{\ensuremath{\lambda}}
\newcommand{\constructarray}{\ensuremath{\square^{[]}}}
\newcommand{\constructbag}{\ensuremath{\square^{\{\{\}\}}}}
\newcommand{\constructtuple}{\ensuremath{\square^{\{\}}}}
\newcommand{\returnarray}{\ensuremath{r^{[]}}}
\newcommand{\returnbag}{\ensuremath{r^{\{\!\{\}\!\}}}}
\newcommand{\returntuple}{\ensuremath{r^{\{\}}}}
\newcommand{\applyplan}{\ensuremath{\alpha}}
\newcommand{\applyfunction}{\ensuremath{\lambda}}
\newcommand{\unnest}{\ensuremath{unnest}}
\newcommand{\unnestout}{\ensuremath{unnest^o}}
\newcommand{\nav}{\ensuremath{nav}}
%\newcommand{\select}{\ensuremath{\sigma}}
\newcommand{\project}{\ensuremath{\pi}}
\newcommand{\innerjoin}{\ensuremath{\Join}}
\newcommand{\psemijoin}{\ensuremath{\hat\ltimes}}
\newcommand{\pjoin}{\ensuremath{\hat\innerjoin}}
\newcommand{\kjoin}{\ensuremath{\bar\ltimes}}
\newcommand{\outerjoinline}{\setbox0=\hbox{$\Join$}%
    \rule[0.1ex]{.27em}{.4pt}\llap{\rule[1.3ex]{.27em}{.4pt}}}
\newcommand{\leftjoin}{\ensuremath{\mathbin{\outerjoinline\mkern-5.8mu\Join}}}
\newcommand{\rightjoin}{\ensuremath{\mathbin{\Join\mkern-5.8mu\outerjoinline}}}
\newcommand{\fulljoin}{\ensuremath{\mathbin{\outerjoinline\mkern-5.8mu\Join\mkern-6.4mu\outerjoinline}}}
\newcommand{\inner}{\ensuremath{\stackrel{\leftarrow}{\Join}}}
\newcommand{\outercorr}{\ensuremath{\stackrel{\leftarrow}{\leftjoin}}}
\newcommand{\groupby}{\ensuremath{\gamma}}
\newcommand{\union}{\ensuremath{\Cup}}
\newcommand{\intersect}{\ensuremath{\Cap}}
\newcommand{\except}{\ensuremath{\setminus}}
\newcommand{\distinct}{\ensuremath{\delta}}
%\newcommand{\semijoin}{\ensuremath{\ltimes}}
\newcommand{\antijoin}{\ensuremath{\vartriangleright}}
\newcommand{\val}[1]{\ensuremath{\textbf{x}}}

% GENERAL definitions
\newcommand{\sql}{SQL Compatibility}
\newcommand{\pl}{Functional Programming}

 
% ENVIRONMENT definitions
\newcommand{\env}{\rho}
\newcommand{\typeenv}{\Gamma}

\newcommand{\db}{\env_{0}}
\newcommand{\typedb}{\typeenv_{0}}

\newcommand{\benv}{\ensuremath{(\db, \env)}}

\newcommand{\type}{\tau}

\newcommand{\eval}{\rightarrow}
\newcommand{\eqv}{\Leftrightarrow}
\newcommand{\neqv}{\nLeftrightarrow}

% FROM clause definitions
\newcommand{\from}{\gl{FROM}\xspace}
\newcommand{\JOIN}{\gl{JOIN}\xspace}
\newcommand{\CROSSJOIN}{\gl{CROSS JOIN}\xspace}
\newcommand{\LEFTCJOIN}{\gl{LEFT CROSS JOIN}\xspace}
\newcommand{\FULLCJOIN}{\gl{FULL CROSS JOIN}\xspace}
\newcommand{\LEFTJOIN}{\gl{LEFT JOIN}\xspace}
\newcommand{\FULLJOIN}{\gl{FULL JOIN}\xspace}
\newcommand{\isCollection}{\gl{isCollection}\xspace}
\newcommand{\at}{\gl{AT}\xspace}
\newcommand{\as}{\gl{AS}\xspace}
\newcommand{\on}{\gl{ON}\xspace}
\newcommand{\unpivot}{\gl{UNPIVOT}\xspace}

% SELECT clause definitions
\newcommand{\select}{\gl{SELECT}\xspace}
\newcommand{\values}{\gl{VALUE}\xspace}
\newcommand{\pivot}{\gl{PIVOT}\xspace}
\newcommand{\pivotPairs}{\gl{PIVOTPAIRS}\xspace}
\newcommand{\tupleunion}{\gl{TUPLEUNION}\xspace}
\newcommand{\anv}{\gl{ANV}\xspace}
\newcommand{\case}{\gl{CASE}\xspace}
\newcommand{\when}{\gl{WHEN}\xspace}
\newcommand{\isTuple}{\gl{isTuple}\xspace}
\newcommand{\inventName}{\gl{inventName}\xspace}

\newtheorem{example}{Example}

\newcommand{\inrecord}[1]   { {#1}_{\operatorname{in}} } % {\dot{#1}}
\newcommand{\outrecord}[1]  { {#1}_{\operatorname{out}} } % {\ddot{#1}}
\newcommand{\inbinding}[1]  { {#1}_{\operatorname{in}} }
\newcommand{\outbinding}[1] { {#1}_{\operatorname{out}} }

\newcommand{\id}[1]{{#1}^{id}}
\newcommand{\evalf}{\gl{eval}}

\newcommand{\update}[2]{\gl{update}(#1,#2)}
\newcommand{\insertbag}[2]{\gl{insertinbag}(#1,#2)}
\newcommand{\inserttuple}[3]{\gl{inserinttuple}(#1,#2,#3)}
\newcommand{\append}[2]{\gl{append}(#1,#2)}
\newcommand{\delete}[1]{\gl{delete}(#1)}
\newcommand{\insertorder}[2]{\gl{insertorder}(#1,#2)}

\newcommand{\highlight}[1]{\noindent\textbf{#1:}}

\newcounter{query}

\begin{document}
\title{PartiQL Specification}
\author{The PartiQL Specification Committee}
\maketitle
\newpage

\tableofcontents
\newpage
\input{license}
\section{Introduction}
\label{sec:introduction}

\paragraph{Draft Status} This document is currently a working draft and subject
to change.  Certain sections are marked as ``work in progress'' (WIP) and will
be expanded soon.

\paragraph{Audience} This document presents the formal syntax and semantics of
PartiQL. It is oriented to PartiQL query processor builders who need the full
and formal detail on PartiQL.

SQL users who are not interested in the full detail and the complete formalism
but are interested in learning how PartiQL extends SQL may also read the
tutorial. Unlike this formal specification, the tutorial has a ``how to''
orientation and is primarily driven by examples.

\paragraph{PartiQL core and PartiQL syntactic sugar}
In the interest of precision and succinctness, we tier the PartiQL specification
in two layers: The PartiQL core is a functional programming language with
composable aspects. Three aspects of the PartiQL core syntax and semantics are
characteristic of its functional orientation: Every (sub)query and every (sub)
expression input and output PartiQL data. Second, each clause of a SELECT query
is itself a function. Third, every (sub)query evaluates within the environment
created by the database names and the variables of the enclosing queries.

Then we layer ``syntactic sugar'' features over the core. Commonly, syntactic
sugar achieves well-known SQL syntax and semantics. Formally, every syntactic
sugar feature is explained by reduction to the core.

\subsection{Notation}
\label{sec:notation}

We use extended Backus-Naur form (EBNF) to describe
grammars. Non-terminal terms appear in italics, e.g., \nt{value}, and
terminal appear in typewriter font, e.g., \terminal{NULL}. A
production rules uses \prod to define the expansion of a non-terminal
and \alt for alternatives, e.g.,

\begin{displaymath}
  \begin{grammar}
    \nt{value} & \prod & \nt{string} & \mbox{\nt{value} expands into either \nt{string}}\\
               & \alt  & \terminal{null} & \mbox{or the literal \terminal{null}}\\
  \end{grammar}
\end{displaymath}

We also use $\seplof{\x{}}{,}$ to denote a possibly empty list of
comma separated elements $\x{}$, i.e.,

\begin{displaymath}
  \begin{array}{lll}
    \seplof{\x{}}{,} & = & \x{1}, \ldots, \x{n} \quad n \geq 0 \\
  \end{array}
\end{displaymath}

\section{Data Model}
\label{sec:model}

\begin{figure}[ht!]
  \centering
  \rule[1ex]{\textwidth}{0.1pt}
  \begin{displaymath}
  \begin{grammar}
    \nt{value}  & \prod & \nt{absent}                                                                   & \\
                & \alt  & \nt{scalar}                                                                   & \\
                & \alt  & \nt{tuple}                                                                    & \\
                & \alt  & \nt{collection}                                                               & \\
    \nt{absent} & \prod & \terminal{\NULL}                                                              & \\
                & \alt  & \terminal{\MISSING}                                                           & \\
    \nt{scalar} & \prod & \ionlit{\ntsub{lit}{\mbox{Ion}}}                                              & \\
                & \alt  & \ntsub{lit}{\mbox{SQL}}                                                       & \\
    \nt{tuple}  & \prod & \pqltuple{\seplof{\pqlpair{\nt{string}}
                                                    {\nt{value}}}
                                           {,}}                                                         & \\
                &       & \begin{array}{r@{\ }l}
                            \mbox{where} & \nt{string} \mbox{ is any valid Ion or SQL string literal}\\
                            \mbox{and}   & \nt{value} \neq \terminal{\MISSING}  \\
                          \end{array}\\

     \nt{collection}            & \prod & \nt{array}                                           & \\
                                & \alt  & \nt{bag}                                             & \\
     \nt{array}                 & \prod & \pqlarray{\seplof{\nt{value}}{,}}                    & \\
     \nt{bag}                   & \prod & \pqlbag{\seplof{\nt{value}}{,}}                      & \\
     \ntsub{lit}{\mbox{Ion}} & \prod & \mbox{any valid Ion literal value~\cite{ion:spec}}   & \\
     \ntsub{lit}{\mbox{SQL}} & \prod & \mbox{any valid SQL literal value~\cite{sql92:spec}} & \\
  \end{grammar}
  \end{displaymath}
  \rule[1ex]{\textwidth}{0.1pt}
  \caption{BNF grammar for \pql values.}
  \label{fig:pgl-values-grammar}
\end{figure}

\begin{figure}[ht!]
\centering
\begin{tabular}{|r|lrl|}
\hline
 1  & \gn{value}                    & \gp   & \gn{absent\_value} \\
 2  &                               & \gd   & \gn{scalar\_value} \\
 3  &                               & \gd   & \gn{tuple\_value} \\
 4  &                               & \gd   & \gn{collection\_value} \\
 5  & \gn{absent\_value}            & \gp   & \NULL \\
 6  &                               & \gd   & \MISSING \\
 7  & \gn{scalar\_value}            & \gp   & \ionquote{\gs{ion\_literal}} \\
 8  &                               & \gd   & \gs{sql\_literal} \\
 9  & \gn{tuple\_value}             & \gp   & \gl{\{\ \}} \\
10  &                               & \gd   & \gl{\{} \gs{string\_value} \gl{:} \gn{value} (\gl{,} \gs{string\_value} \gl{:} \gn{value})* \gl{\}} \\
11  &                               &       & \ \ \ \ \ \ \ \gn{value} cannot be \MISSING \\
12  & \gn{collection\_value}        & \gp   & \gn{array\_value} \\
13  &                               & \gd   & \gn{bag\_value} \\
14  & \gn{array\_value}             & \gp   & \gl{[} (\gn{value} (\gl{,} \gn{value})*)? \gl{]} \\
15  & \gn{bag\_value}               & \gp   & \gl{\ob} (\gn{value} (\gl{,} \gn{value})*)? \gl{\cb} \\
\hline
\end{tabular}
\caption{BNF Grammar for PartiQL Values}
\label{figure:values:bnf}
\end{figure}

\newcommand{\linevalues}[1]{%
    \IfEqCase*{#1}{%
    {value}{BNF lines~1--4}%
    {missing}{(BNF line~6)}%
    {scalar}{(BNF lines~7--8)}%
    {tuple}{(BNF lines~9--11)}%
    {tuplemissing}{(BNF line~11)}
    {collection}{(BNF lines~12--13)}%
    }[\errmessage{Unable to ref #1 for value BNF}]%
}

Figure~\ref{figure:values:bnf} shows the BNF grammar for PartiQL values. A
PartiQL database generally contains one or more PartiQL \textit{named values}. A
\textit{name}, is an identifier, such as a table name, that is associated with a
PartiQL value.  Section~\ref{section:environment-and-sfw} defines what these
names are, and how SQL qualified names work, in detail.

The type of a value is \gn{absent}, \gn{scalar}, \gn{tuple} or \gn{collection}.
Further subtyping applies to scalars, tuples, and collections. We will often use
the name \emph{complex value} to refer to any non-scalar and non-absent value.
That is, complex values include \textit{tuples} and \textit{collections}. A
tuple is a set of attribute name/value pairs, where each name is a string (as in
SQL). A tuple in the data model is unordered. A conventional SQL tuple is an
ordered tuple since the schema dictates the order of the attributes and certain
SQL operations may use the order---support for this is described in detail in
Section~\ref{sec:schema}.

\yannis{OS: Next paragraph is odd. We give up on data model and talk about a known format (JSON)
and Ion (as a format). I recommend that we, instead, cast the Ion type system as PartiQL's
recommended extension over SQL. I'm not sure why JSON deserves a special mention.
}

\almann{How does the following look?}

PartiQL's data model extends SQL to Ion's type system to cover schema-less and
nested data. Such values can be directly quoted with \bt quotes.

Unlike SQL, PartiQL allows the possibility of duplicate attribute names, in the
interest of compatibility with non-strict JSON/Ion datasets. However PartiQL
does not encourage duplicate attribute names; navigation into tuples with the
conventional dot notation (Section~\ref{section:paths}) is tuned to the
assumption that the attribute names are unique.

A \gn{collection\_value} is either an ordered or unordered
\linevalues{collection}. Both arrays and bags may contain duplicate elements. An
array is ordered (similar to a JSON array or Ion list) and each element is
accessible by its ordinal position. (See specifics of access by position in
Section~\ref{section:paths}.) Arrays are delimited with \gl{[} \gl{]}. For
example, the value of the attribute \texttt{configurationItems} in
Figure~\ref{figure:values:example-value} is an array. Arrays have size, which
is not explicitly denoted but is implied by the number of elements in the array.
For example, the size of the \texttt{configurationItems} in
Figure~\ref{figure:values:example-value} is 2. The first element of an array
corresponds to index 0; the last element corresponds to index size minus one.

In contrast, a bag is unordered (similar to a SQL table) and its elements cannot
be accessed by ordinal position. Bags are denoted with \gt{\ob} and \gt{\cb}.

Finally, note that PartiQL has two kinds of absent values: \NULL and
\MISSING. The motivation is as follows: Unlike SQL, where a query that
refers to a non-existent attribute name is expected to fail during compilation,
in semi-structured data one expects a query to operate even if some of the tuples
do not define some of the attributes that the query's paths mention. Hence
PartiQL contains the special value \MISSING \linevalues{missing}, which is
the path result in cases where navigation fail to bind to any information. The
distinction between \MISSING and \NULL enables retaining the original
distinction between a missing attribute and a null-valued attribute. The utility
of \MISSING (as opposed to just having \NULL) will become further
apparent when navigation into semi-structured data and construction of
semi-structured results is discussed.

The value \MISSING may not appear as an attribute value. Notice that in the
interest of readability, the syntax of Figure~\ref{figure:values:bnf} does
exclude these cases; rather the ``\gn{value} cannot be \MISSING''
restrictions \linevalues{tuplemissing} indicate that \MISSING cannot appear
as an attribute value.

\paragraph{Comparisons to the relational model} In summary, the PartiQL
data model extended the SQL data model in the following ways:

\begin{compact_enum}
\item The elements of an array/bag can be any kind of value---not just tuples.
Furthermore they can be heterogeneous. That is, there are no restrictions
between the elements of an array/bag. For example, the two tuples in
\texttt{configurationItems} array of are \textit{heterogeneous} because: (i)
each tuple has a different set of attributes (for example, the second tuple has
\texttt{configurationStateId} while the first does not), (ii) an attribute of a
first tuple may map to some type while the same attribute in a second tuple may
map to another type.

\item More broadly, unlike SQL where the values are tables that have homogeneous
tuples that have scalars, PartiQL complex values are {\em arbitrary compositions
of arrays, bags and tuples}. E.g., the top level value of
Figure~\ref{figure:values:example-value} is a tuple and the
\texttt{configurationItems} array has two heterogeneous tuples. Note that in
this example, the top-level name refers to a value that is \textit{not} a bag
(e.g. a table).

\item There is a distinction between null-valued attributes and missing
attributes.

\yannis{OS: it carries over to array elements, right?
But it makes no sense to carryover to bags and thus we need to mark
the "cannot be MISSING) exception there also.}

\almann{MISSING in a bag and array is fine, the problem is in serializing such
data.}

\item PartiQL makes an explicit distinction between arrays and bags, where the
former have order and their elements can be addressed by ordinal position.
\footnote{Despite the ``SQL table is a bag'' and ``the results of an SQL query
is a table'' statements of SQL textbooks, SQL also recognizes that the result of
a query that has an \gl{ORDER BY} is a list, i.e., an ordered collection of
tuples.}
\end{compact_enum}

\begin{figure}[htb!]
\begin{lstlisting}
{
  'fileVersion':'1.0',
  'configurationItems':[
    {
        'configurationItemCaptureTime': `2016-08-03T08:56:52.415Z`,
        'resourceId':'h-0337bfe6793cf9e0c',
        'configuration':{
          'hostId':'h-0337bfe6793cf9e0c',
          'hostProperties':{
              'sockets':2,
              'cores':20,
              'totalVCpus':32,
              'instanceType':'m4.medium'
          },
        'tags':{
          'CostCenter':'Prod'
        },
    },
    {
        'configurationItemCaptureTime':`2016-08-03T09:41:56.906Z`,
        'resourceId':'h-0337bfe6793cf9e0c',
        'configurationStateId':3,
        'configuration':{
          'hostId':'h-0337bfe6793cf9e0c',
          'autoPlacement':'off',
          'hostProperties':{
              'sockets':2,
              'cores':20,
              'totalVCpus':32,
              'instanceType':'m3.medium'
          },
        'tags':{
        },
    }
  ]
}
\end{lstlisting}
\caption{An Example of a PartiQL Value}
\label{figure:values:example-value}
\end{figure}

\input{environment}
\input{paths}
\section{\from Clause Semantics}
\label{sec:from}

The formal semantics of a \from clause describe the collection of binding
tuples $B^{out}_{\from}$ that is output by the \from clause. The semantics
specify three cases and essentially extend the tuple calculus that underlies the
SQL semantics.

\begin{enumerate}
\item The semantics specify what is the core semantics of a \from clause with a
single \from item (Sections~\ref{sec:single-item-from} and ~\ref{sec:unpivot}).
The term ``semantics of the \from item $f$'' is synonymous to the term
``semantics of a \from clause with the single item $f$''. In either case, we
refer to the specification of the collection of binding tuples $B^{out}_{\from}$
that results from the evaluation of ``\from $f$''.

\item Then the semantics specify how multiple \from items combine, according to
the core semantics, using the join and outerjoin operations
(Sections~\ref{sec:combining-multiple-item-join},
\ref{sec:combining-multiple-item-leftjoin} and
~\ref{sec:combining-multiple-item-full-outerjoin}). 

\item Finally, the semantics specify the syntactic sugar structures that are
overlaid over the core semantics. Their primary purpose is SQL compatibility.
\end{enumerate}


\subsection{Ranging Over Bags and Arrays}
\label{sec:single-item-from}

Next we define the semantics of a \from clause that has a single \from item
and such item ranges over a bag or array. First consider the \from clause:

\begin{lstlisting}
FROM (*$a$*) AS (*$v$*) AT (*$p$*)
\end{lstlisting}

\noindent Let us call $v$ to be the \emph{element variable} and $p$ to be the
\emph{position variable}. In the normal case, $a$ is an array $[ e_0, \ldots,
e_{n-1} ]$. The \from clause outputs a bag of binding tuples. For each $e_i$,
the bag has a binding tuple $\langle v: e_i, p:i \rangle$.

\begin{example}
\label{xmpl:single-from-item-with-order}

Consider the following $\db$ (database environment):

\begin{lstlisting}
(*$\db = \langle$*)
    someOrderedTable:[
        {'a':0, 'b':0},
        {'a':1, 'b':1}
    ]
(*$\rangle$*)
\end{lstlisting}

\noindent then the following \from clause:

\begin{lstlisting}
FROM someOrderedTable AS x AT y
\end{lstlisting}

\noindent outputs the bag of binding tuples:

\begin{tabbing}
\ \ \ \=$B^{out}_{\from} = \ob $\=
    $\langle$ \lstinline|x:{'a':0, 'b':0}, y:0| $\rangle$\\
\>\>$\langle$ \lstinline|x:{'a':1, 'b':1}, y:1| $\rangle$\\
\>\>$\cb$
\end{tabbing}
\end{example}

As in SQL, the keyword \as is optional. The same applies to all cases below
where \as appears. If there is no \at clause, then the binding tuples have only
the element variable. In particular, consider:

\begin{lstlisting}
FROM (*$a$*) AS (*$v$*)
\end{lstlisting}

\noindent Normally $a$ is a collection, i.e, an array
$[ e_0, \ldots, e_{n-1} ]$ or a bag $\ob e_0, \ldots, e_{n-1} \cb$.
In either case, the \from clause outputs a bag. For each $e_i$, the bag
has a binding tuple $\langle v:e_i \rangle$.

\begin{example}
Consider again the database of Example~\ref{xmpl:single-from-item-with-order}
and then the \from clause

\begin{lstlisting}
FROM someOrderedTable AS x
\end{lstlisting}

\noindent this \from clause outputs:

\begin{tabbing}
\ \ \ \=$B^{out}_{\from} = \ob $\=
    $\langle$ \lstinline|x:{'a':0, 'b':0}| $\rangle$\\
\>\>$\langle$ \lstinline|x:{'a':1, 'b':1}| $\rangle$\\
\>\>$\cb$
\end{tabbing}
\end{example}

\subsubsection{Mistyping Cases}
\label{sec:bag-array-mistypings}

In the following cases the expression in the \from clause item has the wrong
type. Under the type checking option, all of these cases raise an error and the
query fails. Under the permissive option, the cases proceed as follows

\begin{itemize}
\item \highlight{Position variable on bags} Consider the clause:

\begin{lstlisting}
FROM (*$b$*) AS (*$v$*) AT (*$p$*)
\end{lstlisting}

\noindent and assume that $b$ is a bag $\ob e_0, \ldots, e_{n-1} \cb$. The
output is a bag with binding tuples $\langle v: e_i, p: \MISSING \rangle$. The
value \MISSING for the variable $p$ indicates that the order of elements in
the bag was meaningless. 

\item \highlight{Iteration over a scalar value} Consider the query:

\begin{lstlisting}
FROM (*$s$*) AS (*$v$*) AT (*$p$*)
\end{lstlisting}

\noindent or the query:

\begin{lstlisting}
FROM (*$s$*) AS (*$v$*)
\end{lstlisting}
 
\noindent where $s$ is a scalar value. Then $s$ coerces into the bag $\ob s
\cb$, i.e., the bag that has a single element, the $s$. The rest of the
semantics is identical to what happens when the lhs of the \from item is a bag.

\begin{example}
Consider again the database of Example~\ref{xmpl:single-from-item-with-order}
and the \from clause: 

\begin{lstlisting}
FROM someOrderedTable[0].a AS x
\end{lstlisting}

The expression \gl{someOrderedTable[0].a} evaluates to \gt{0} and,
consequently, the \from clause outputs a single binding tuple:

\begin{tabbing}
\ \ \ \=$B^{out}_{\from} = \ob
        \langle$ \lstinline|x:0| $\rangle \cb$\\
\end{tabbing}
\end{example}

\item \highlight{Iteration over a tuple value} Consider the query:

\begin{lstlisting}
FROM (*$t$*) AS (*$v$*) AT (*$p$*)
\end{lstlisting}

\noindent or the query:

\begin{lstlisting}
FROM (*$t$*) AS (*$v$*)
\end{lstlisting}
 
\noindent where $t$ is a tuple. Then $t$ coerces into the bag $\ob t \cb$

\item \highlight{Iteration over an absent value} Consider the query

\begin{lstlisting}
FROM (*$a$*) AS (*$v$*) AT (*$p$*)
\end{lstlisting}

\noindent or the query

\begin{lstlisting}
FROM (*$a$*) AS (*$v$*)
\end{lstlisting}

\noindent whereas $a$ evaluates into an \gn{absent\_value} (i.e., either
\MISSING or \NULL). In either case the \gn{absent\_value} $a$ coerces
into the bag $\ob a \cb$. Then the semantics follow the normal case.

\begin{example}
Consider again the database of Example~\ref{xmpl:single-from-item-with-order}
and the \from clause 

\begin{lstlisting}
FROM someOrderedTable[0].c AS x
\end{lstlisting}

The expression \gl{someOrderedTable[0].c} evaluates to \MISSING and,
consequently, the \from clause outputs the binding tuple:

\begin{tabbing}
\ \ \ \=$B^{out}_{\from} = \ob \langle$ \lstinline|x:MISSING| $\rangle \cb$\\
\end{tabbing}
\end{example}

\end{itemize}

\subsection{Ranging over Attribute-Value Pairs}
\label{sec:unpivot}

The \gl{UNPIVOT} clause enables ranging over the attribute-value pairs of a
tuple. The \from clause

\begin{lstlisting}
FROM UNPIVOT (*$t$*) AS (*$v$*) AT (*$a$*)
\end{lstlisting}

\noindent normally expects $t$ to be a tuple, with attribute/value pairs
$a1: v1, \ldots, a_n:v_n$. It does not matter whether the tuple is ordered
or unordered. The \from clause outputs the collection of binding tuples 

\[
    B^{out}_{\from} = \ob
        \langle v:v_1, a:a_1 \rangle 
        \ldots
        \langle v:v_n, a:a_n\rangle
    \cb 
\]

\begin{example}
Consider the $\db$:

\begin{lstlisting}
(*$\db = \langle$*)
    justATuple: {'amzn': 840.05, 'tdc': 31.06}
(*$\rangle$*)
\end{lstlisting}

\noindent The \from clause:

\begin{lstlisting}
FROM UNPIVOT justATuple AS price AT symbol
\end{lstlisting}

\noindent outputs:

\begin{tabbing}
\ \ \ \=$B^{out}_{\from} = \ob $\=
    $\langle $\lstinline|price: 840.05, symbol:'amzn'|$ \rangle$\\
\>\>$\langle $\lstinline|price: 31.06, symbol:'tdc'|$ \rangle$\\
\>\>$\cb$
\end{tabbing}
\end{example}

\subsubsection{Mistyping Cases}
\label{sec:unpivot-mistypings}

In the following cases the expression in the \from \unpivot clause item has the
``wrong" type, i.e., it is not a tuple. Under the type checking option, all of these cases raise an error
and the query fails. Under the permissive option, the cases proceed as follows:

\begin{lstlisting}
FROM UNPIVOT (*$x$*) AS (*$v$*) AT (*$n$*)
\end{lstlisting}

\noindent whereas $x$ is not a tuple and is not \MISSING, is equivalent to:

\begin{lstlisting}
FROM UNPIVOT {'_1': (*$x$*)} AS (*$v$*) AT (*$n$*)
\end{lstlisting}

\noindent Effectively, a tuple is generated for the non-tuple value.  When $x$ is \MISSING
then the above is equivalent to:

\begin{lstlisting}
FROM UNPIVOT {} AS (*$v$*) AT (*$n$*)
\end{lstlisting}

\noindent remember that a tuple cannot contain \MISSING. So the present case is equivalent to the empty tuple case.

\subsection{Combining Multiple \from Items with Comma, \CROSSJOIN, or \JOIN}
\label{sec:combining-multiple-item-join}

The \from clause expressions:

\begin{lstlisting}
(*$l$*) , (*$r$*) (*$\eqv$*)
(*$l$*) CROSS JOIN (*$r$*) (*$\eqv$*)
(*$l$*) JOIN (*$r$*) ON TRUE (*$\eqv$*)
\end{lstlisting}

\noindent have the same semantics. They combine the bag of bindings produced
from the \from item $l$ with the bag of binding tuples produced by the \from
item $r$, whereas the expression $r$ may utilize variables defined by $l$. Again, the term ``the semantics of $l\ \CROSSJOIN\ r$'' is equivalent
to the term ``the semantics of \from $l\ \CROSSJOIN\ r$''. In both cases, the
semantics specify a bag of binding tuples.

\paragraph{Associativity of $\CROSSJOIN$} We explain the $\CROSSJOIN$ and
``$,$'' as if they are left associative binary operators, despite the fact that
one can write more than two \from items without specifying grouping with
parenthesis. Since the ``$,$'' and $\CROSSJOIN$ operators are associative, we
may write (as is common in SQL):

\begin{lstlisting}
(*$f_1$*), (*$f_2$*), (*$f_3$*) (*$\eqv$*)
(*$f_1$*) CROSS JOIN (*$f_2$*) CROSS JOIN (*$f_3$*) (*$\eqv$*)
(*$f_1$*) JOIN (*$f_2$*) ON TRUE JOIN (*$f_3$*) ON TRUE (*$\eqv$*)
((*$f_1$*), (*$f_2$*)), (*$f_3$*) (*$\eqv$*)
((*$f_1$*) CROSS JOIN (*$f_2$*)) CROSS JOIN (*$f_3$*) (*$\eqv$*)
((*$f_1$*) JOIN (*$f_2$*) ON TRUE) JOIN (*$f_3$*) ON TRUE (*$\eqv$*)
\end{lstlisting}

\paragraph{Semantics} Consider the following:

\begin{lstlisting}
(*$l$*) CROSS JOIN (*r*)
\end{lstlisting}

\noindent unlike SQL, the rhs $r$ of the expression may use variables defined by
the lhs item $l$.  The result of this expression for a database environment
$\db$ and variables environment $\env$ is the bag of binding tuples produced by
the following pseudo-code. The pseudo-code uses the function $\evalf(\db, \env,
e)$ that evaluates the expression $e$ within the environments $\db$ and $\env$,
i.e., $\db, \env \vdash e \rightarrow \evalf(\db, \env, e)$.

\begin{tabbing}
\ \ \ \=for \=each binding tuple $b^l$ in $\evalf(\db, \env, l)$\\
\>\>for \=each binding $b^r$ in $\evalf(\db, (\env \| b^l), r)$\\
\>\>\>add $b^l \| b^r$ to the output bag
\end{tabbing}

In other words, the ``$l\ \CROSSJOIN\ r$'' outputs all binding tuples $b = b^l
\| b^r$, where $b^l \in \evalf(\db, \env, l)$ and $b^r \in \evalf(\db, (\env \|
b^l), r)$. The key extension to SQL is that $r$ is evaluated in the variables
environment $\env \| b^l$, i.e., it can use the variables that were defined by
$l$. The details of the variable scoping aspects are described in
Section~\ref{sec:scoping-variables}.

\begin{example}
This example simply reminds the tuple calculus explanation of the
\from SQL semantics. It does not yet endeavor into special aspects
of PartiQL. Consider the following database, which is conventional SQL:

\begin{lstlisting}
(*$\db = \langle$*)
    customers: [
        {'id': 5, 'name': 'Joe'},
        {'id': 7, 'name': 'Mary'}
    ],
    orders: [
        {'custId': 7, 'productId': 101},
        {'custId': 7, 'productId': 523}
    ]
(*$\rangle$*)
\end{lstlisting}

\noindent Then consider the following \from clause, which could be coming from
a conventional SQL query:

\begin{lstlisting}
FROM customers AS c, orders AS o
\end{lstlisting}

\noindent Note that in PartiQL this could also be written using the \CROSSJOIN
keyword, and presumably, one would put the sensible equality condition
\gt{c.id=o.custId} in the \gl{WHERE} clause. At any rate, this \from clause
outputs the bag of binding tuples:

\begin{tabbing}
\ \ \ \=$B^{out}_{\from} = \ob $\=
    $\langle$ \lstinline|c: {'id': 5, 'name': 'Joe'}, o: {'custId': 7, 'productId': 101}| $\rangle$\\
\>\>$\langle$ \lstinline|c: {'id': 5, 'name': 'Joe'}, o: {'custId': 7, 'productId': 523}| $\rangle$\\
\>\>$\langle$ \lstinline|c: {'id': 7, 'name': 'Mary'}, o: {'custId': 7, 'productId': 101}| $\rangle$\\
\>\>$\langle$ \lstinline|c: {'id': 7, 'name': 'Mary'}, o: {'custId': 7, 'productId': 523}| $\rangle$\\
\>\>$\cb$
\end{tabbing}

\end{example}

\noindent Due to scoping rules that will be justified and elaborated in
Section~\ref{sec:variable-scoping}, when the rhs of a $\CROSSJOIN$ is a path or a
function that uses a variable named $n$, such variable must be referred as
$\gt{@}n$. 

\begin{example}
Consider the database:

\begin{lstlisting}
(*$\db = \langle$*)
    sensors: [
        {'readings': [{'v': 1.3}, {'v': 2}]},
        {'readings': [{'v': 0.7}, {'v': 0.8}, {'v': 0.9}]}
    ]
(*$\rangle$*)
\end{lstlisting}

\noindent Intuitively, the following \from clause unnests the tuples that are
nested within the \gt{readings}.

\begin{lstlisting}
FROM sensors AS s, s.readings AS r
\end{lstlisting}

\begin{tabbing}
\ \ \ $B^{out}_{\from} = \ob$\=
  $\langle$ \lstinline|s: {'readings': [{'v': 1.3}, {'v': 2}]}, r: {v:1.3}| $\rangle$,\\
\>$\langle$ \lstinline|s: {'readings': [{'v': 1.3}, {'v': 2}]}, r: {v:2}| $\rangle$,\\
\>$\langle$ \lstinline|s: {'readings': [{'v': 0.7}, {'v': 0.8}, {'v': 0.9}]}, r: {'v':0.7}| $\rangle$,\\
\>$\langle$ \lstinline|s: {'readings': [{'v': 0.7}, {'v': 0.8}, {'v': 0.9}]}, r: {'v':0.8}| $\rangle$,\\
\>$\langle$ \lstinline|s: {'readings': [{'v': 0.7}, {'v': 0.8}, {'v': 0.9}]}, r: {'v':0.9}| $\rangle$,\\
\>$\cb$
\end{tabbing}
\end{example}

\subsection{Combining Multiple \from Items with \LEFTJOIN}
\label{sec:combining-multiple-item-leftjoin}

The \from clause expression:

\begin{lstlisting}
(*$l$*) LEFT CROSS JOIN (*$r$*) (*$\eqv$*)
(*$l$*) LEFT JOIN (*$r$*) ON TRUE
\end{lstlisting}

\noindent replicates SQL's \gl{LEFT JOIN} functionality and, in addition, it
also works for the case where the lhs of $r$ uses variables defined from $l$.

\yannis{OS: Do you really mean to make the CROSS necessary in the absence of ON?}

\almann{Yes, there is a parser problem if we don't, this was raised by Redshift
a while ago.}

Let's assume that the variables defined by $r$ are $v^r_1, \ldots, v^r_n$. The
result of evaluating $l\ \LEFTCJOIN\ r$ in environments $\db$ and $\env$ is the
bag of binding tuples produced by the following pseudocode, which also uses the
$\evalf$ function (See Section~\ref{sec:combining-multiple-item-join}). 

\begin{tabbing}
\ \ \ \=for \=each binding $b^l$ in $\evalf(\db, \env, l)$\\
\>\>$B^r \leftarrow \evalf(\db, (\env \| b^l), r)$\\
\>\>if \=$B^r$ is the empty bag\\
\>\>\>add $b^l \| \langle v^r_1:\NULL \ldots v^r_n:\NULL \rangle$ to the output bag \\
\>\>else\\
\>\>\>for \=each binding $b^r$ in $B^r$\\
\>\>\>\>add $b^l \| b^r$ to the output bag
\end{tabbing}

\begin{example}
Consider the database:

\begin{lstlisting}
(*$\db = \langle$*)
    sensors: [
        {'readings': [{'v':1.3}, {'v':2}]}
        {'readings': [{'v':0.7}, {'v':0.8}, {'v':0.9}]},
        {'readings': []}
      ]
(*$\rangle$*)
\end{lstlisting}

\noindent Notice that the value of the last tuple's \gt{readings} attribute is the empty
array. The following \from clause unnests the tuples that are nested within
the \gt{readings} but will also keep around the tuple with the empty
\gt{readings}. (See the last binding tuple.)

\begin{lstlisting}
FROM sensors AS s LEFT CROSS JOIN s.readings AS r
\end{lstlisting}

\begin{tabbing}
\ \ \ $B^{out}_{\from} = \ob$\=
  $\langle$ \lstinline|s: {'readings': [{'v':1.3}, {'v':2}]}, r: {'v':1.3}| $\rangle$,\\
\>$\langle$ \lstinline|s: {'readings': [{'v':1.3}, {'v':2}]}, r: {'v':2}| $\rangle$,\\
\>$\langle$ \lstinline|s: {'readings': [{'v':0.7}, {'v':0.8}, {'v':0.9}]}, r: {'v':0.7}| $\rangle$,\\
\>$\langle$ \lstinline|s: {'readings': [{'v':0.7}, {'v':0.8}, {'v':0.9}]}, r: {'v':0.8}| $\rangle$,\\
\>$\langle$ \lstinline|s: {'readings': [{'v':0.7}, {'v':0.8}, {'v':0.9}]}, r: {'v':0.9}| $\rangle$,\\
\>$\langle$ \lstinline|s: {'readings': []}, r: NULL| $\rangle$,\\
\>$\cb$
\end{tabbing}
\end{example}

\subsection{Combining Multiple \from Items with \FULLJOIN}
\label{sec:combining-multiple-item-full-outerjoin}

The \from clause expression:

\begin{lstlisting}
(*$l$*) FULL JOIN (*$r$*) ON *$c$*
\end{lstlisting}

\noindent replicates SQL's \gl{FULL JOIN} functionality. It
assumes that (alike SQL) the lhs of $r$ does not use variables defined from $l$. 
Thus, we do not discuss further.

\subsection{Expanding \JOIN and \LEFTJOIN with \on}
\label{sec:rewriting-on}

In compliance to SQL, the \JOIN and \LEFTJOIN have an optional \on
clause. The semantics of \gl{ON} can be explained as syntactic sugar over the core
PartiQL. They can also be explained by a simple extension of the semantics of
Sections~\ref{sec:combining-multiple-item-join},
~\ref{sec:combining-multiple-item-leftjoin}, and
~\ref{sec:combining-multiple-item-full-outerjoin}. The semantics of:

\begin{lstlisting}
(*$l$*) JOIN (*$r$*) ON (*$c$*)
\end{lstlisting}

\noindent are the following modification of the pseudocode of
Section~\ref{sec:combining-multiple-item-join}. The modification is the
inclusion of the underlined line.

\begin{tabbing}
for \=each binding $b^l$ in $\evalf(\db, \env, l)$\\
\>for \=each binding $b^r$ in $\evalf(\db, (\env \| b^l), r)$\\
\>\>\underline{if $\evalf(\db, (\env \| b^l \| b^r), c)$ is true}\\
\>\>\ \ \ add $b^l \| b^r$ to the output bag
\end{tabbing}

\noindent The semantics of:

\begin{lstlisting}
(*$l$*) LEFT JOIN (*$r$*) ON (*$c$*)
\end{lstlisting}

\noindent are the following modification of the pseudocode of
Section~\ref{sec:combining-multiple-item-leftjoin}. In essence, the \LEFTJOIN
\on outputs a tuple padded with \NULL whenever there is no binding of $r$
that satisfies the condition $c$.

\begin{tabbing}
for \=each binding $b^l$ in $\evalf(\db, \env, l)$\\
\>$B^r \leftarrow \evalf(\db, (\env \| b^l), r)$\\
\>$Q^r \leftarrow \ob \cb$\\
\>for \=each binding $b^r$ in $B^r$\\
\>\>if \=$\evalf(\db, (\env \| b^l \| b^r), c)$ is true\\
\>\>\>add $b^r$ in $Q^r$\\
\>if \=$Q^r$ is the empty bag\\
\>\>add $b^l \| \langle v^r_1:\NULL \ldots v^r_n:\NULL\rangle$ to the output bag \\
\>else\\
\>\>for \=each binding $b^r$ in $Q^r$\\
\>\>\>add $b^l \| b^r$ to the output bag
\end{tabbing}

\subsection{SQL's \gl{LATERAL}}
\label{sec:lateral}

SQL 2003 used the \gl{LATERAL} keyword to correlate \from clause items. In
the interest of compatibility with SQL, PartiQL also allows the use of the
keyword \gl{LATERAL}, though it does not do anything more than the comma itself
would do. That is ``$l\ \gl{, LATERAL}\ r$'' is equivalent to ``$l\ \gl{,}\
r$''.


\input{select}
\input{predsFunctions}
\input{where}
\input{subqueryCoercion}
\input{scoping}
\input{groupby}
\input{orderby}
\input{setops}
%\input{let}
\input{pivot}
\input{schema}
\end{document}
