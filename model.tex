\section{Data Model}
\label{sec:model}

\begin{figure}[ht!]
  \centering
  \rule[1ex]{\textwidth}{0.1pt}
  \begin{displaymath}
  \begin{grammar}
    \nt{value}  & \prod & \nt{absent}                                                                   & \\
                & \alt  & \nt{scalar}                                                                   & \\
                & \alt  & \nt{tuple}                                                                    & \\
                & \alt  & \nt{collection}                                                               & \\
    \nt{absent} & \prod & \terminal{\NULL}                                                              & \\
                & \alt  & \terminal{\MISSING}                                                           & \\
    \nt{scalar} & \prod & \ionlit{\ntsub{lit}{\mbox{Ion}}}                                              & \\
                & \alt  & \ntsub{lit}{\mbox{SQL}}                                                       & \\
    \nt{tuple}  & \prod & \pqltuple{\seplof{\pqlpair{\nt{string}}
                                                    {\nt{value}}}
                                           {,}}                                                         & \\
                &       & \begin{array}{r@{\ }l}
                            \mbox{where} & \nt{string} \mbox{ is any valid Ion or SQL string literal}\\
                            \mbox{and}   & \nt{value} \neq \terminal{\MISSING}  \\
                          \end{array}\\

     \nt{collection}            & \prod & \nt{array}                                           & \\
                                & \alt  & \nt{bag}                                             & \\
     \nt{array}                 & \prod & \pqlarray{\seplof{\nt{value}}{,}}                    & \\
     \nt{bag}                   & \prod & \pqlbag{\seplof{\nt{value}}{,}}                      & \\
     \ntsub{lit}{\mbox{Ion}} & \prod & \mbox{any valid Ion literal value~\cite{ion:spec}}   & \\
     \ntsub{lit}{\mbox{SQL}} & \prod & \mbox{any valid SQL literal value~\cite{sql92:spec}} & \\
  \end{grammar}
  \end{displaymath}
  \rule[1ex]{\textwidth}{0.1pt}
  \caption{BNF grammar for \pql values.}
  \label{fig:pgl-values-grammar}
\end{figure}

\begin{figure}[ht!]
\centering
\begin{tabular}{|r|lrl|}
\hline
 1  & \gn{value}                    & \gp   & \gn{absent\_value} \\
 2  &                               & \gd   & \gn{scalar\_value} \\
 3  &                               & \gd   & \gn{tuple\_value} \\
 4  &                               & \gd   & \gn{collection\_value} \\
 5  & \gn{absent\_value}            & \gp   & \NULL \\
 6  &                               & \gd   & \MISSING \\
 7  & \gn{scalar\_value}            & \gp   & \ionquote{\gs{ion\_literal}} \\
 8  &                               & \gd   & \gs{sql\_literal} \\
 9  & \gn{tuple\_value}             & \gp   & \gl{\{\ \}} \\
10  &                               & \gd   & \gl{\{} \gs{string\_value} \gl{:} \gn{value} (\gl{,} \gs{string\_value} \gl{:} \gn{value})* \gl{\}} \\
11  &                               &       & \ \ \ \ \ \ \ \gn{value} cannot be \MISSING \\
12  & \gn{collection\_value}        & \gp   & \gn{array\_value} \\
13  &                               & \gd   & \gn{bag\_value} \\
14  & \gn{array\_value}             & \gp   & \gl{[} (\gn{value} (\gl{,} \gn{value})*)? \gl{]} \\
15  & \gn{bag\_value}               & \gp   & \gl{\ob} (\gn{value} (\gl{,} \gn{value})*)? \gl{\cb} \\
\hline
\end{tabular}
\caption{BNF Grammar for PartiQL Values}
\label{figure:values:bnf}
\end{figure}

\newcommand{\linevalues}[1]{%
    \IfEqCase*{#1}{%
    {value}{BNF lines~1--4}%
    {missing}{(BNF line~6)}%
    {scalar}{(BNF lines~7--8)}%
    {tuple}{(BNF lines~9--11)}%
    {tuplemissing}{(BNF line~11)}
    {collection}{(BNF lines~12--13)}%
    }[\errmessage{Unable to ref #1 for value BNF}]%
}

Figure~\ref{figure:values:bnf} shows the BNF grammar for PartiQL values. A
PartiQL database generally contains one or more PartiQL \textit{named values}. A
\textit{name}, is an identifier, such as a table name, that is associated with a
PartiQL value.  Section~\ref{section:environment-and-sfw} defines what these
names are, and how SQL qualified names work, in detail.

The type of a value is \gn{absent}, \gn{scalar}, \gn{tuple} or \gn{collection}.
Further subtyping applies to scalars, tuples, and collections. We will often use
the name \emph{complex value} to refer to any non-scalar and non-absent value.
That is, complex values include \textit{tuples} and \textit{collections}. A
tuple is a set of attribute name/value pairs, where each name is a string (as in
SQL). A tuple in the data model is unordered. A conventional SQL tuple is an
ordered tuple since the schema dictates the order of the attributes and certain
SQL operations may use the order---support for this is described in detail in
Section~\ref{sec:schema}.

\yannis{OS: Next paragraph is odd. We give up on data model and talk about a known format (JSON)
and Ion (as a format). I recommend that we, instead, cast the Ion type system as PartiQL's
recommended extension over SQL. I'm not sure why JSON deserves a special mention.
}

\almann{How does the following look?}

PartiQL's data model extends SQL to Ion's type system to cover schema-less and
nested data. Such values can be directly quoted with \bt quotes.

Unlike SQL, PartiQL allows the possibility of duplicate attribute names, in the
interest of compatibility with non-strict JSON/Ion datasets. However PartiQL
does not encourage duplicate attribute names; navigation into tuples with the
conventional dot notation (Section~\ref{section:paths}) is tuned to the
assumption that the attribute names are unique.

A \gn{collection\_value} is either an ordered or unordered
\linevalues{collection}. Both arrays and bags may contain duplicate elements. An
array is ordered (similar to a JSON array or Ion list) and each element is
accessible by its ordinal position. (See specifics of access by position in
Section~\ref{section:paths}.) Arrays are delimited with \gl{[} \gl{]}. For
example, the value of the attribute \texttt{configurationItems} in
Figure~\ref{figure:values:example-value} is an array. Arrays have size, which
is not explicitly denoted but is implied by the number of elements in the array.
For example, the size of the \texttt{configurationItems} in
Figure~\ref{figure:values:example-value} is 2. The first element of an array
corresponds to index 0; the last element corresponds to index size minus one.

In contrast, a bag is unordered (similar to a SQL table) and its elements cannot
be accessed by ordinal position. Bags are denoted with \gt{\ob} and \gt{\cb}.

Finally, note that PartiQL has two kinds of absent values: \NULL and
\MISSING. The motivation is as follows: Unlike SQL, where a query that
refers to a non-existent attribute name is expected to fail during compilation,
in semi-structured data one expects a query to operate even if some of the tuples
do not define some of the attributes that the query's paths mention. Hence
PartiQL contains the special value \MISSING \linevalues{missing}, which is
the path result in cases where navigation fail to bind to any information. The
distinction between \MISSING and \NULL enables retaining the original
distinction between a missing attribute and a null-valued attribute. The utility
of \MISSING (as opposed to just having \NULL) will become further
apparent when navigation into semi-structured data and construction of
semi-structured results is discussed.

The value \MISSING may not appear as an attribute value. Notice that in the
interest of readability, the syntax of Figure~\ref{figure:values:bnf} does
exclude these cases; rather the ``\gn{value} cannot be \MISSING''
restrictions \linevalues{tuplemissing} indicate that \MISSING cannot appear
as an attribute value.

\paragraph{Comparisons to the relational model} In summary, the PartiQL
data model extended the SQL data model in the following ways:

\begin{compact_enum}
\item The elements of an array/bag can be any kind of value---not just tuples.
Furthermore they can be heterogeneous. That is, there are no restrictions
between the elements of an array/bag. For example, the two tuples in
\texttt{configurationItems} array of are \textit{heterogeneous} because: (i)
each tuple has a different set of attributes (for example, the second tuple has
\texttt{configurationStateId} while the first does not), (ii) an attribute of a
first tuple may map to some type while the same attribute in a second tuple may
map to another type.

\item More broadly, unlike SQL where the values are tables that have homogeneous
tuples that have scalars, PartiQL complex values are {\em arbitrary compositions
of arrays, bags and tuples}. E.g., the top level value of
Figure~\ref{figure:values:example-value} is a tuple and the
\texttt{configurationItems} array has two heterogeneous tuples. Note that in
this example, the top-level name refers to a value that is \textit{not} a bag
(e.g. a table).

\item There is a distinction between null-valued attributes and missing
attributes.

\yannis{OS: it carries over to array elements, right?
But it makes no sense to carryover to bags and thus we need to mark
the "cannot be MISSING) exception there also.}

\almann{MISSING in a bag and array is fine, the problem is in serializing such
data.}

\item PartiQL makes an explicit distinction between arrays and bags, where the
former have order and their elements can be addressed by ordinal position.
\footnote{Despite the ``SQL table is a bag'' and ``the results of an SQL query
is a table'' statements of SQL textbooks, SQL also recognizes that the result of
a query that has an \gl{ORDER BY} is a list, i.e., an ordered collection of
tuples.}
\end{compact_enum}

\begin{figure}[htb!]
\begin{lstlisting}
{
  'fileVersion':'1.0',
  'configurationItems':[
    {
        'configurationItemCaptureTime': `2016-08-03T08:56:52.415Z`,
        'resourceId':'h-0337bfe6793cf9e0c',
        'configuration':{
          'hostId':'h-0337bfe6793cf9e0c',
          'hostProperties':{
              'sockets':2,
              'cores':20,
              'totalVCpus':32,
              'instanceType':'m4.medium'
          },
        'tags':{
          'CostCenter':'Prod'
        },
    },
    {
        'configurationItemCaptureTime':`2016-08-03T09:41:56.906Z`,
        'resourceId':'h-0337bfe6793cf9e0c',
        'configurationStateId':3,
        'configuration':{
          'hostId':'h-0337bfe6793cf9e0c',
          'autoPlacement':'off',
          'hostProperties':{
              'sockets':2,
              'cores':20,
              'totalVCpus':32,
              'instanceType':'m3.medium'
          },
        'tags':{
        },
    }
  ]
}
\end{lstlisting}
\caption{An Example of a PartiQL Value}
\label{figure:values:example-value}
\end{figure}
